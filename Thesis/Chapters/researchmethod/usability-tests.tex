
\subsection{Purpose}
\label{sec:usabilitypurpose}
Usability tests are usually performed in order to detect errors in a system. We wanted to perform usability tests in a slightly different manner. In addition to discover errors, we observed how children take their medicine in combination with technology. We used the results of the usability tests to find potential for improvement and design ideas for  the product.
 
The tasks given to the participants were created with routine use of the application in mind. Usability tests were performed with the help of participants with no prior knowledge of the application. These participants were chosen in order to receive valuable feedback on usability problems with the current design and structure, and to prevent invalid feedback from users who already know how to perform the tasks. In addition, this situation resembled everyday life of the users.



\section{Test Method}
The execution of the usability tests were based on the theory described in Section \ref{sec:howtotestusability} and Section \ref{sec:usabilitytestchildren}.

Before each test, we performed a quick-and-dirty pilot test in order to discover critical errors that could make an impact on the result.

The participants were given an Android mobile device to perform tasks on. The different tasks were given one by one. The participants were introduced to the ``think-aloud''-method, and will be told to ask questions during the process, even though the test leader was not allowed to answer questions during the test. The main reason for gathering questions was for the discussion afterwards or in order to facilitate the ``think-aloud''-method. 

The test leader finished the test by asking questions regarding what the participants thought of the system and by answering the questions that were asked during the test. We also asked parents how they felt that the medication process went, so that they can compare it to their daily situation at home, and if they got the impression that the product could be helpful.  

The results were later analyzed in order to discover any improvements needed to the system. The errors were rated after level of severity\cite{dumas1995practical}. 

\begin{itemize}
\item{Critical (Level 1) - Prevents the participant from completing the task.}
\item{Siginificant (Level 2) - Generates significant problems when trying to complete the task.}
\item{Minor (Level 3) - Has minor effect on the usability of the application.}
\item{Non-essential (Level 4) - Enchancements to the system. When a participant states that ``it would be nice to have this''.}
\end{itemize}



\subsection{Scenario and tasks given to the users}
We planned to use test users that speak fluent Norwegian, since the application has Norwegian as it's main language. Therefore the scenario and tasks were also written in Norwegian. The exact scenario and tasks handed to the participants can be found in Appendix \ref{app:scenarioandtasks}, but for convenience the next paragraph gives a short summary of the scenario and tasks.

The scenario explained that the user was a parent of a 4-year-old child with asthma. They have recently seen a doctor, and will now have to set up treatment plans according to advice given by the specialist. Since they have little experience with asthma, they would have to look up information about the medicines and how the treatment will be done. In order to motivate the child to continue taking his/her medicine, they will have to add a reward via the application menu. Finally, they would have to look through the calendar log in order to find correlations between the child's health state and the use of medicines. 
