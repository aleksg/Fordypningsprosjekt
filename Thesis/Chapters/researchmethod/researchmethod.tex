\chapter{Research Method}
\label{chp:researchmethod}

\section{Finding Relevant Information}
\label{sec:literaturestudy}

We conducted a literature study early in the project, in order to get an overview of the research that is fundemental for our project. In order to find reliable research, we used Google Scholar\fnurl{Google Scholar}{scholar.google.com} and the IEEE Xplore Digital Library\fnurl{IEEE Xplore Digital Library}{http://ieeexplore.ieee.org/}. 

Initially, we found two key search phrases; ``Gamification'' and ``Tangible User Interfaces'', which were used to build a foundation of relevant material. Later, we used those terms in a combination with other relevant terms, eg. ``Health'', ``Asthma'', ``Development'', etc. We also twisted words to find related research fields, eg. replacing ``Gamification" with ``Game Elements''. 

In order to determine if an article was relevant to our project, we read through the papers' abstract and its' discussion and conclusions, before eventually reading the paper in detail. We then searched through the papers' references and skimmed their abstracts to potentially find new sources of information. 

To assess an article's validity, we used our judgement, based on a combination of whether an article had been quoted and a check for critiques or comments on the respective article. If the critiques were mainly negative, or pointed out serious deficiencies in the research performed, we decided not to put more effort into the article. In order to create a balanced viewpoint when discussing a specific theme, we tried to find contradicting arguments on the specific topic.  

Searches for information regarding tangible user interfaces were also performed in standard search engines (i.e. Google) in addition to Google Scholar. The reason behind this is that there exists few commercially available tangible user interfaces, which has been subject for a research article. Thus, it became hard to assess whether the project was a success on a larger scale. 
 
\section{Semi-structured Interviews}
\label{sec:semistructuredinterviews}

Semi structured interviews is a data collection method that allows interview subjects to put more weight upon their opinions, and what they perceive is important in a certain topic. As opposed to structured interviews, the interviewers can explore interesting answers more in depth by asking follow-up questions, instead of sticking to a fixed schedule. The interviewer comes up with a plan beforehand, with the main topics that should be covered during the interview, and some of the key questions that should be answered. If the interview stagnates within a specific topic, the interviewer can change the subject according to the plan.

The negative effects of this method is that it limits the creativity to our subjects. For instance, if people are initimidated or surprised by the research performed, getting the best possible feedback could become challenging. It is therefore encouraged to give a brief summary of the research field beforehand, so that the interview subjects are able to make up their own opinions about a topic before meeting for the interview.   

The purpose of conducting these interviews was to ensure that the end-product was not limited by our own imagination. We wanted feedback on the work we had done so far, in addition to exploring new functionality and elements we should keep in mind.

The interview subjects consisted of the following: 

Nanna S\o nnichsen Kayed, PhD/Researcher in Psychology.

Rose Lyngra, Senior Advisor at NAAF.

Marikken H\o iseth, PhD candidate in Industrial Design at NTNU. 

Two nurses with asthma within their field of expertise.

Two parents of children suffering from asthma.    


\section{Prototyping}
\label{sec:researchmethodprototyping}
A part of our research involves developing prototypes; one Android application and a tangible user interface. There are several reasons as for why we chose to include prototyping in our research; Firstly, it would be great if we were able to develop tools that help children get rid of their disease. This aspect provided a lot of motivation during the project. Secondly, we wanted to validate whether some of the literature we had studied were suitable for such a tool. Thirdly, we are creative people, who thrive when we do creative work. If we did not develop anything useful, we would easily have been burned out. Additionally, it is generally considered a good idea to develop prototypes in order to test concepts with potential users\iref{}.
      
\section{Usability Testing}
\label{sec:usabilitytesting}
\subsection{Introduction to Usability}
\label{sec:introductiontousability}

There are many ways to describe the term usability. 

The International Organization for Standardization(ISO) uses the following definition\cite{isousability}:

\textit{``Extent to which a system, product or service can be used by specified
users to achieve specified goals with effectiveness, efficiency
and satisfaction in a specified context of use.''}

The same document defines the term ``context of use'' as:

\textit{``Users, tasks, equipment (hardware, software and materials), and
the physical and social environments in which a product is used.''}

These definitions cover how the system is used, the user's thoughts about the use and the context of the system. This can be broken down further into several subgoals in order to achieve better usability, and to give a better insight as to what usability is. 
These subgoals are:

\begin{enumerate}
\item{How precisely is the user able to perform a task by using the application?}
\item{How much resources (for example time, or number of tries) was used to perform the given task using the application?}
\item{How many errors occurred?}
\item{Did the user find the use satisfactory?}
\end{enumerate}

Schneiderman stated eight golden rules in order to achieve a good usability of computer systems\cite{shneiderman2003designing}. In his rules he mentions consistency, informative feedback, reducing short term memory load and permitting easy reversal of actions. These eight golden rules have since their publication become a central part of usability engineering.

Today usability is extremely important in order to achieve success for a system. The users expect a functional and easy-to-use system. From the user's point of view, working with a product which is easily understood, leads to increased productivity, which again may lead to increased sales/usage\cite{folmer2004architecting}. Proper usability engineering may also lead to lower costs for the developers and higher chances for projects being finished on time\cite{nielsen1994usability}. 

\subsection{Purpose}
\label{sec:usabilitypurpose}
Usability tests are usually performed in order to detect errors in a system. We wanted to perform usability tests in a slightly different manner. In addition to discover errors, we observed how children take their medicine in combination with technology. We used the results of the usability tests to find potential for improvements and design ideas for the product, in addition to validate whether or not our concept worked in a satisfactory manner.
 
The tasks given to the participants were created with routine use of the application in mind. Usability tests were performed with the help of participants with no prior knowledge of the application. These participants were chosen in order to receive valuable feedback on usability problems with the current design and structure, and to prevent invalid feedback from users who already know how to perform the tasks. In addition, this situation resembled everyday life of the users.

As one of the systems we tested was an Android application, the question of whether we should test it on an emulator installed on a computer, or by the use of a smartphone was raised. While the emulator allows for easier screen and input capturing, it suffers from the drawbacks that interaction is performed through a mouse and a keyboard, in addition to running far slower than an Android device. Using an Andriod device would lead to a more realistic interaction pattern, but it suffers from the drawback that screen capturing is not possible unless the device is rooted\fnurl{Android rooting}{http://en.wikipedia.org/wiki/Android\_rooting}. We chose to test the system on an Android device, as we figured it would seem more natural for a child and the NSEP laboratory has capabilities to record the user interaction through cameras.  


\subsection{Test Method}
\label{sec:testmethod}

The usability tests were performed at the NSEP Usability Lab\fnurl{NSEP Usability Lab}{www.ntnu.no/nsep}, which provided measures for recording the sessions.  

Before each test, we performed a pilot test in order to discover last minute critical errors that could make an impact on the result.

The test was divided into two stages. In the first stage, the parent were to use \app{} for a couple of basic tasks. In the second stage, the child was to perform a set of tasks (the tasks can be found in Appendix \ref{app:scenarioandtasks}). We wanted children to observe while their parent performed the tasks, in order for them to understand that the process was harmless. Additionally, we let their parents sit next to them and explain the tasks for them, such that the children were not told to perform some seemingly random task by a stranger.    
 
The participants were given an Android mobile device to perform their tasks on. The different tasks were given one by one. The participants were introduced to the ``think-aloud''-method\cite{lewis1982using}, and was told to ask questions during the process, even though the test leader was not allowed to answer questions during the test. The main reason for gathering questions was for the discussion afterwards and facilitate for the ``think-aloud''-method. 

The test leader finished the test by asking questions regarding what the participants thought of the system and by answering the questions that were asked during the test. We also asked parents how they felt that the medication process went, so that they could compare it to their daily situation at home, and if they got the impression that the product could be helpful.  

The results were later analyzed in order to discover any improvements needed to the system. The errors were rated after level of severity\cite{dumas1995practical}. 

\begin{itemize}
\item{Critical (Level 1) - Prevents the participant from completing the task.}
\item{Significant (Level 2) - Generates significant problems when trying to complete the task.}
\item{Minor (Level 3) - Has minor effect on the usability of the application.}
\item{Non-essential (Level 4) - Enchancements to the system. When a participant states that ``it would be nice to have this''.}
\end{itemize}


An often used approach to measure the usability of a system, is to use the System Usability Scale (SUS)\cite{sus}, together with the observations made during the test. These scores may give an indication of the usability of a system\cite{susform}. As we were dealing with children, we had to use a different approach in order to get feedback from the children. 

Zaman et. al. proposes a way to measure the likeability of tangible interaction with preschoolers\cite{zaman2007measure}. They based their research on work done by Read, MacFarlane and Casey\cite{read2002endurability}, who found that traditional measures for likeability, for instance a smileyometer, proved to give false results. In fact, Read et. al. found that more than 80\% of the children being tested gave a ``Brilliant'' score. Zaman et. al. implies that children are actually lying when giving these scores, which is understandable from a questionnaire perspective. Instead of using scales as a measure of what is likeable or not, they propose a model where they compare different interaction systems against eachother. They call it the ``This or that'' method\cite{zaman2007measure}. For instance, the interviewer asks the child which system the child prefers, followed by ``this or that'' while pointing to the different systems. We used a similar approach to understand which forms of interaction children like most, and which system they liked to interact with the most, \app{} or \ab{}.   


\subsection{Scenario and Tasks Given to the Users}
\label{sec:scenarioandtasksgiventotheusers}
Since the test users has spoke Norwegian, and some of them were children, the scenario and tasks were given in Norwegian. A translation of the scenario and tasks handed to the participants can be found in Appendix \ref{app:scenarioandtasks}, but for convenience the next paragraphs gives a brief summary.

\textbf{Scenario given to adult test users}

The scenario explained that the user was a parent of a 4-year-old child with asthma. They have recently seen a doctor, and will now have to set up treatment plans according to advice given by the specialist. Since they have little experience with asthma, they would have to look up information about the medicines and how the treatment will be done. In order to motivate the child to continue taking his/her medicine, they will have to add a reward via the application menu. Finally, they would have to look through the calendar log in order to find correlations between the child's health state and the use of medicines. 


\textbf{Scenario given to child test users}

The scenario given to children were explained to them by their respective parents.

The scenario explained that the child needed to take his/her medicine, and that \app{} and \ab{} were there to help him/her through the process. The children started out by checking for potential rewards in \app{}. They then did a treatment with the help of \ab{}, before doing a treatment through \app{}. They were then told to check their amount of stars earned through \ab{}, before they were instructed to purchase their reward from \app{}'s shop.     
 