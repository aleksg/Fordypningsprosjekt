\chapter{Research Method}
\label{chp:researchmethod}

\section{Literature Study}
\label{sec:literaturestudy}

We conducted a literature study early in the project, in order to get an overview of the research that is fundemental for our project. In order to find reliable research, we used Google Scholar\fnurl{Google Scholar}{scholar.google.com} and the IEEE Xplore Digital Library\fnurl{IEEE Xplore Digital Library}{http://ieeexplore.ieee.org/}. 

Initially, we used two key search phrases; ``Gamification'' and ``Tangible User Interfaces'', which were used to build a foundation of relevant material. Later, we used those terms in a combination with other relevant terms, e.g. ``Health'', ``Asthma'', ``Development'', etc. We also twisted words to find related research fields, e.g. replacing ``Gamification" with ``Game Elements''. 

In order to determine if an article was relevant to our project, we read through the papers' abstract and its' discussion and conclusions, before eventually reading the paper in detail. We then searched through the papers' references and skimmed their abstracts to potentially find new sources of information. 

To assess an article's validity, we made an assessment, based on a combination of whether an article had been quoted and a check for critiques or comments on the respective article. If the critiques were mainly negative, or pointed out serious deficiencies in the research performed, we decided not to put more effort into the article. In order to create a balanced viewpoint when discussing a specific theme, we searched for contradicting arguments on the specific topic.  

Searches for information regarding tangible user interfaces were also performed in standard search engines (i.e. Google) in addition to Google Scholar. The reason behind this is that there exists few commercially available tangible user interfaces, which has been subject for a research article. Thus, it became hard to assess whether the project was a success on a larger scale. 
 
\section{Semi-structured Interviews}
\label{sec:semistructuredinterviews}

Semi-structured interviews is a data collection method that allows interview subjects to put more weight upon their opinions, and what they perceive as important on a certain topic. As opposed to structured interviews, the interviewers can explore interesting answers more in depth by asking follow-up questions, instead of sticking to a fixed schedule. The interviewer comes up with a plan beforehand, with the main topics that should be covered during the interview, and some of the key questions that should be answered. If the interview stagnates within a specific topic, the interviewer can change the subject according to the plan.

The negative effects of this method is that it limits the creativity to our subjects. For instance, if people are initimidated or surprised by the research performed, getting the best possible feedback could become challenging. It is therefore encouraged to give a brief summary of the research field beforehand, so that the interview subjects are able to make up their own opinions about a topic before meeting for the interview. \iref{}.

The purpose of conducting these interviews was to ensure that the end-product was not limited by our own imagination. We wanted feedback on the work we had done so far, in addition to exploring new functionality and elements we should keep in mind.

The interview subjects consisted of the following: 

Nanna S\o nnichsen Kayed, PhD/Researcher in Psychology.

Rose Lyngra, Senior Advisor at NAAF.

Marikken H\o iseth, PhD candidate in Industrial Design at NTNU. 

Two nurses with asthma within their field of expertise.

Two parents of children suffering from asthma.    


\section{Prototyping}
\label{sec:researchmethodprototyping}
A part of our research involves developing prototypes; one Android application and a tangible user interface. There are several reasons as for why we chose to include prototyping in our research; Firstly, it would be great if we were able to develop tools that help children get rid of their disease. This aspect provided a lot of motivation during the project. Secondly, we wanted to validate whether some of the literature we had studied were suitable for such a tool. Thirdly, we are creative people, who thrive when we do creative work. If we did not develop anything useful, we would easily have been burned out. Additionally, it is generally considered a good idea to develop prototypes in order to test concepts with potential users\iref{}.
      
\section{Usability Testing}
\label{sec:usabilitytesting}
\chapter{Usability Tests}
\label{chp:usabilitytests} 