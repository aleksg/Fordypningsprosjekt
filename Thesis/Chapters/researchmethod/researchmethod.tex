\chapter{Research Method}
\label{sec:researchmethod}

\section{Finding Relevant Information}
\label{sec:literaturestudy}

We conducted a literature review early in the project, in order to get an overview of the research that is fundemental for our project. In order to find reliable research, we used Google Scholar\fnurl{Google Scholar}{scholar.google.com}. 

Early on in the project, we discovered two key search phrases; ``Gamification'' and ``Tangible User Interfaces'', that were used in a combination with other relevant terms, eg. ``Health'', ``Asthma'', ``Development'', etc. We also twisted words to find related research fields, eg. ``Game elements'' instead of ``Gamification''. 
In order to discover if an article was relevant or not, we read through the papers' abstract and its' discussion and conclusions. We then searched through the papers' references and skimmed their abstracts to find new sources of potential information. Once we decided we had a decent overview around a specific topic, we narrowed our search down with more keywords.

To assess an article's validity, we used our judgement, based on a combination of whether an article had been quoted and a check for critiques and comments on the repective article. If the critiques were relatively negative, we decided not to put more effort into the article. 

Searches for information regarding tangible user interfaces were also performed in standard search engines (i.e. Google) in addition to Google Scholar. The reason behind this is that very few of the tangible user interfaces where research papers are published is actually commercialized. It then became hard to assess whether the project was actually a success, even though we could use some of their findings.      
 
\section{Semi structured Interviews}
\label{sec:semistructuredinterviews}

Semi structured interviews is a data collection method that allows interview subjects to put more weight upon their opinions, and what they perceive is important in a certain topic. As opposed to structured interviews, the interviewers can explore interesting answers more in depth by asking follow-up questions, instead of sticking to a fixed schedule. The interviewer comes up with a plan beforehand, with the main topics that should be covered during the interview, and some of the key questions that should be answered. If the interview stagnates within a specific topic, the interviewer can change the subject according to the plan.

The negative effects of this method is that it limits the creativity to our subjects. For instance, if people are initimidated or surprised by the research performed, getting the best possible feedback could become challenging. It is therefore encouraged to give a brief summary of the research field beforehand, so that the interview subjects are able to make up their own opinions about a topic before meeting for the interview.   

The purpose of conducting these interviews was to ensure that the end-product was not limited by our own imagination. We wanted feedback on the work we had done so far, in addition to exploring new functionality and elements we should keep in mind.

[TODO: NEED QUOTES AND MORE]

\section{Usability testing}
\label{sec:usabilitytesting}
\chapter{Usability Tests}
\label{chp:usabilitytests}

