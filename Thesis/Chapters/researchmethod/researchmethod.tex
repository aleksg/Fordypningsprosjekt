\chapter{Research Method}
\label{sec:researchmethod}


\section{Literature Study}

\section{Semistructured Interviews}

Semi structured interviews is a data collection method that allow interview subjects to put more weight upon their opinions, and what they perceive is important. As opposed to structured interviews, the interviewers can explore interesting answers in depth by asking follow-up questions, instead of sticking to a fixed schedule. The interviewer comes up with a plan beforehand, with the main research topics that should be covered. If the interview stagnates within a specific topic, the interviewer can change the subject according to the plan.

The negative effects of this method is that it limits the creativity to our subjects. For instance, if people are initimidated or surprised by the research performed, getting the best possible feedback could become challenging. It is therefore encouraged to give a brief summary of the research field beforehand, so that the interview subjects are able to make up their own opinions about a topic before meeting for the interview.   

The purpose of conducting these interviews was to ensure that the end-product was not limited by our own imagination. We wanted feedback on the work we had done so far, in addition to exploring new functionality and elements we should keep in mind.

[TODO: NEED QUOTES AND STUFF]

\section{Usability testing}
\chapter{Usability Tests}
\label{chp:usabilitytests}

