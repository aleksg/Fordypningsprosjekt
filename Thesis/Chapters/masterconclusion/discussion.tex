\subsection{Gamification}
\label{sec:gamificationresults}

\subsubsection{Understanding Children's Perception of Rewards}
\label{sec:understandingchildrensperceptionofrewards}

Our proposal for gamification elements contained within \app{} is tightly coupled with the parents' initiative. In order for our reward system to have any motivational effect, parents have to be closely involved. They need to understand how often their child needs a reward, in addition to understanding what defines a ``good'' reward for their children. 

Webster-Stratton and Herbert claims:

\textit{``Preschool children aged between the ages of three and four may be rewarded by the special sticker or token itself without needing a back-up reinforcer. Youngsters aged four to six should be able to trade in stickers for something each day if they like. Children of seven and eight can wait a few days before getting a reward.''}\cite{webster1994troubled}

In order for our systems to have a motivational effect, parents have to be aware of their children's maturity. Rewards that suit a three year old girl do not necessarily suit a six year old boy. Webster-Stratton further claims that parents should expand the efforts children have to put into a task, in order to receive their reward. We will elaborate more on this in Section \ref{sec:gamificationovertime}. 

Parents could be under the impression that the rewards given should be something material. However, according to Nanna S. Kayed, researcher/PhD in psychology: 

\textit{``Rewards do not have to be a material reward, it may be a fun activity or letting the children choose what they will eat for dinner. Doing something entertaining with their parents can be as much of a reward as a physical toy. An example of an easy and fun reward is to eat dinner underneath the table or taking a walk in the woods. It is important for you [the developers] to tell the parents that the reward does not need to be material, but can be simple and easy rewards.''}

She implies that spending quality time with the family can in fact be just as effective as material rewards. She also pointed out that we should include some sort of manual to our reward system, in order to maintain the maximum motivational effect and ensure that parents understand that the rewards do not have to be material. However, creating a manual for reward systems for children could be an own master thesis in psychology, and we will not delve further into this aspect. 

It is reasonable to assume that most parents have used some sort of reward system with their child, for instance to get their child to behave the way the parents want them to. One of our interview subjects had used stickers in order to toilet-train her son. The amount of rewards and their ``attractiveness'' should be correlated to the task performed by the child.

\textbf{``Collectors'' vs. ``spenders''}

Children perceive rewards in a different manner. Some children like to collect their rewards and will never choose to spend the rewards as a currency. To these children the collected amount of rewards is important, and they often value the rewards more as a collector's item than as a currency. 
Some children like to spend their rewards and cash them in as currency. These children often care more about spending the rewards on items rather than saving the rewards, even if the item they spend their reward on is a short-lasting joy, such as an edible item or an arcade credit.
\app{}'s reward system is made to accomodate the wishes of both ``collectors'' and ``spenders''. Since the reward system is a milestone-based system where the stars do not disappear when a reward is purchased, both user types may have what they want. Making the reward system as motivating as possible will be up to the parents.

\textbf{Understanding the family situation}

It may be argued that our reward system could cause internal jealousy within a family. For instance, if a family has three children, where two of them suffers from asthma, the last child could potentially become jealous of the other two, as they may receive rewards which seems unfair for the last child. In such cases, we leave the responsibility to the parents to balance these rewards, such that it does not seem unfair for the healthy children in their family.    

\subsubsection{Bartle's Four Player Types}
\label{sec:bartlesfourplayertypes}
In Section \ref{sec:bartlesplayertypes} we presented Bartle's four player types and how they enjoy gamification. In order to understand how gamification can be used in order to motivate children with asthma, we have explored some solutions for how to build an enjoyable gamification system for the four player types; achievers, explorers, killers and socializers. 

\textbf{Killers}

In Chapter \ref{sec:killers} we stated that hopefully no children in our target group hope to gamify their experience by imposing themselves on others. We found no positive and meaningful way to support ``killers'' in our system. The only scenario we could come up with was that children could slow down the progress for others (e.g. steal other children's stars), which is against the purpose of \app{}.  

\textbf{Socializers}

While AsthmAPP has little support for socializers, we believe there is potential for use of social features in a system such as AsthmAPP. An example of such would be the use of an avatar system where the children may share their avatar with others and meet other users of AsthmAPP in a social hub similar to Club Penguin\fnurl{Club Penguin}{www.clubpenguin.com} or Farmville\fnurl{Farmville}{www.farmville.com}. 


\textbf{Explorers}

AsthmAPP in its current state has little to explore. However, we believe that there is a potential for functionality to motivate explorers. The use of progress bars, leveling systems or achievements and badges can easily be transferred to a system like AsthmAPP in order to achieve a gamified experience targeted at explorers. Linander's interactive story concept showed how the progress in a story may be tied to the use of asthma medicine\cite{linander2013utvikling}. Possibilities for adding new stories over time would create an even more engaging system for children. 

\textbf{Achievers}

AsthmAPP mainly provides game elements suitable for achievers. The gamification system in \app{} is built around performing a treatment correctly and being rewarded for doing so. The use of stars as experience points and support for real-world rewards through the shop is applicable for all children, but may be of most interest to achievers. There is endless potential for how gamification may be used to engage achievers, and there are many possibilities for further research on these areas. 


\subsubsection{Reception of AsthmAPP's Reward System}
\label{sec:receptionofrewardsystem}
In order to get feedback on the gamification system of \app{}, we asked our test users and interviewees what they thought about our solution for gamification. Specifically regarding how the reward system racks up when the user is following the yellow and red treatment plan, we received interesting feedback.

\emph{``I think that there should be no differentiation in the rewards at all. One star per treatment should do, regardless if the child is in good or bad shape. In general a sick child would need to take more [typically blue] medicine anyway, resulting in more stars. If there is a multiplication factor in addition to the increased number of treatments the number of starts would go up quite quickly and from the psychological perspective it might make the child think it is a good thing to be ill.''}

We have not done enough research on our gamification system to make a judgement on how the stars should be rewarded. Due to the contradictory arguments, we are not in a position to make an assessment, and will advice further research on how the different treatment schemes should be linked to the amount of stars. 

\subsubsection{Gamification over time}
\label{sec:gamificationovertime}
According to Webster-Stratton and Herbert, parents have a tendency to not phase out reward systems\cite{webster1994troubled}. When this occurs, children do not receive the message that is in the essence of reward systems; that parents expect their child to perform a task on their own without receiving a reward. In the example Webster-Stratton \etal{} describe regarding raising a child, parents could give a reward for making the bed every day for one week. After a week, the parents should expand the task to include making his/her own breakfast every morning. Similarily, parents should increase the cost for receiving a reward in \app{}.  

Based on findings we did during Customer Driven Project\cite{CustomerDriven}, we believe that gaining access to a new star could be rewarding enough when a child first starts using AsthmAPP. After a while, it will become boring, and parents should provide a means of a tangible reward (material or social). For instance, parents could start out by giving an ice cream sandwich to their child if they take his/her medicine as planned during the first day. Then they could expand the challenge by one day, giving the child some extra allowance if they manage to do it two days in a row. When time passes and their child has gotten used to taking the medicine, the system should be phased out. \app{} in its current state would then serve the purpose of reminding, informing and logging the user's use of medicine.

\subsubsection{The use of different gamification mechanics}
\label{sec:gamificationinthefuture}
When building \app{} and \ab{} we chose to focus on the use of real-world rewards, mirroring user behaviour and experience points. While these were the elements we chose, there are endless possibilities for other combinations of game elements. In the following, we discuss the potential use of different game mechanics in the future.

\textbf{Avatar Systems}

The use of avatar systems has huge potential for gamifying the treatment for children suffering from asthma. An avatar system can be a simple game where the user is rewarded with clothing and equipment for their avatar, or too a extensive game where the users use of asthma medicine controls an avatar in a game. Linander's ``Concept for Improved Experience of the Treatment of Asthma''\cite{linander2013utvikling} showed potential for this.

\textbf{Achievements and Badges}

There are many possibilities for the use of achievements and badges. Examples of this is badges for ``Follow treatment plan for 5 consecutive days'' or ``''. The use of achievements are two-edged sword. They must not be of a manner that may lead to a non-positive behaviour, i.e. ``Not have an asthma attack for one week'' where the user may not want to use Ventoline, in order to win an achievement. 

\textbf{Real-World Rewards}

Our application is built around real-world rewards. We believe there are many possibilities for the use of real-world rewards when it comes to treating children with asthma. We also believe that having real-world rewards will motivate the children over time, since a candy bar, a trip to the local lake or a ticket to the local football match will not wither over time. 
Real-world rewards demands more from the users, since it demands that a parent or an adult gives the child the rewards. While this might help motivate the children it may also put too much work on the parents, and they may see the reward system as too demanding.

\textbf{Mirroring User Behaviour}

Mirroring user behaviour has recieved positive results from younger users, and there have been an increased amount of applications using this gamification technique. Applications such as the Grush toothbrush\fnurl{Grush}{https://www.indiegogo.com/projects/grush-the-gaming-toothbrush-for-kids\# home} is built around mirroring user behaviour. 
We find user behaviour a positive and useful technique.

\textbf{Leaderboards}

We had trouble finding how to implement leaderboards in an ethical and positive manner. Using a leaderboard risk breaking Norwegian privacy laws, which is unacceptable. Making the use of medicines into a competition would probably recieve heavy critique, since it may be viewed as a move to increase the income of the companies manufacturing medicine. While the use of anonymous avatars and usernames may combat the privacy concerns, it may still be viewed as unethical and a bad marketing scheme.  

\textbf{Social Networking}

One should be very careful when designing social networks for people who suffer from a disease. There are many privacy concerns to take into account. An anonymous social network may be positive for parents with asthma. They may share success stories, ask questions and recieve help through the network. 

\textbf{Progress Bar}

There are possibilities for the use of a progress bar within an application for health care. While it will be impossible to evaluate the progress towards being cured from a disease, there are other uses. Combining a progress bar with experience points is easy to implement and have many possibilities for how developer wants to make use of the gamification element. 

\textbf{Experience Points}

As mentioned previously, experience points there are many possibilities for the use of experience points. The main challenge with using experience points is to make them have a meaningful value over time. As McGonigal argues, gamification withers over time and there is risk for boring the user\cite{jane2011reality}.  

\textbf{Contests}

As with leaderboards, making the use of medicines has it's risks. One might think of contests such as ``remember to follow treatment plan perfectly for a long period of time'' as a suitable contest, since it has a positive competition. However, this will be in conflict with Norwegian privacy laws. 

\subsection{Tangible User Interfaces}
\label{sec:resultstui}
During our project we discovered different areas where tangible user interfaces may be of use for asthmatic children, their parents and other caregivers. The main areas include learning, motivating, distracting and informing. Other areas where TUIs could be of use are elaborated on in Section \ref{sec:otheraspects}.


\subsubsection{TUI as a Tool for Learning}
\label{sec:tuiasatoolforlearning}

When a child gets diagnosed with asthma, his/her parents receive a lot of information at the doctors. It occurs that the parents are not paying attention, do not understand the information received, or do not communicate the information correctly to other caregivers. As two asthma nurses stated in an interview: 

\textit{``We always make sure to teach the parents and the children how to apply the medication correctly, however, they may forget it over time. If the parents do not remember how and when to give the children medications, it may have a negative effect on the treatment of the child's asthma.''}

\buddy{} can be used to relieve parents from the responsibility of remembering exactly how and when the medicine should be taken. Parents will probably remember the process after a couple of days, but \buddy{} could help them get started. 

Hospitals hand out flyers and treatment schemes to parents when children are diagnosed with asthma. However, flyers are often lost, and miscommunication often occurs when parents leave their child with other caregivers, like grandparents, babysitter, etc. By sending the child off together with \buddy{}, the implications of any miscommunication could be minimized.    

\textbf{Teaching children}

In addition to teaching parents about asthma, \buddy{} could be used to teach children about their disease. In some cases, parents do not explain to their children why they are sick, and what is causing their breathing problems. After a child has taken a dosage, \buddy{} could proceed to read a book about asthma, specifically written for children. As more treatments have gone by, new chapters can be read, increasing children's knowledge and awareness of their disease.   

\textbf{Information correctness}

When informing children and parents about asthma, it is vital that they receive correct information. If the information provided contains errors, it could have significant consequences for the asthma treatment. The information provided should be approved by either medical doctors or organizations like the NAAF. This will provide a quality assurance, in addition to gaining potential families' trust.    

\textbf{Motivating}

One of the nurses we interviewed, stated the following: 

\textit{``Children below your target group (i.e. younger than 3 years old) can be even harder to motivate, as children in the group 3 - 5 years old have an understanding as to why they need to take their medicine.''}

Children in the age of 3 - 5 years old understand that they get better from taking their medicine. However, not all parents actually tell their child specifically what is wrong with them [TODO: Sitat?]. \buddy{} could have been used to inform children about what happens with their lungs before and after they take their medicine.
Some children may get a better understanding of their disease and therefore understand better why they need to remember their medication. There is always a risk that children will be scared when told what their asthma is doing to their lungs. The parents must be aware of how worrisome their child is. 


\subsubsection{TUI as a Tool for Motivating}
\label{sec:tuiasatoolformotivating}

\textbf{Feedback}

Tangible user interfaces could be used to give feedback to children about their treatment. They could for instance notify how good they have been at taking their medicine during a week, and notify if they have forgotten a medicine at a day. This feedback should however be discrete and implemented in a non-obtrusive manner, as parents could interpret \ab{} as yet another annoying toy in the house.   


\textbf{Calming children down before a treatment}

If children are scared before they take their medicine, tangible user interfaces could help calming children down. A friendly character such as a teddy bear might help distract the children from the situation and make them forget about what scared them in the first place. 


\textbf{Turn the process into a game}

TUIs could be used to motivate children by turning each treatment into a game. At the moment, \ab{} and \app{} makes the long run process into a game. A TUI could be used to turn a single treatment into a game, e.g. by saying that \textit{``If you take the cap off the medicine, you get ten points''} and \textit{``If you breathe for ten seconds, you get a hundred points''}. Then children could do the math to calculate their total sum, that the TUI can state is correct or incorrect. 


\textbf{Responsibility}

A TUI could be used to make children more responsible regarding their own disease. Children could for instance be prompted to check whether there is enough medicine in their inhaler, and be responsible for telling their parents if they need new supply. \ab{} could be used for the same purpose by waiting to trigger an alarm, and check to see if the child reacts upon this. If the child reacts, it could provide positive feedback, and if the child does not react, it could give a strict notice that they can't rely entirely on a stuffed toy animal\footnote{This could prove to have a negative effect, but is an option to give children more responsibility.}.  
%TODO: Should we write this somewhere
%Children often become dependent on certain artefacts to accomplish a task. For instance, some children are not able to go to sleep without their favorite stuffed animal. A similar dependency could occur when taking their medicine with \ab{}. 

\subsubsection{TUI as a Tool for Distracting}
\label{sec:tuiasatoolfordistracting}

\textbf{Distracting children while taking a medicine}
In its current state, \ab{} distracts children while taking their medicine by counting to 10, which is the number of seconds a child should breathe in his/her breathing chamber. Additionally, the LED-light at its nose is blinking, which we will discuss further in Section \ref{sec:dosanddontsfortui}. 

Since we only have 10 seconds to work with, there are big limitations on what \ab{} can actually do in order to distract them in a natural way. It could be argued that having \ab{} by their side is actually enough distraction, as they have something to look on. If a TUI with movable parts had been developed, these parts could be used to give children something else to look on, by for instance introducing them to a new dance once in while.        

A more interesting case would be to distract children that are using a nebulizer in their treatment, as these often lasts for about 10 minutes or more. Children would then have to rely on something more exciting, like an audio book. 

\textbf{Distracting children between medicines}

Children often have to take two different medicines after eachother, but not immediately. A reguler scheme is to first take Ventoline, then wait for five minutes, before a dosage of Flutide should be taken. One of the interview subjects pointed out that: 

\textit{``The teddybear could help children to keep up with the time while they're waiting to for the next dosage.''}

He further noted that it was easy to get ``overly excited'' and take the Flutide dosage earlier than 5 minutes after Ventoline has been taken, which will reduce the effect of Ventoline. \ab{} could in this case distract the child from taking the medicine too early, by for instance reading an audio book, playing songs or even count down to the next dosage. The effect this distraction could vary from child to child. If a child really hates to take his/her medicine, having \ab{} count down to the next dosage could seem frightening for the child.   
  

\subsubsection{TUI as a Tool for Informing}
\label{sec:tuiasatoolforinforming}
For children with astma there is a lot to remember. We have already discussed the theme of learning about asthma, which is a very important aspect. There is still more to remember, and we have found some ideas for what the tangible user interface may help for. 

\textbf{Counting doses left in the inhaler}

One of parents we interviewed noted that: 

\textit{``It is annoying when medicines go empty, so we're keeping a journal''}

Since the inhalers have a certain amount of doses in them, and the amount of doses varies between different medicines and their vendors, it may require an effort to remember how many doses are left. The disk-formed medicine often come with an indicator, while inhalers do not. Inhalers make a ``poof'' sound when pressed, and this sound may occur regardless of whether or not there are any doses left. By using a RFID on the medicine and a RFID-reader on the TUI, may keep up with how many doses have been taken, and the system automatically warns the user when the number of doses is running low. There are many other ways to solve this digitally, but we believe that having \buddy{} count how many doses are left and  tell the user by sound could be fun for children and helpful for parents.


\textbf{Pollution levels and pollen forecast}

There are many different applications and web pages for reading and recieving information about air quality and pollen forecast. These web sites often offer data to third party services. AsthmAPP has the functionality for getting air quality readings and pollen forecast in the same application (see Section \ref{sec:description-medicine-log}).
\buddy{} could help the children with the same functionality, starting the day by telling how the air quality and pollen readings are for the day. 

In a future version, \buddy{} could calculate and foresee the amount of medicine needed when there is much pollen or bad air quality and update its alarm schedule based on the knowledge gathered. However, this is a functionality which will require further research.


\textbf{Reminders}

\buddy{} has functionality for firing alarms based on the treatment plan set in AsthmAPP. This makes it easier for the children and parents to remember to apply the treatment at the planned time. While taking medicine is often a routine for families with asthmatic children, \buddy{} could be a useful embodyment of a reminder. 

In a future version \buddy{} should be able to know that if a treatment has been completed within a short time before a planned treatment, the planned treatment will not be necessary. For a user having completed their planned treatment, an alarm firing 5 minutes later can be annoying. \buddy{}'s purpose is to help the user to remember, and if the user has already completed the treatment, \ab{} should be satisfied. 


\subsection{Other Aspects}
\label{sec:otheraspects}
During the project we discovered different areas where \ab{} could be of use, some of these did not sort under the topics learning, motivating, distracting and informing. These findings are presented in this subsection.


\subsubsection{Helping Kindergartens, Schools and Caregivers}
\label{sec:helpingkindergartenschoolandcaregivers}
During our interviews, we discovered a potential problem when children are in the kindergarten or at pre-school. It may occur that the caregivers do not know how to handle an asthma attack properly. According to one of our interview subjects: 

\textit{``The biggest problem is that the teachers/kindergarten teacher may not have knowledge of what to do when an asthma attack occurs. An application with instructions may be of help to them.''}

A solution to this problem was provided by a kindergarten teacher we interviewed, who said that it was hard to keep track of which child was supposed to take his/her medicine at the correct time. They sometimes had this information stored on their own phone, or had a note in their pocket. In some cases, no such tool were used, which relies heavily on the teachers' memory. If the teacher forgets it, there is a possibility that the child does not take their medicine properly on the given day. 
The kindergarten teacher proposed an application that allowed parents to register a medicine that was to be taken, and sent push notifications to the teachers, that could remind them of their child's need for a treatment. We concluded that this functionality is out of the scope in this thesis, but we found the idea interesting.         

In our opinion, having a shared \buddy{} in a kindergarten could lead to complexity and problems. First, \buddy{} would have to learn the names of the children in order to keep track of children whose turn it is. Second, there could be no overlapping of treatments, which might become inefficient (depending on the teachers). Third, having a shared \buddy{} in a kindergarten could easily be destroyed. If placed in a kindergarten, \buddy{} in its current state would probably cause more problems rather than helping the kindergarten teachers. With changes and modifications, we still see the use for tangible interfaces in kindergartens and preschools, as a useful tool to help teachers and children.    


\subsubsection{Tangible Interfaces to Help Parents Help Children}
\label{sec:tuitohelpparentshelpchildren}
When children suffer from asthma, they often have to rely on their parents in order to maintain control of the disease. Parents have to maintain a clean house and they have to keep an eye on pollen, as pollen and asthma are often related. One of the features \buddy{} could have in order to help parents is a morning forecast, informing parents about the weather, pollen distributions and air quality. 

In the future, \buddy{} could communicate with dust sensors, that could indicate whether or not parents actually needed to clean the house. Additionally, \buddy{} could communicate with a Roomba \fnurl{iRobot Roomba}{http://www.irobot.com/us/}, which in turn could start cleaning. \buddy{} could also indicate the air humidity at the child room, starting up the air condition.   


\section{Do's and Don'ts when Using a TUI}
\label{sec:dosanddontsfortui}

\textbf{Mobility}

When developing a TUI for children it is important that the TUI is mobile. Children become attached to their toys and like to take them with them. To make the most out of a tool such as \buddy{} it is important that the children may take it with them. The problem of power usage may be solved by a battery. The problem of recharging can be solved by charging at night. It is also possible to buy a WiFi shield for the \rpi{} that could handle internet connection. This would solve mobility within a home. However, it would not solve the problem when a child is travelling, for instance by car. Being able to use \ab{} in a car would require their parents to have created a hotspot.   

\textbf{Use of colored lights}

With \buddy{} we tried the use of LED lights to make \buddy{} more interesting than a normal teddy bear. During a treatment the LED light would indicate which medicine was supposed to be taken, by beaming lights in the same color, blink to count the seconds when breathing and using red light to indicate the seriousness of having to find an adult to overview the process. 

[TODO: Paavirket det pustingen til barn?]
[TODO: Ble barna forvirret?]

One of our interview subjects, a PhD candidate of product design stated in an interview: 

\textit{``People's perception of and preference for sensory stimuli differs. The use of lights and sound may affect the children in different ways, but that will have to be explored in user studies.''} 

We believe that the use of colored lights as a distraction method during a treatment is a complete field of research on it's own. There already exists some research, such as M\ae hlum's ``Concept for Increased User Acceptability When Treating Children With Respiratory Infections\cite{mahlum2013}''. During our project, we have not conducted enough research to draw conclusions on the use of colored lights during treatments. 


\textbf{Interaction Methods}

\buddy{} in its prototype form was not able to sense interaction, and was operated by using a ``Wizard-of-Oz technique''\cite{wilson1988rapid}. \buddy{} was not able to give feedback that the user did interaction correctly. The only form of confirmation was that the next sound clip would start playing. This may lead to confusion and uncertainties among users as to whether they interact correctly. 
