\chapter{Introduction}
\label{chp:introduction}

This chapter will give an introduction to the study. It will state the purpose, motivation, research questions and the research method for this study. 

\section{Purpose}
\label{sec:purpose}
The goal of this study is to evaluate the CAPP, GAPP and Karotz Applications created by Aaberg, Aarseth, Dale, Gisvold and Svalestuen \cite{CustomerDriven}.
The evaluation will be done through usability testing carried out on all three applications. The results of these initial tests will thereafter be used to improve the applications for a newer version. 
We will also plan a thorough testing of the applications.


\section{Motivation}
\label{sec:motivation}
According to NAAF, 20\% \cite{NAAF} of the Norwegian population has or has had asthma at the age of 10, and 8\% of the adult population suffers from asthma. Many of the children find it unpleasant to use their medicine as they often do not understand why the medicine must be taken [Should have a reference]. This may result in parents applying the medication incorrectly, applying the wrong treatment, or even forgetting to give the medicine to their children. 


\section{Research Questions}
\label{sec:researchquestions}
The main goal for this study is to evaluate the CAPP, GAPP and Karotz application, and identify the usability problems in these systems. Structuring the goal into different research questions will help this study with the evaluation of the goal. The goal has been composed into these questions:

\paragraph{RQ1:}
\textbf{How will guardians of a child react on having a Karotz constantly ``watching'' over their child?}


\paragraph{RQ2:}
\textbf{What are the usability problems of the current system?}


\paragraph{RQ3:}
\textbf{Will the physicians benefit from having detailed logs and information sent by email?}

This evaluation should be done through user testing and feedback from future users of the applications. The testing will give information on how well the 
\ldots

\section{Research Method}
\label{sec:researchmethod}


