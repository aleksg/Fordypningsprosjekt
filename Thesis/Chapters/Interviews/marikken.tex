\textbf{PHd candidate in industrial design.} 

\textbf{Date: 26th of March, 2014}

\textbf{Place: Trondheim}

\textbf{[Interview started with a demonstration of AsthmAPP and AsthmaBuddy]}

It's a good idea that the stars represent something different than the digital picture itself. The stars by itself may lose value after a short period of time. Children and parents can decide the rewards together. This could become a social activity for them. 
It seems smart to guide them as far as the rewards goes, and it does not necessarily need to be material. It is entirely up to the user to choose the amount of effort they'll put in it, and they decide how they reward their children. It is easy to forget that stuff like deciding where the sunday trip goes could be rewarding enough. It is important not to limit these rewards. 

\emph{Do you have any experiences as far as repetivity goes for this usage pattern?}

It is a lot of difference between children in the different age groups. It is important that the application is an aid to help them, and not something they're forced to use. The application could help during the startup of their treatment. Then parents can decide whether or not they should continue using it. It is important to creative a positive frame around the treatments. In the best case scenario, the child understands why they use their medicines. The application could help by creating a better attitude around the use of medicines. 

\emph{Do you have other ideas to what AsthmaBuddy can be used for, besides guiding them through their treatments?}

It is important not to exaggerate the amount of functions it can have. It will become annoying if AsthmaBuddy required a lot of attentions. We had a case where we had to put Furby into a closet because it become to needy. It is important that the user remains in control, and that AsthmaBuddy is subtle. It depends on how AsthmaBuddy is presented and introduced. The user has to be able to control the tool. It is important to include the end user in the development. So that we (designers and developers) can learn about them and make products that are adapted to their lives.   

\emph{Do you have any ideas regarding the interaction methods between the child and AsthmaBuddy?}

It is useful to build a relation between AsthmaBuddy and the child. It is important to show what can be done in order to get a technological response. It is also important to reduce interactive touch points. As far as ideas go, maybe ability to speak with AsthmaBuddy could be an idea?

\emph{Regarding use of lights/sounds/pictures in applications targeted towards asthmatic children. Do you have any experiences or do's and don'ts we should know of?}

People's perception of and preference for sensory stimuli differs. You should take a look at Tori's master thesis\footnote{DIPP - Utvikling for konsept for \o kt brukeraksept innen medisinsk behandling av barn med luftveisinfeksjoner}. The use of lights and sound may affect the children in different ways, but that will have to be explored in user studies. 


\emph{In AsthmAPP we use pictures of the Karotz as an avatar. What are your thoughts on this?}

Have you considered using pictures of the bear, in order to create a connection between the application and the bear? It is preferable to have a relation between icons in the application and the look of the bear. If the bear is not present at all times, the bear within the application could be the little brother to the big bear to create a tighter relation. 

\textbf{[Demonstration of AsthmaBuddy in use]}

There is no feedback that the bear knows you have interacted?

\emph{The bear will continue talking after the user interacts.}

That may not be enough feedback for the child. It is smart to show the child that he/she interacts in the right way. You should also look for how the color of the light affects the user. The change of color of the light may be confusing, but that will have to be user tested. 

You should remember that interaction between the bear and the user can be difficult when taking taking a medicine. The user only has two hands. 

The teddy bear could help with other things such as remembering to brush teeth for a long enough time or similar activities that may be boring for children.

