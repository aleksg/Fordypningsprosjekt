\textbf{Senior advisor at NAAF.}

\textbf{Date: 26th of March, 2014}

\textbf{Place: Trondheim and Oslo \footnote{Performed over the phone}}

\textbf{[The interview started with an explanation of AsthmAPP's reward system]}

\emph{Are there any aids that are used on a regular basis?}

At the hospital or during a treatment, there are a lot of people who read for the child, or play an audio book, use an iPad to watch a movie or play an easy game. In order to distract the youngest children, they sometimes use a toy (something with sound or light that are ``active''). Having something interesting to distract the children usually works good. These tactics are also used for children with eczema, where they have to sit in a bath for 15-20 minutes.
During shorter treatments, the children usually gets a reward after the treatment, as the treatment is too short to distract them from anything.

\emph{Do parents remember to apply the medicine at the right time? And do they use any tools in order to help them keep up with the plan?}

Usually, they medicine are given at the morning or by evening. Some children need medicines more often, and this must be explained to employees at the kindergarten or at the school. Children often have to take responsibility, as they can't necessarily calll their parents to check. It is possible the employees use this (tools), but I'm not sure about the security. There are a lot of temporary employees in the kindergartens, and they do not necessarily have the same overview as the full time employees. A tool specifically targeted at temporary employees could be useful. 

\emph{Do parents use forecasts for pollen and air quality?}

NAAF has an app for pollen casts. I consider it useful to gather more information in one application. 

\emph{Do you see any problems regarding logkeeping of medicines?}

Is it only the parents who have access to the data?
\emph{Yes, but the idea is that other caregivers could use the application.}
You should ask the Norwegian Data Protection Authority in order to be sure. 

I recommend you checking out the application ``MinAstma''.

\emph{Regarding ACT-test, is this something the child would benefit from showing to the doctor?}

Some take ACT tests at home and bring them to the doctor, these results are useful for the consultation.

\emph{Will the doctor/nurse consulting the child have any benefit from having detailed logs of the use of medicine?}

If it is a log showing the use over a period of time, it will be very useful. ``The dream'' is to have a complete record of what medicines were taken and the health state of the child. Sending such logs to the doctor ahead of an appointment will be useful. I personally believe this will become standard in the future, when the tools for easy log keeping have become widespread.

\emph{The children receive more stars if they are following the yellow or red treatment plan. Do you think children will exploit this?} 

Yes, I believe children may want to trick the system. Children are sneaky, and all children are different. It will be important to talk to parents in order to find a suitable reward system that will work over time. 

\textbf{[Oral explanation of AsthmaBuddy]}

\emph{Do you have ideas for what else our system can do, such as measure the indoor climate?}
The indoor climate is a combination of many factors, how are you planning to measure all the different factors? It is problematic to measure all the parameters in order to achieve a complete overview of the situation. 

Children often care very much about their stuffed toys. They often play with them and carry them with them wherever they go. To parents it will be important to be able to wash the stuffed toys, since it get dirty over time. Parents to children suffering from asthma often wash stuffed toys even more often. I would advice having a teddy bear with some durability.

The following question was asked via email at a later time.

\emph{Do parents explain to their children what is actually happening to them and why asthma attacks occur?}

Yes, most parents explain it for them, but it depends on whether they have received enough information and whether they (the parents) understand it.

Health care personell is responsible for making sure that parents have reliable information, in addition to explaining children about it if necessary. However, mistakes do occur, and as such, having an application with more information is a a good idea.   
