[WAIT FOR QUOTE CHECK]

\textbf{Father of a 6 year old girl suffering from asthma.}

\textbf{Date: 1st of April, 2014}

\textbf{Place: Trondheim}

\emph{Do you use the traffic light scheme?}
No, I use a treatment scheme instead, but the details are essentially the same as the traffic light scheme.
Are you under the assumption that the child don't want to take the medicine?
\emph{Yes. If not, everything is fine, and the application will be obsolete.}
She had difficulties during the first couple of weeks. After a while, she realized that she became healthier by taking her medicine. 

\emph{The reward system works by receiving stars after a treatment. It is up to the parents to choose a suitable reward (which implies that it does not necessarily need to be materialistic rewards)}

It is annoying when medicines go empty, so we're keeping a journal. The different medicines has a set amount of doses. The application could give a warning/notification that a medicine will soon be empty. 

\textbf{[Demonstrating AsthmaBuddy]} 

RFID-tags attached to the medicine could be useful way of interacting with AsthmaBuddy. 
Does your application require that children have their own smartphone?
\emph{No. Parents should have this application installed on their own phone.} 
I think it is custom for children to have their own smartphone. By giving the application to children, it could give them a sense of responsibility. 

\emph{Has it been a problem to get your child to take her medicine?}

She was 4 years old when she first started. It was a problem then, as she was scared of taking her medicine. It took about a week before she understood that she needed it, and after that it has not been a problem.

\emph{Have you used reward systems to motivate her?}

Yes. We promised her a trip to to the local water park if she took her medicine correctly the first week. Financially, it could be cumbersome if this ought to be a habit. You have to be careful when designing how the rewards, as there is a lot of differences across children. 

\emph{We have previously spoken to a child psychologist. She recommended that the reward comes quickly after a treatment.}

In my case, the fact that she breaths more easily after a treatment was rewarding enough, and it was worth more than the material reward she got. 

\emph{Did you use an alarm on your phone or similar in order to remember when she should take her medicines?}

No. Mainly, we just had to remember to give it to her before sending her to school and before she went to sleep. 
If she had an asthma attack, she's supposed to take the blue medicine, then wait for 5 minutes, before taking the orange medicine. The teddybear could help children to keep up with the time while they're waiting for the next doses. 

\emph{Do you think the teddy bear could have told a story or similar to make those 5 minutes seem shorter?}

My child is very understanding, but other children might become impatient. In that case, it could be useful to have something that could distract them. 

\emph{Did you use a journal to keep up with the dates where your child switched treatment plans?}

No, I have pretty good memory. I usually remember when she has been feeling ill. 

\emph{When the child is in kindergarten/school. Have problems with not taking medicine occured?}

The child does not need to take preventive medicine during the day. The child only need to take medicine if an asthma attack occurs. The child always carries extra equipment in the backpack to be prepared. The biggest problem is that the teachers/kindergarten teacher may not have knowledge of what to do when an asthma attack occurs. An application with instructions may be of help for them. 


\emph{When you leave the child with relatives/other caretakers; do you spend much time explaining how to apply a treatment?}

Yes, we always make sure that the relative/caretaker knows how the medicine is applied and what to do if an asthma attack occurs. We also tell the caretaker that the child is not allowed to take the medicine on its own, and they need to watch that the medicine is taken correctly. 

\emph{AsthmaBuddy is very stationary. What do you think of the fact that it can't be moved?}

If you have to set up the bear every time it is unplugged that may be a problem for some. For my child I don't think it would be a problem that the bear is not present at all times.

\emph{Do you have any other ideas for functionality or areas that AsthmaBuddy can be of use?}

Pollen forecast is a useful tool, but it can often be too general. I have trees in my backyard, they may release pollen before other trees and the general pollen forecast may not pick that up. 

Regarding dust sensors, what should they be used for? If the dust sensors may communicate with my robot vacuum cleaner, that could be a cool and helpful tool. 


\emph{When your child has an asthma attack. Do you think you would take the time to use AsthmaBuddy as a help tool?}

No. Taking the time to locate the bear and start the treatment would take to long time. It is more important to apply the medicine quickly. Although, I think that the functionality for registering the use of medicine afterwards would be a useful functionality.

\emph{How long time does it usually take to apply a treatment of preventive medicine?}

It usually takes about 1-2 minutes. 


The following question was asked via email at a later time.

\emph{Children will be rewarded 1 star per treatment when following the ``doing well (green)'' treatment plan, 3 stars per treatment when following ``caution (yellow)'' and 5 stars per treatment when in ``danger (red/syk)''. What do you think of this system where children recieve more rewards when they are ill?"}

I think that there should be no differentiation in the rewards at all. One star per treatment should do, regardless if the child is in good or bad shape. In general a sick child would need to take more [typically blue] medicine anyway, resulting in more stars.  
If there is a multiplication factor in addition to the increased number of treatments the number of starts would go up quite quickly and from the psychological perspective it might make the child think it is a good thing to be ill. In fact, one star for completing the plan might even be a better approach than one star per treatment (though the gifts can be adjusted in ``price''). Maybe even have two types of stars? For example a silver one for taking the medicine as planned, followed by a gold star once the treatment for the day has been completed.
