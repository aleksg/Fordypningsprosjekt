\textbf{Two nurses working at St. Olavs Hospital, with expertise on asthma.} 

\textbf{Date: 27th of March, 2014}

\textbf{Place: Trondheim}

\textbf{[Brief demonstration of AsthmAPP]}

The colors (of the medicines) differs between different companies, which should be kept in mind. 
Children of 3-8 eight years usually remember their medication as their parents keep control and remember to give them their treatment. 
Youngsters are among the hardest, as they have to take care of themselves and often do not remember or do not want to take their medication. The application has to belong to parents (reminding parents that their children should give medicine to children). 

\emph{We're developing a physical user interface which we'll try to use for motivational purposes among children. It contains light, sound and sensors that will communicate to children.} 

Not all children wants to take their medicine. They fight with parents because it is boring. Children below your target group (i.e. younger than 3 years old) can be even harder, as children in the group 3-5 years old has an understanding as to why they need to take their medicine. 

\emph{Do you have any tools that are provided to parents for taking medicine, remembering their children's medication, or to motivate them?}

Some companies have developed reward systems, but we don't give them to children.

\emph{Why is that?}

These tools often come with commercials for other tools from the company, which causes us to not want to use them. 

\emph{Do you have any suggestions for a reward system?}
It could be nice to use reward systems, as they often help the motivation for doing activities children normally don't want to perform.
Do you have any past experiences with parents that have created reward systems for their children?
 We can't remember being told that parents have created reward systems. It could be that parents don't tell us about it, or they simply don't do it. Taking an asthma medication is quickly over with, and children often realize that they get better by taking their medication.

\emph{We're thinking about displaying pollen forecast and information about air quality. Is this information something you notify parents about, with regards to paying attention every day?}

Yes, we talk a lot about pollen, as it has a direct relation to asthma. We also mention air quality, especially when it is cold outside. Highly trafficated streets will also have worse air quality, and parents should be aware of this.

\textbf{[More detailed demonstration of AsthmAPP]}
\emph{Users can register a medicine after it has been taken, instead of going through a 2 minutes long sequence]}

The medicine log is really nice. To ask a user how they've been today is important, as they won't remember how they felt a few weeks ago. 

Children don't need to wash their mouth after taking the blue medicine. This could create confusion if they should take it ahead of exercising. 

\emph{If parents were to use the log, do you believe parents could cheat, in order to make it look better when they're at the doctors office?}

A lot of parents are already writing a journal, and consider it useful. 

\emph{How do you provide information about usage of medicines?}

We hand out some flyers and talk to tgem. We demonstrate how they should use their medicines, in addition to distributing a treatment form. Parents also receive information specific to the medicine they are given. 


\emph{When the parents receive the medications and the instruction of how to take the different medications, do they have problems with understanding and remembering how to use the different medications?}

We always make sure to teach the parents and the children how to apply the medication correctly, however, they may forget it over time. If the parents do not remember how and when to give the children medications, it may have a negative effect on the treatment of the child's asthma. 


\emph{When school teachers, grandparents or babysitters take care of the children, parents will have to relay the information about the medicine and the treatment. Have you 
experienced or been told of problems with this relay of information?}

This is very individual. Some parents are very good at telling other caretakers about how to do the treatments. There is of course the risk that if a parent remembers the information incorrectly, the information relayed will be wrong, which may have effect on the treatment of the children. In order to make AsthmAPP or AsthmaBuddy be of help, it is important that the information stored within the app and the bear is correct and easily understandable. Having false information in the application would be negative for the user. 


\textbf{[Demonstration of AsthmaBuddy, through a video]}

The bear looks like a nice tool for the youngest children. Here at our clinic we have a bear called ``Asbj\o rn''. We use him to demonstrate to children and parents how you should take the different medications.
