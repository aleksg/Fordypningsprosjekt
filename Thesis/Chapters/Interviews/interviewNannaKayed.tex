\section{Interview with Doctor of Psychology Nanna S\o nnichsen Kayed}
\label{sec:psychinterview}

How do children react to reward systems such as the one in AsthmAPP?
- Children react very differently regarding reward systems. Some children find the rewards interesting, while some may not care at all. A common problem is that they may not be able to see the value of a reward given in the future. Younger children tend to choose rewards given immediately instead of a bigger reward given in the future \cite{mischel1972cognitive}.
The difference in how children percieve the value of a reward is very individual and there may be huge differences between age groups. 


Is there a possible that children may manipulate the reward system by pretending to be sicker than they are?
- It is difficult to determine if there is a risk of children pretending to be sick. Children often do not like going to the doctor's office, which may stop them from pretending to be sick.


The parents set the rewards themselves, how do you think this will work?
- The rewards has to be interesting for the individual child, and must be tailored to their interests. 

Do you think the reward system can lead to jealousy between children?
- Children percieve rewards differently, and a reward for one child may not be interesting for the other child. However if one child gets a reward that is much more valuable than the other children, if may lead to a situation where the children who recieve lesser rewards may be jealous. The parents of the children in the same class/kindergarten can make an agreement on what rewards should be set, in order to give equal rewards. It is important to find a reasonable level for the value of the rewards, to ensure it will not be a burden to the parents.


Rewards do not have to be a material reward, it may be a fun activity or letting the children choose what they will eat for dinner. Doing something entertaining with their parents can be as much of a rewards as a physical toy. An example of an easy and fun reward is to eat table sitting underneath the table or taking a walk in the woods. It is important for you [the developers] to tell the parents that the reward does not need to be material, but can be simple and easy rewards.


Can the reward system lead to differences between children that use the application and recieve rewards and children that do not use the application? 
- There at not that many children with asthma in each kindergarten. I believe that parents will talk to each other and share experiences which may result in positive sharing of knowledge. 


The stars earned in the application are not removed. Will this cause a problem with regarding teaching children the value of money and rewards?
- This is different from one child to another. Some children are ``collectors'' and would not spend the stars if they go away, but instead hoard them. There will be a risk that these children never spend their stars on rewards if they have to remove stars to get rewards.

[Demonstration of how AsthmAPP works]
- Have you [the developers] thought about letting the players play a game as a reward? Games such as Flappy Bird \fnurl{Flappy Bird}{http://flappybird.io/} are very popular with children, and a round doesn't take such a long time. 


It is also important to remember that the reward must be reflected in the severity of the task. 


[Demonstration of AsthmaBuddy]
- I believe that a cute bear like this may help in making children more positive to remembering their medicine. For a small child the bear itself may be enough of a reward.


The use of AsthmAPP and AsthmaBuddy are quite repetitive over time. Do you think it can be boring over time?
- A treatment takes less than 2 minutes, which will help. The app should have support for experts children, so they do not have to follow the instructions when they have become expert users. 

I do not see any negative aspects with the system. Leaving the reward system for the parents to choose is a good solution, but will require advice and information in order for it to work as a positive manner for both parents and children. 
You must also remember that it must not take a long time between the activity and the rewards, since children may not see the connection.

