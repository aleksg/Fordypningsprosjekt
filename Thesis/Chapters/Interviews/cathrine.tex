
\textbf{Mother of a 4.5 year old boy with a ``weak'' form of asthma.}

\textbf{Date: 17th of March, 2014}

\textbf{Place: Trondheim and Bergen \footnote{Performed over the phone}} 

\emph{Do you have to ``fight'' with your children when applying the asthma medicine?}

No, usually it is over quickly.

\emph{Is it hard to keep up with the treatment plan}

I find it a bit difficult. 

\emph{Have you used any tools in order to keep up with the plan?}

No. I miss an app that can easily structure when and how much a medicine have been taken. 
I also find it difficult to meet up at a doctors appointment, and remember when I have switched the medicine plans. 

\emph{How do you calm him down when he is stressed with regards to taking the medicine}

Usually by either singing or counting while he's breathing. 

\textbf{[Explanation of our reward system]}
\emph{Have you experimented with the use of rewards, either in the context of treamtents or similar?}

We have used bumperstickers on a sheet of paper while toilet training. In my opinion, this was rewarding enough. I do believe that your reward system could be useful for other kids though. A target group consisting of children between 3-7 years old is a very differing group in terms of cognitive development, and you should keep this in mind.  

\emph{Do you use demonstration to prove that the treatment is harmless?}

Demonstration works pretty good, but is not necessary in the long run. The demonstration helped calming him down during the start phase. 

\emph{What do you think of having a tangible user interface (a bear) that helps him taking his medicine?}

I'm positive to that. But I do not think it could replace the role of the parent, as someone has to watch that he actually takes his medicine. 

\emph{What do you think of the requirement to have a shorter sequence for by need treatments? }
I do not really see the need, as it would slow the process down (finding it, starting it, etc.). On the other hand, children are often stressed out when they have an asthma attack, so having a teddybear could help during the process in order to calm them down. 

\emph{Do you have any further comments?}
As a kindergarten teacher, I give several children help to take their medicine. An application that are able to help reminding kindergarten teachers when a specific child was to take their medicine would be much appreciated. 

Also, teaching children about their disease could be given more of an effort.

All in all, I think a shared mobile application that can help other caregivers as well as parents can be very benefitiary, as I often have to explain how and when to give a medicine to others. I regard the alarms as the most important feature.  
   
