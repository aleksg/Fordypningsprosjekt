\section{Focus Group Interviews}

When conducting focus groups, one should remember that they are first and foremost interviews and should be treated as such. One can not rely solely on the participants to come with ideas and opinions. In order to ensure a creative discussion one should prepare a number of topics to be discussed \cite{krueger2009focus}. In order to achieve creative answers it is important to let the participants have possibilities to comment on each others answers, but while they are allowed to comment, it is not necessary for them to come to an agreement. The facilitator of the discussion should change the topic if too much time is spent on disagreement. 
The interaction and discussion between the participants may enhance data quality compared to traditional one-on-one interviews, since a comment from one of the participants may trigger a thought for the other people taking part in the discussion \cite{krueger2000focus}.  

When putting together a focus group it is important to have diversity in a group. Having members who know each other on before-hand may lead to a situtation where there is too much agreement and to little criticism of ideas. Diversity can be ensured by inviting people of different genders, ages, opinions, marital status, expertise etc. 
Focus groups may be a good solution for getting feedback on major themes, while not as well suited for very specific technicalities, since the group may not be experts on the topic discussed \cite{krueger2009focus}. For a thorough and deep exploration of a theme it is often better to ask domain experts.

When conducting a focus group, especially when the participants are strangers to each other, there is a risk that the participant's do not want to share personal information or answer personal questions \cite{kaplowitz2000statistical}. A solution to this may be to interview the participants one-on-one afterwards, as they may be more open to answer questions when there is not strangers present. 