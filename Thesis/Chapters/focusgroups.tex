\section{Focus Group Interviews}

Focus groups are first and foremost an interview and should be treated as such. One can not rely solely on the participants and one should prepare a number of topics to be discussed. In order to achieve creative answers it is important to let the participants have possibilities to comment on each others answers, but while they are allowed to comment, it is not necessary for them to come to an agreement. The interaction and discussion between the participants may enhance data quality compared to traditional one-on-one interviews. 

When putting together a focus group it is important to have diversity in a group. Having members who know each other on before-hand may lead to a situtation where there is too much agreement and to little critisism of ideas. Focus groups may be a good solution for getting feedback on major themes, while not as well suited for very specific technicalities (better to ask experts). 

When conducting a focus group, especially when the participants are strangers to each other, there is a risk that the participant's do not want to share personal information or answer personal questions.