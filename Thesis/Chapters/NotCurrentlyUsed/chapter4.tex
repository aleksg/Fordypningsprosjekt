\chapter{Security Requirements}
\label{chp:securityrequirements}

This chapter will give a brief explanation of the security requirements enforced upon systems and applications that store medical information on Norwegian inhabitants. 


\section{Norwegian Law}
\label{sec:helseregisterloven}

Norway has specific laws for storing of medical information. The most significant law is ``The Health Register Act\footnote{Lov om helseregistre og behandling av helseopplysninger}''\cite{helseregisterloven}. This law regulates who is allowed to store health records and how they store the records. 

The most significant consequences are that we will need permission from ``The Norwegian Data Protection Authority''\fnurl{http://www.datatilsynet.no/} in order to store medical records in the application, and that the information has to be stored on servers on Norwegian soil. This eliminates the option of using cloud-based storage\footnote{Such as Amazon EC2 or Windows Azure}. 

%Jeg er ikke fornøyd med dette kapittelet. Skal jobbe mer med det når vi får svar fra Datatilsynet. /Aleks

\section{Measures for Anonymization}
Pursuant to section 16 of the Health Register Act \cite{helseregisterloven} all information that may identify a person, must be encrypted\footnote{There is no notion as of what level of encryption is required}. 

Since we have no interest in the data values or the personal information of the test persons we made the following measurements to completely anonymize the data:

\paragraph{Encryption}
In order to identify children, we have a few problems. First, it should not be possible to identify children by gaining access to the database. Second, we need a way that uniquely identifies the children, as both CAPP, GAPP and KAPP relies on uniquely identifying them. 

We propose the following level of encrypting a child's identity:
First, we will make use of the Android UUID (Unique Unit IDentifier). We will let the guardian type in the children's names. Then we will concatenate these values, and hash them using SHA-1\cite{sha1}. By including the Android UUID, we will get a one-way encryption function, which should be acceptable for storage.
 


\section{Basic security}
\paragraph{HTTPS vs HTTP} If the application is to be published by NAAF, there are some requirements towards sending data over HTTPS. However, in order to get HTTPS certificate, we have to pay a set fee (REFERANSE)\footnote{A small number of companies are allowed to sell HTTPS certificates. One of them is Symantec - http://www.symantec.com/verisign/ssl-certificates}. In addition, the communication will run more slowly, since data must be encrypted and decrypted. For demonstration value and early usability testing, we want to make sure that communication towards the database runs as smoothly as possible. As a consequence, we will not use HTTPS during the usability testing.


\paragraph{Passwords}
We will allow the user to create a password in order to protect medication records. These will be encrypted and stored securly in our database. 
