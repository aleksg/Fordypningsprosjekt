Intervju med Nanna, PhD Psykologi 25. mars 2014

Hvordan tenker barn rundt belønningssystemer?
- 	Hennes kompetanse er gjennom barne/ungdomspsykatri og generell oppdragelse.
	Veiledningsprogram for foreldre. Fokus på grensesetting og øke adferden man ønsker.
	[Artikkel vi bør lese]

- Tror du barn kan finne på å manipulerer systemet?
	Veldig ulikt fra person til person hvilket det fungerer.
	Noen barn liker det, noen ser ikke langsiktigheten av belønningen.
	Yngre barn krever at belønningen kommer i nær fremtid.
	Stort aldersspenn. [Gjort forsøk på å få en karamell nå, eller to karameller om 10 min]
	Relevant å se på nærhet i tid.

	Rød helsekategori krever at man må til legen, og det er sjelden noe barn ønsker
	Ønsker å øke adferden "ta medisin". Vanskelig å si om man øker hypkonderfaktoren.


- Foreldre setter belønningene selv. Tror du det kan skape forskjeller og misunnelse?
	Belønningene som settes må treffe det enkelte barn.
	Hvis naboen har bedre belønningen, kan det oppstå problemer fordi barna ikke verdsetter belønningen.
	Foreldrene kan snakke sammen for å bli enig om hva som er bra premier.
	Viktig å legge seg på et fornuftig nivå. 


Belønningen kan være at de får lov til å bestemme kveldsmat eller en aktivitet. Det å ha kvalitetstid med foreldrene kan være en like så god belønning.
Noen ganger har det "tatt helt av", eksempel med julekalendere med pakker. 
Tur i skogen eller en morsom opplevelse kan være en like god belønning.
Lett at foreldre tenker at belønning må være noe materielt, men de kan fortelles at det ikke må være det.
Eksempel: Spise frokost under bordet.

- Tror du kan oppstå forskjeller mellom brukere som bruker belønningssystemet og de som ikke bruker det?
	Ikke så mange i hver barnehage som tar medisiner. 
	Mulig at foreldre kan fortelle hverandre om sine opplevelser.
	Barn er veldig aksepterende om at det er ulikheter, og de blir ikke nødvendigvis misunnelige. 


- Belønningene foregår på milepælsbasis. Stjernene forsvinner ikke. Kan det skape et problem med at barn må lære at "ting koster"
	Grensene blir hevet istedenfor å 
	Veldig individuelt. Noen barn liker å spare, noen ønsker å bruke. 
	Risiko for at "sparerne" ikke vil ta ut noen belønninger.


- De oppnår 3-5 stjerner hver dag.
	Må gi en veiledning til foreldrene for at de skal forstå det. 
	Belønningen må komme såpass tett på adferden at det skapes en sammenheng.
	Bra at foreldrene kan legge det opp selv, men det bør gis en guide for at barna skal få belønningen sin.
	"Ta medisinen din hver dag i et halvt år, så får du en fotballbillett"


- Demo av appen og hvordan behandlingen foregår
	Har dere vurdert at belønningen kan være et lite spill i appen?
	Noen av disse spillene er veldig avhengighetsskapende (Flappy Bird). Kanskje belønningen kan være at de får spille et spill i appen.


- Cambridge Cognition. Ingen belønninger, fungerer sub-optimalt. Måler impulsivitet.
	Lese mer om dette. 

- Appen må ikke være avhengighetsskapende. Barna kan ta medisinen "for mye"
	Belønningen må stå i forhold til oppgaven som skal gjennomføres. 
	Umiddelbar belønning fremfor å måtte spare opp belønninger frem i tid. 



- Vise frem AsthmaBuddy
	Tror at det kan være hjelp til å gjøre det mindre skremmende.
	For et lite barn kan dette være et morsomt brukergrensesnitt.
	Å gjøre "legeting" og medisiner til en positiv og snill greie er bra.


- Repetitivitet. Tror du det kan bli kjedelig i lengden?
	Det er et kort tidsspenn.
	Burde være støtte for "ekspertløsning" for å slippe å gå gjennom hele
	Plotte inn alder kan føre til ulik funksjonalitet.
	Veldig forskjellig for de ulike alderene. 
	Litt kulere belønninger for eldre barn.
	Se på "Josefine-spillene" for å vite hvor avanserte de er.

Ser ingen negative konsekvenser med systemet. Fint at foreldrene har fleksibilitet, men det kreves god veiledning. 
Det kan ikke ta for lang tid mellom adferd og belønning. 



Pearson (de lager IQ-tester). CogMed.com - Et system for å trene opp arbeidsminnet. Barna får kjøre en bilbane. Kan være noe å hente her. 