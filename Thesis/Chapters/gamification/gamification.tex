\chapter{Gamification}
\label{chp:gamification}

This chapter will give a description of the term ``Gamification'', describe some of the uses of Gamification and how we plan to use Gamification in our solution.

\section{What is Gamification?}
\label{sec:whatisgamification}
``Gamification'' as a term was first mentioned by Currier in 2008\cite{gamificationcurrier}, but did not become a wide-spread term before 2010. 

There are many different ways of describing gamification. Deterding, Dixon, Khaled and Nacke\cite{Deterding:2011:GDE:2181037.2181040} define Gamification as:

\textit{``Gamification is the use of game design elements in non-game
contexts.''}

Huotari and Hamari\cite{huotari2012defining} defines gamification as:

\textit{``Gamification is a process of enhancing a service with affordances for gameful experiences in order to support user's overall value creation.''}

Deterding, Dixon, Khaled and Nacke's definition is often commonly referred to, because of it's simplicity and understandability for people who have little or no connection to traditional games or game consoles.

Today gamification is a much used term both in programming and in the spoken language. Smartphone applications and manufacturers have helped make the term gamification a widespread notion. Examples of this is the application Foursquare, which is built around gamifying ``checking in'' at restaurants, historical sites and similar places\fnurl{Foursquare}{www.foursquare.com}. Apple developed a Game Center for iOS in 2010, giving every iPhone/iPod and iPad user a hub for challenges, awards and other gamelike activities\fnurl{Apple Game Center}{http://support.apple.com/kb/HT4314}, which made every iOS user a potential target for Gamification. Lately there has been many games built singularily around gamification, such as Cookie Clicker\fnurl{Cookieclicker}{http://orteil.dashnet.org/cookieclicker/} or Farmville\fnurl{Farmville}{www.farmville.com}. Even game consoles like Playstation 3 and Xbox contain gamification support per default, with their achievement/trophy systems\fnurl{Xbox}{http://xbox.com}\fnurl{Playstation}{http://playstation.com}. While there are many users of such games, they are often critiqued for using gamification to lure players into playing. 


\section{What is Serious Games?}

The term ``serious game'' became a concept with the emergence of the Serious Game Initiative in 2002\cite{seriousgamesinitative}. Their website defines serious games as: 

\textit{``The Serious Games Initiative is focused on uses for games in exploring management and leadership challenges facing the public sector. Part of its overall charter is to help forge productive links between the electronic game industry and projects involving the use of games in education, training, health, and public policy.''} 

This definition has been critiqued for being to narrow, and not including any reason as to why businesses should care. An anonymous author\footnote{The essay is only signed with the name 'Danc'. Still, we regard this essay interesting and relevant, and it has been mentioned in several scientific publications.} posted an essay on \url{www.lostgarden.com} criticizing the definition and suggesting the following definition:

\textit{``Serious Games: The application of gaming technology, process, and design to the solution of problems faced by businesses and other organizations. Serious games promote the transfer and cross fertilization of game development knowledge and techniques in traditionally non-game markets such as training, product design, sales, marketing, etc.''}


Since it's debut in 2002, serious games has later grown to becoming a multi-billion industry.
Pilots are being trained in simulators, lecturers make lecture quizzes for students\cite{wang2007lecture}, swedish firefighters have used serious games for training\cite{lebram2009design} and persons suffering from diabetes have the ability of using serious games for learning about the illness. These are just a few of the many ways of using serious games. 

Foldit is a very interesting example of how a serious game may lead to knowledge and solving bigger problems than the game itself\cite{cooper2010predicting}. Foldit is a massive multiplayer online game (MMO). The objective for the player is to fold protein following a set of rules. The system records how players fold protein and learns patterns for interaction. Humans have much higher skills at interacting with 3D objects than computers, and the system learns patterns and techniques from the players. By playing Foldit, researchers were able to solve the crystal structure of the M-PMV retroviral protease\fnurl{Mason Pfizer Monkey Virus}{http://microbewiki.kenyon.edu/index.php/Mason\_pfizer\_monkey\_virus}\cite{khatib2011crystal}.

Serious Games and gamification have many similarities, whereas serious games are mainly targeted towards making education or learning more fun, gamification is used in a number of different ways. 


\section{Discussion about Gamification}
\label{sec:gamificationdiscussion}

Gamification is a much discussed theme, where there does not seem to be an agreement as to which gamification is a useful or not. 
Antin and Churchill argues that gamification may be used for goal setting or instruction\cite{antin2011badges}. Goal setting challenge the users to meet the mark that is set for them, and is known to be an effective motivator\cite{ling2005using}. 

Bogost goes as far as naming gamification as ``marketing bullshit'', used as a way of moneytizing bad business\cite{gamificationbullshit}\footnote{While this is not a scientific publication, we found it interesting and relevant to the discussion}.

McGonigal's studies on how rewards are perceived over time show that: 

\textit{``After three hours of consecutive online play, gamers receive 50 percent fewer rewards (and half the fiero\footnote{Fiero is an italian term for personal triumph\cite{ekman2007emotions}}) for accomplishing the same amount of work.''}\cite{jane2011reality}

Steinung arguments for gamification not being powerful enough to make a task interesting\cite{steinung2012interessante}. Simply adding points, badges, a leveling system or similiar, won't make a task interesting on its own. Since gamification is based on behavioural pshychology, poor design may be perceived as interesting, for a shorter period of time\cite{steinung2012interessante}. Zichermann makes a similar statement, saying gamification needs to take ethical precautions\cite{zichermann2011gamification}.

While McGonigal's research dives into how rewards are percieved when playing over a longer consecutive time, our intent is to make the user spend only small amounts of time using the application. AsthmAPP is a tool, not a time-waster.

In order to achieve a meaningful use of gamification Nicholson\cite{nicholson2012user} suggests using a user-centered design approach\cite{usercentereddesign} when developing system with elements of gamification. Since AsthmaBuddy is a computer supported learning system\cite{stahl2006computer} it will be important for us to maintain focus on the learning and awareness created by our system, making gamification a tool and not the key feature.

\section{Game elements}

This section will take a brief look into the different classifications of players that exists, and will introduce the reader to game mechanisms commonly used to gamify users' experiences. 

\subsection{Bartle's Four Player Types}
Richard Bartle defined four different player types \cite{bartle-gamers}. These types are \emph{Achievers},  \emph{Explorers},  \emph{Socialisers} and \emph{Killers}. We'll take a brief look on each of these in this section. 

\subsubsection{Achievers}
\textit{``Achievers regard points-gathering and rising in levels as their main goal, and all is ultimately subserviant to this''} \cite{bartle-gamers}. 

Most young children will fall under this category. Achievers mostly play games just for the fun of it, and don't necessarily need other incentives to the game than being able to clear it. Most children like to see progress in terms of points, clearing a lever, etc. 

\subsubsection{Explorers}
\textit{``Explorers delight in having the game expose its internal machinations to them''} \cite{bartle-gamers}.

Explorers are thus the players who easily enjoy a game more than once, and potentially want to find every secret embedded in the game. Children will in some cases fall under this category, but with our target group, it is hard to separate between achievers and explorers.   

\subsubsection{Socialisers}
\textit{``Socialicers are interested in people, and what they have to say. The game is merely a backdrop, a common ground where things happen to players''} \cite{bartle-gamers}. 

This implies that socialisers play games in order to connect with new people or hang out with their friends. The youngest children in our target group probably won't fall under this category, as they won't understand whether John or Hanna is on the ``other side of the screen''.    

\subsubsection{Killers}
\textit{``Killers get their kicks from imposing themselves on others''} \cite{bartle-gamers}.

Killers thrives upon destroying other people's game experience. Hopefully, no children fall into this category, at least not in our target group.  

\subsection{Game Mechanisms Used to Achieve Gamification} 

There are some game mechanisms that are widely used for gamifying every day tasks. This section will explain some of them. We will use a stick figure to examplify each game mechanism. 

\subsubsection{Avatar systems}
Avatars are commonly used in children games. It gives a player a virtual character, which can be upgraded with different clothing and equipment when players reach certain points in the game. The equipment can usually be bought for either points awarded or through \emph{In-app purchases}. Players can then show their avatar to other users, compare, and have fun with them. This approach could be seen as giving avatar a piece of their personality. For instance, some players would prefer that their avatar looked as ridiculous as possible, while others would prefer if they looked as cool as possible. Showing off ``expensive'' gear would also give some sort of accomplishment (\emph{``I'm so good in this game, that I could afford this. Can you?''}). 

\textbf{Example:} The stick figure will be a players avatar, which can be modified to have different pieces of clothing or equipment.  

\subsubsection{Acheivements and badges}
Achievements and badges are systems well incorporated into Microsoft's Xbox and Sony's Playstation. These achievements are typically given if the player achieves something in the game. For instance, on Foursquare, you get a badge called ``Adventurer'' if you check in at 10 different venues. The stick figure

\textbf{Example:} If we combine this mechanism with avatar systems, we could give out a badge when the stick figure have obtained complete sets of clothes or a specific set (i.eg. buying all the green clothing).   

\subsubsection{Real-world rewards}
Often used with leaderboards, real-world awards could be given to some of the best players of the game. For instance, they could be rewarded with exclusive tickets to concerts. These real-world awards are often given during marketing campaigns, for instance ``Invite your friends to use this system, and get one ticket in the lottery to win a brand new computer''.  

\textbf{Example:} A company could send a real world example of a player's stick figure. [WEIRD]

\subsubsection{Social networking}
During the last couple of years, Facebook feeds as tended to get flooded by updates from third-party applications, like Runkeeper, who updates everyone on your friend list that you have been working out. The idea here is to have a common platform, where users can brag of what they have done. For instance, if Tommy has an update that says ``I just ran 12 km in 59 minutes'', Mari could upload her records, beating Tommy's, which gives Mari a sense of achievement, while Tommy is motivated to work harder for his results.     

\textbf{Example:} Social networking could be used to upload images of a player's stick figure, and show it to his/her friends. 

\subsubsection{Mirroring user behaviour}
This is most commonly used for children, where an animation or a character shows how to go forward with a procedure. For instance, there are a lot of apps on App Store mirroring the process of brushing a child's teeth. A kid can then use this app as a reference that indicates when it is time to switch sides.  

\textbf{Example:} The stick figure could mirror the player's intended behavior. 

\subsubsection{Leaderboards}
A leaderboard is a list of the players ordered by their collected points, completed activites or any other predefined system. Each user has a score defined rules set before a competition is started, the score is compared and then players are ranked based on the scores. Leaderboards may be fully dynamic, changing when a player has scored points, or state based, where the new order is calculated after a set period of time.

\textbf{Example:} If the stick figure gathers enough experience points, it could find itself on a regional leaderboard of some sort. 

\subsubsection{Progress bar}
A progress bar is used to indicate how far a user has come towards a given goal. When the player/user completes task and/or activities, the progress bar is filled to indicate the progress of coming closer to a goal. How much the progress bar is moved is often determined by the severity of a task or by a system using points. The progress bar can often be combined with experience points, where the experience points collected determines the movement of the progress bar. 

\textbf{Example:} The stick figure could be placed on a road. The figures position on that road, would mirror the progress a player has made. 

\subsubsection{Experience points}
Experience points is an indicator of how much experience the player has gathered within a game or setting. These points may be awarded from completing tasks, exploring areas and features or other similar activities. Experience points are usually combined with a leveling system, where the player ``climbs a ladder'' using these experience points, for example by being awarded with a new level, new rewards or unlocking new features in a system. A player with many experience points is considered an experienced user, and is percieved as higher ranking than a player with less experience points. Experience points are also often combined with leaderboards. 

\textbf{Example:} The experience points could be represented as a stick figure, and the goal is to gather as many stick figures as possible. 

\subsubsection{Contests}
Gamification can be done through having contests either with players in a duel-like head-to-head contest or a free-for-all contest with no limit on the capacity of players. A duel-like contest may be a knockout style of competition where players compete to get the most points within a given time period or a similar goal. A player may make progress in the tournament without being the player with the highest score, but being better than his/hers opponent. In a free-for-all contest it is simply the goal who fulfills a specific goal to the best degree. The goal which players try to achieve is set be a specific set of rules determined before the start of the contest.

\textbf{Example:} A player's stick figure could go into a voting contest, where votes are given on the best looking one, etc. 

\subsection{Combining game mechanisms in AsthmAPP}
%TODO: This should probably go somewhere else..


\begin{table}[H]
\begin{tabular}{| p{3.0cm} | p{2.5cm} | p{8.0cm} | }
	\hline
	\textbf{Mechanism} & \textbf{Included in AsthmAPP} & \textbf{Rationale} \\
	\hline
	Avatar Systems & No & It would simply take too much time for two CS students with approximately 0 drawingskills to implement this. \\
	\hline
	Achievements and badges & No & We belive our target group would not enjoy this feature as much as older children, like 12-16 years of age.  \\
	\hline 
	Real-World awards & Yes & Children enjoy the feeling of being rewarded with something real (Said the guy with no children). \\
	\hline
	Social Networking & No & For reasons too obvious to mention here. \\
	\hline
	Mirroring User Behavior & Yes & Demonstration has a positive effect on children.
	\\
	\hline
	Leaderboards & No & There are no way to implement this in a realistic and legal way. Children would have to share their data, which consists of points based on medicine doses. Parents could be blamed if their child were at the bottom of the list. It would also be negative for small children's motivation if they have been really good at taking their medicine, and performs poorly on a leaderboard. 
	\\
	\hline
	Progress bar & No & The only idea we could come up with for a progress bar in AsthmAPP, was to show how close the child was to becoming treated for Asthma. However, identifying how close a patient is to becoming healthy, is virtully impossible. 
	\\
	\hline
	Experience Points & Yes & The stars will serve as experience points, which can used to buy things. 
	\\
	\hline
	Contests & No & This would be the most depressing contest in the world. ``Who's best at taking their asthma medicine?''   
	\\
	\hline
\end{tabular}
\caption{Assessment of different game mechanisms}
\label{tab:game-mech-in-astmapp}
\end{table}




\section{Summary}
\label{sec:gamificationinapp}

Stapleton argues that:

\textit{``A variety of [serious game] applications can be thought of here [in Health Care] such as games as a form of motivation and reward for patients undergoing some form of treatment. Games could also be to distract patients during certain procedures such as dental work, for example."}\cite{stapleton2004serious}


Stapleton's argument is a central argument for why we believe in our the use of tangible user interfaces and applications as a method of treating children with asthma. Children are easily distracted, and we believe that making the treatment into a serious game will distract the children enough to forget what they are doing and instead look on the treatment process as a fun game.


In AsthmAPP we aim to use gamification as a distraction and rewarding element for the children. Instead of putting a lot of predefined badges, rewards, experience points or other rewarding elements, we let the users choose their own rewards, which implies that users can decide what works best for them. This stands to reason with the arguments of Nicholson\cite{nicholson2012user} regarding how to design gamification.

As McGonigal's studies show, the effect of gamification tend to wither down after a longer period of time\cite{jane2011reality}. We aim to solve this problem by letting the user choose their own rewards. While putting the responsibility for the reward system in the hands of the user will lead to more work for the user, we believe that the positive effect of the reward system will make it worthwhile. 
 

The children are rewarded with stars based on their health state. The rationale behind this is that the children may have to take more medicine when they have a cold or there is a lot of pollen in the air. The parents have access to a administrator menu where they may set new rewards for the children. The children will then be able to order the rewards when they have earned a sufficient number of stars. This way the parents and their children create their own gamification environment. Examples of possible rewards could be to give their child an extra 10 NOK in allowance, taking them to soccer matches or even to the local amusement park. It is an option where the only boundary is the imagination and how much cost and effort parents want to put into it.    


The rewards will appear on a ``milestone'' basis. We do not want children to feel they lose something if they buy a reward, which some might experience if stars were taken away from them. We do not want to force parents into giving away rewards they can not afford or do not wish to use. Therefore we will be testing the application without the gamification elements, in order to find how introduction of gamification elements will affect the use. The use of rewards is also optional and decided by the user, making the user in control of how they wish to gamify the experience. 
We do not wish to have the children spending too much time using the application, since using a tablet or phone at such a young age is considered unhealthy, which is taken into consideration with making an application which is mostly used ``outside'' the digital application itself. The children may receive rewards and use the TUI without touching the smart phone.

