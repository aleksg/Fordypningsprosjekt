\chapter{Security Requirements}
\label{chp:securityrequirements}

This chapter will give a brief explanation of the security requirements enforced upon systems and applications that store medical information about Norwegian inhabitants. 


\section{Norwegian Law}
\label{sec:helseregisterloven}

Norway has specific laws for storing of medical information. The most significant law is ``The Health Register Act\footnote{Lov om helseregistre og behandling av helseopplysninger}''\cite{helseregisterloven}. This law regulates who is allowed to store health records and how they are supposed to store the records, among other regulations. 

The most significant consequences are that we will need permission from ``The Norwegian Data Protection Authority''\footnote{Datatilsynet} in order to store medical records in the application, and that the information has to be stored on servers on Norwegian soil. This eliminates the option of using cloud-based storage. 

%Jeg er ikke fornøyd med dette kapittelet. Skal jobbe mer med det når vi får svar fra Datatilsynet. /Aleks

\section{Measures for Anonymization}
As The Health Register Act states in \S 16 \cite{helseregisterloven} all information that may identify a person, must be encrypted\footnote{There is no notion as of what level of encryption is required}. 

Since we have no interest in the data values or the personal information of the test persons we made the following measurements to completely anonymize the data:

\paragraph{Encryption}
In order to identify children, we have a couple of problems. First off, it should not be possible to identify children by gaining access to the database. Second, we need a way that uniquely identifies the children, as both CAPP, GAPP and KAPP relies on uniquely identifying them. 

We propose the following level of encrypting a child's identity:
First, we will make use of the Android UUID (Unique Unit IDentifier). We will let the guardian type in the children's names. Then we will concatenate these values, and hash them using SHA-1. By including the Android UUID, we will get a one-way encryption function, which should be acceptable for storage.
 

\section{Personalized Access Control for a Personally Controlled Health Record}
\label{personalhealthrecords}

One of the most wanted features for CAPP/GAPP/KAPP was to be able to share the treatment history recorded in the application with the doctor's office. Keeping a medical journal is no revolution, but sharing detailed information about treatment history in the way CAPP/GAPP/KAPP does is not done today. The guardians and physician's opinions about this sharing of information are one of the central questions we aim to answer, as mentioned in \ref{sec:researchquestions}.

The idea of a complete Patient/Personally Controlled Health Record was presented by Mandl et al.\cite{mandl2007indivo} in 2001. The idea is to assemble the complete health history of the patient in one place. Røstad and Nytrø \cite{rostad2008personalized} made a list of security requirements for PCHR, one of which is ``The patient is the administrator of access to his/her information. The patient decides what permissions to assign to who''. This specific requirement and other requirements\cite{rostad2008personalized} have been taken into consideration when developing CAPP/GAPP/KAPP. 



\paragraph{Nasjonal Kjernejournal}
%Burde vi skrive noe om dette? Jeg vet bare ikke hvordan det skal være aktuelt /Aleks


\section{Basic security}
\paragraph{HTTPS vs HTTP} If the application are ever to be published by NAAF, there are some requirements towards sending data over HTTPS. However, in order to get HTTPS certificate, we have to pay a set fee (REFERANSE)\footnote{A small number of companies are allowed to sell HTTPS certificates. On of them is Symantec - http://www.symantec.com/verisign/ssl-certificates}. In addition, the communication will run slower, since data must be encrypted and decrypted. For demonstration value and early usability testing, we want to make sure that communication towards the database runs as smoothly as possible. As a consequence, we will not use HTTPS during the usability testing.


    
