\section{Use of Gamification in \app{}}
\label{sec:useofgamificationinapp}

\subsection{Design rationale for gamification system in \app{}}
\label{sec:designrationalegamification}
In AsthmAPP we aimed to use gamification as a distraction and rewarding element for the children. When deciding to use gamification we arranged a brain-storming session to find the different solutions for how we could make use of different gamification techniques. The summary of this session is listed in Section \ref{sec:combininggamemechanismsinasthmapp}. Firstly we wanted to make an avatar system, since this leaves much room for the user to make a personalized avatar, and we believed this may also leave many possibilities for expanding in order to combat fatigue with the gamification system. Unfortunately the avatar system was left out after we spent some hours programming it, we chose to discard it, due to little progress and the lack of resources. Making a complete working avatar system would require skills in photo design, which none of us had, and we found it too time-consuming to learn during our project. Secondly we chose to aim for a system we were sure we were able to implement in a well-working fashion in order to test the system with users, an argument for why we chose to use experience points (represented by stars) is the possibilites for expanding upon this on a later stage. Experience may be combined with many other elements on a later stage. We also chose to let the user choose their own rewards and how they would make use of the gamification system. This stands to reason with the arguments of Nicholson regarding how to design gamification\cite{nicholson2012user}.

The children are rewarded with stars based on their health state. The rationale behind this is that the children may have to take more medicine when they have a cold or there is a lot of pollen in the air. The parents have access to a administrator menu where they may set new rewards for the children. The children will then be able to order the rewards when they have earned a sufficient number of stars. This way the parents and the children create their own gamification environment. Examples of possible rewards could be to give the child an extra 10 NOK in weekly allowance, taking him/her to soccer matches or even to the local amusement park. It is an option where the only boundary is the imagination and how much cost and effort parents wish to invest in it.    

The rewards will appear on a ``milestone'' basis. We do not want children to feel they lose something if they spend stars on a reward, which some may feel as if stars are taken from them. We do not want to force parents into giving away rewards they can not afford or do not wish to give. The use of the reward system is optional and decided by the user, making the user in control of how they wish to gamify the experience. 
We do not wish to have the children spending too much time using the application, since using a tablet or phone at such a young age is considered unhealthy. This had some implications on the complexity of our gamification system. 


\subsection{Gamification Elements in AsthmAPP}
\label{sec:combininggamemechanismsinasthmapp}

\begin{onehalfspacing}
\begin{table}[H]
\begin{tabular}{| p{2.5cm} | p{2.1cm} | p{9.5cm} | }
	\hline
	\textbf{Mechanism} & \textbf{Included in AsthmAPP} & \textbf{Rationale} \\
	\hline
	Avatar systems & No & We did not have time to implement this during our thesis, but we do believe this could be a good feature if it was implemented in a right manner.    
	 \\
	\hline
	Achievements and badges & No & We believe our target group would not enjoy this feature as much as older children, those of 12 - 16 years of age.  \\
	\hline 
	Real-World awards & Yes & Children enjoy the feeling of being rewarded with something real.
	 \\
	\hline
	Mirroring user behavior & Yes & Demonstration has a positive effect on children.
	\\
	\hline
	Leaderboards & No & There are no way to implement this in a realistic and legal way. Children would have to share their data, which consists of points based on medicine doses. Parents could be blamed if their child were on the bottom of the list. It would also be negative for small children's motivation if they have been really good at taking their medicine, and perform poorly on a leaderboard. 
	\\
	\hline
	Social networking & No & Much of the same reasons as why we did not choose leaderboards. We also believe that children in our target group would not understand this aspect.  
	\\
	\hline
	Progress bar & No & In an ideal world, we could have shown how close the child was to becoming fully treated for asthma. However, identifying how close a patient is to becoming healthy, is virtually impossible. Another usage may be to match the child's progress for each week against his/her medicine plan. This would however, imply that by-need treatments would not be included, which may seem unfair for the child. Additionally, their progress would be deleted every week, which would not have much motivational effect.  
	\\
	\hline
	Experience points & Yes & The stars will work as experience points, which may used to cash in rewards from the parents. 
	\\
	\hline
	Contests & No & Using medical history to participate in contests would be a violation of Norwegian privacy laws. It may also be used to pinpoint ``bad'' parents.      
	\\
	\hline
\end{tabular}
\caption{Assessment of different game mechanisms}
\label{tab:game-mech-in-astmapp}
\end{table}
\end{onehalfspacing}
