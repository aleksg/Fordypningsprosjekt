
\chapter{AsthmaBuddy}
\label{chp:our-solution}

\section{Background}
In 2012, we did a similar project using Karotz \fnurl{Karotz}{www.karotz.com} as our platform\cite{CustomerDriven}. The thought behind Karotz is great. It is an open source robot which allows people to build applications and launch it to the Karotz store. However, in our subjective opinion, it is not ideal to work with. The Karotz starters kit costs ~\$200, not including customs, thus it is a pretty large investment for a family wanting to buy the product to use with our application. The API is only documented in French, which makes it a hassle, in addition to the fact that it is pretty cumbersome to configure for a ``non-technological'' family. 

\section{Technology}
We have looked around for other options than Karotz to create our tangible user interface, i.e. Arduino and Raspberry Pi. Arduino is an open source electronics prototyping platform\cite{arduino}, which allows for many different combinations of configurations, while \rpi{} is a cheap computer on the size of a credit card. Arduino shields comes in many shapes and sizes and is built for modularity and extendability. A wide-range of components are available if you want to add technical functionality to an Arduino system, such as Bluetooth, WiFi or small motors. 
While Arduino allows complex hardware configurations, Raspberry Pi makes larger abstractions, which seemed like the better choice for us as developers. Arduino programs are normally written in C\cite{strahl2000language}, and while there exists many tutorials and manuals for how to write Arduino code, we found the thought of writing in a programming language we had little prior knowledge of too challenging. Arduinos generally have low-powered CPUs, in order to keep them cheap. These low-powered CPUs tend to have problems with decoding MP3-files, which would lay constraints on our system. Due to these facts, we choose to develop the system on a Raspberry Pi.


\subsection{Raspberry Pi - Specifications Overview}
The \rpi{} was initially intended to teach british school children about computer programming\cite{rasperrypi-about}. Since its release, it took an unexpected turn when lots of computer enthusiasts bought the product to do their own mini projects for a cheap price. 

The specification of a \rpi{} (Model B) is included in Table \ref{tab:pi-specs}. Figure \ref{fig:pi-arch-overview} shows an overview of the \rpi{}.        

\begin{table}[H]
\begin{tabular}{|p{6.0cm} | p{6.0cm} |}
\hline 
\textbf{Property} & \textbf{Specification} \\
\hline
CPU & 700 MHz ARM1176JZF-S core \\
\hline
Memory & 512 MB \\
\hline
USB 2.0 ports & 2 \\
\hline
Video Output & HDMI \\
\hline
Audio Output & 3.5 mm jack, in addition to ability to play sound through HDMI \\
\hline
Low-level Peripherals & 8 x GPIO (General Purpose Input/Output) \\
\hline
Power Source & 5 volt MicroUSB \\
\hline
Storage & SD card (available with preinstalled OS) \\
\hline
Network & 10/100 Mbps Ethernet.  \\
\hline
\end{tabular}
\caption{Raspberry Pi specifications}
\label{tab:pi-specs}
\end{table}

\begin{figure}[H] 
	\begin{minipage}[b]{0.4\linewidth}
	\centering
		\includegraphics[width=0.3\paperwidth]{Pictures/rpi-arch-overview.png}
	\caption[Raspberry Pi Model B Architecture]{Raspberry Pi Model B architecture. \emph{Image source: http://raspberrypi.org/faqs} }
	\label{fig:pi-arch-overview}
	\end{minipage}
	\hspace{2.0cm}
	\begin{minipage}[b]{0.4\linewidth}
		\centering
			\includegraphics[width=0.3\paperwidth]{Pictures/pi-fritzing-model.png}
		\caption{Digital schematic over \rpi{} }
		\label{fig:pi-fritzing}
	\end{minipage}
\end{figure}
 
  
\subsection{Additional Components}
In addition to the \rpi{} we needed some components that children are able to interact through. These components and their functionality are summarized in this section. 


\paragraph{RFID Reader}
A child needs to be able to interact with AsthmaBuddy. We identified two approaches; using a button and/or using RFID technology to do it. We figured that having a big button on the top of a teddy bear would seem somewhat unnatural for a child. Additionally, this could cause problems with the wires inside the teddybear, as a child pushing a button could imply that AsthmaBuddy would be moved around. This could have been avoided by using a battery pack for the \rpi{}, but their capacity does not seem to exceed 24 hours when the \rpi{} is in idle mode. The conclusion was to use RFID technology to proceed in the medication process.


The RFID reader we used was a Sparkfun ID-12LA \fnurl{Sparkfun ID-12LA documentation}{http://tiny.cc/sparkfundoc}. One of the requirements of the USB reader was that it should be able to connect through an USB-port. 
         
\paragraph{USB speakers}
In order to play sounds to children, we decided to integrate speakers inside AsthmaBuddy. Since we did not want to pull too many wires out of the bear, we decided to use USB-powered speakers.    

\paragraph{LED lights}
We used LED lights connected to a breadboard in order to play around with the first prototype. The LED lights emits light in different colors to correspond to what action(s) is expected from the user during a treatment (see more in \ref{sec:proto1}).

\paragraph{Pi4j}
Pi4j\fnurl{Pi4j}{http://pi4j.com} is a Java framework that allows development for \rpi{} in Java, without having to write anything in C. 


Figure \ref{fig:pi-fritzing} shows a digital overview of \buddy{}. The green figure to the left is our \rpi{}. While it is also connected to a power supply and an internet cable, we chose not to include these in our figure. The red figure to the right is the RFID reader. It is connected to the \rpi{} through an USB cable. The black figure on top is the speaker, connected to the \rpi{} through the audio port. 
The grey lines and the lamp represents our LED light. It is connected to three of the GPIO (General Purpose I/O) ports on the \rpi{}, through a resistor. The last leg of the LED light is connected to ground on the \rpi{}, without a resistor.

\section{Design Rationale}
\subsection{Why a Teddy Bear?}
When designing \buddy{} we choose to use a teddy bear as an avator for our system. There are several reasons as to why we think this is an appropriate avatar. Teddy bears are well known toys, and has been loved for a long time. They are considered gender-neutral\cite{stagnitti1997determining}\cite{cherney2006gender} and in our subjective opinion it is a toy that could be discretely placed in a children's room. With the look of a teddy bear, \buddy{} can easily be placed among other toys and not be too visible. It was also important for us to choose a teddy bear of some size. A too thin bear could lead to problems with fitting the system inside the bear, and could in order lead to scepticism from the children. \buddy{} will have similarities to Tamagotchi\cite{tamagotchi} and Furby\cite{furby}, but \buddy{}'s purpose is to motivate, instruct and learn children about asthma, not being a toy meant purely for play. 

While designing our system we also wanted to make sure our system did not have robot-like features or robotic similarities. While children tend to find technology very interesting, we want to make \buddy{} seem as natural as possible, making a stuffed animal buddy, rather than a technological toy. We believe that these design choices serves our purpose of making children more aware of their asthma, while not being a constant reminder and a stress element. 

Norwegian fire fighters have used a teddy bear in order to calm down children who find themselves in dramatic situations \fnurl{NRK: Firefighters use teddy bears to calm down children}{http://www.nrk.no/trondelag/bamser-i-utrykningsbilene-1.11548966}. The fire fighters state that the children respond positively to the teddy bear. 
While we were not able to find scientific research done on the use of teddy bears in dramatic situations, we found this news article interesting and a relevant mention to our research.

[Pictures of the bear we chose?]

\subsection{Interaction design}
When we started developing the interaction design of \buddy{}, we had a brainstorming session with the intentions of coming up with reasonable interaction patterns.        
By ``reasonable'', we imply that the underlying functionality should be relatively cheap to implement. The interactions should also be fun for the children to perform, in addition to being efficient.  

\begin{table}[H]
	\begin{tabular}{| p{3.0cm} | p{5.5cm} | p{5.5cm} |}
		\hline
		\textbf{Interaction Process} & \textbf{Rationale} & \textbf{Possible implementation} \\
		\hline
		Give \buddy{} a ``High five'' & Demonstrates to children that \buddy{} is friendly. It is intended that a child should keep \buddy{}'s arm up, and high five \buddy{} with the other arm & A gyroscope and a preassure sensor combined could verify that a high five has been received. \\
		\hline
		Hold \buddy{}'s hand & Demonstrates to children that \buddy{} is friendly. & Preassure sensor within the hand of \buddy{} could solve this. \\
		\hline
		Hold smartphone close to AsthmaBuddy's belly & Could demonstrate the ``smartness'' of \buddy{}, i.e. it can communicate with other things. & Could be solved by Bluetooth. \\
		\hline 
		Press \buddy{}'s nose. & Demonstrates to children that \buddy{} is friendly. & Preassure sensor within the nose of \buddy{} could solve this. \\
		\hline
		Press \buddy{}'s belly. & Same as above & Same as above \\
		\hline
		Hold medicine close to \buddy{}'s mouth & Demonstrates that \buddy{} also needs his medicine. & An RFID-tag attached to the medicine, and an RFID-reader inside the nose of \buddy{} could be used here to control the flow. \\ 
		\hline
		Hold RFID-tag close to \buddy{}'s mouth & Is a relatively easy way to proceed with the process. & A loose RFID-tag could be used together with an integrated RFID-reader, in order to proceed. \\ 
		\hline
		Hold RFID-tag close to \buddy{}'s belly & Same as above & Same as above \\
		\hline
		Clap your hands & Should be a fun way of interacting with systems, considering the age of our target group & Sound recognition could be used here. \\ 
		\hline
	\end{tabular}
	\label{tab:interaction-rationale}
	\caption{Rationale behind \buddy{}'s interaction design}
\end{table}

\subsection{Answering to Champoux's Development Framework}
\label{sec:answeringchampoux}
In Section \ref{sec:champoux} we described the development framework presented by Champoux \etal{}. When developing \buddy{}, we tried to answer the proposed questions that we considered relevant. 

\textbf{BO1: What should the user experience?}
The user should experience an interactive guide to applying an asthma treatment correctly. \buddy{} should give correct information in an understandable manner. 


\textbf{BO2: What are the human tasks?}
\begin{itemize}
  \item Fetch an adult
  \item Fetch inhaler and mask
  \item Shake medicine
  \item Attach the inhaler to the mask
  \item Put the mask around your mouth
\end{itemize}

\textbf{BO3: What should the artefact represent and control?}
\buddy{} represents a caregiver, who supervises children during their treatment. The artefact controls that children take their medicine at the correct time, and in a correct manner.   

\textbf{BO4: What are the conventions?}
Children have their inhaler and mask stored within a short distance of \buddy{}. Their RFID-tags are attached on their inhalers.    

\textbf{OC5a: What is the nature of the interaction for each sub task (Continuous vs Discrete vs Assembly)?}
The subtasks performed when taking a medicine is the following:
\begin{enumerate}
  \item Fetch parents
  \item Fetch inhaler
  \item Fetch mask
  \item Prepare medicine
  	\begin{enumerate}
  	  \item Shake the inhaler
  	  \item Attach inhaler to the mask
  	 \end{enumerate}
  \item Inhale dosage
  	\begin{enumerate}
  	  \item Hold medicine towards mouth
  	  \item Press the inhaler
  	  \item Breathe heavily for 10 seconds
  	 \end{enumerate}
  \item Optional, depending on the medicine: Rinse mouth
\end{enumerate}

Step 4 is an assembly task, 5(c) is a continuous task for a short period of time, while the remaining tasks are all discrete.  

\textbf{OC6: Does the sub-task need any relational interaction?}
None of the sub-tasks needs relational interaction.
[Pieter: Vil RFID-tag og RFID-leser her telle som en form for relational interaction?]
 
\subsection{Dealing with Bellotti's challenges}

Section \ref{sec:challenges-with-TUI} introduced some of the challenges that are exposed when designing tangible interfaces. Here, we'll discuss how we intend to handle the challenges presented, by answering his questions that we found relevant. 

\textbf{Address: How do I address one of many possible devices?}
One of the challenges mentioned here is ``How to not address the system''. This is an interesting challenge when the system is intended for children, as they like to pick things up and carry them around. \buddy{} will only start a treatment in two ways, either by an alarm firing or an RFID-tag attached to an inhaler is read. The RFID-reader is only capable of reading within a distance of 3-5 cm. Thus, as long as no alarms is fired, and an inhaler with attached RFID-tag is not withing the reach of the RFID-reader, the system should not respond.  

\textbf{Attention: How do I know the system is ready and attending to my actions?}
\buddy{} will as mentioned previously have a LED-light on it's nose. When this light is green, the user can expect that the system is running. 

\textbf{Action: How do I effect a meaningful action, control its extent and possibly specify a target or targets for my action?}
The part that regards specifying a target or several targets is considered irrelevant. The interesting part is how the user can effect a meaningful action and control its extent. This will be taken care of by interactions between the user and \buddy{}. \buddy{} will never run several instructions at once. It will always give short and clear instructions, and wait for feedback from the user in order to proceed. 

\textbf{Alignment: How do I know the system is doing the right thing?}
By listening when \buddy{} speaks, the user should be aware of what is going on. There is not a lot of room for human errors here. By following the normal sequence of operation, the worst thing that can happen is a system crash. This will be handled by shutting down the lights, and \buddy{} will not be running.  

\textbf{Accident: How do I avoid mistakes?}
Again, as long as the user is paying attention to what \buddy{} says, and does the intended interaction. A part of the challenge here is to recover from mistakes that has happened. For instance, if the user goes further proceeds further than was actually intended, and missed out on an instruction they needed, \buddy{} should have a way to go back to missed instruction. This was handled by having a separate ``back''-interaction method.
[TODO: Finn en interaksjon som kaller Back].   

\section{System Overview}

\subsection{Use Cases}
Figure \ref{fig:pi-use-cases} shows a general overview of the use cases we have included in our prototype. A medication process can be started in one out of two ways. 
A parent can register an alarm by using AsthmAPP. This alarm is then triggered by AsthmaBuddy, giving the child a notification that it is time to take their medicine.

The alternative is if children need to take their medicine by need. If they need to take a medicine by need, they simply register their RFID-tag before children are taken through a quicker process (see Manuscript \ref{chp:anuscript}).  

\begin{figure}[H] 
	\centering
		\includegraphics[width=0.8\paperwidth]{Pictures/usecases.png}
	\caption{AsthmaBuddy Use Cases}
	\label{fig:pi-use-cases}
\end{figure}

\subsection{Textual Use Cases}

%--------- TEXTUAL USE CASE ----------
%--------- BY NEED TREATMENT ---------
\begin{table}[H]
\begin{tabular}{|p{4.0cm} | p{9.0cm} |}
\hline
\textbf{Title} & By need treatment \\
\hline
\textbf{Preconditions} & - \\
\hline 
\textbf{Scenario} & 
	\begin{enumerate}
	  \itemsep0em
	  \item User triggers treatment by holding a specific RFID-tag close to AsthmaBuddy
	  \item System flashes LED-lights to notify user that the system is ready for use
	  \item System plays sound to instruct the user to shake the medicine
	  \item System plays sound to instruct user to mount the medicine on the mask and place the mask on his/her face
	  \item User starts a treatment by interacting with AsthmaBuddy (by pressing it's hand or similar interaction)
	  \item System plays sound to count during treatment (1-2-3-4-5-6-7-8-9-10), while flashing lights for each count
	  \item System plays sound to tell user he/she has done a good job
	  \item System calculates reward based on health state
	  \item System plays sound to award user with the calculated amount of stars
	  \item System plays sound to tell the user how many stars he/she has collected totally
	\end{enumerate}
\\
\hline
	\textbf{Extensions} & 
		x.a User aborts treatment by not continuing the sequence
\\
\hline
\end{tabular}
\caption{Textual use case: By need treatment}
\label{tab:textual-use-case}
\end{table}


%--------- PREVENTIVE TREATMENT -------

\begin{table}[H]
\begin{tabular}{|p{4.0cm} | p{9.0cm} |}
\hline
\textbf{Title} & Planned treatment \\
\hline
\textbf{Preconditions} & The current time corresponds with the time for a planned treatment \\
\hline 
\textbf{Scenario} & 
	\begin{enumerate}
	  \itemsep0em
	  \item The system recognizes the time for a planned treatment
	  \item The system starts blinking with LED-lights and playing sound to notify user
	  \item Child registers interacts with AsthmaBuddy, to notify that he/she is ready for the treatment
	  \item Start instructions by playing sound, telling the user to find a grown-up that can keep watch
	  \item System waits for interaction to make sure the user is ready
	  \item System tells the user to mount the medicine on the mask and put the medicine towards AsthmaBuddy's face
	  \item System plays sound to simulate breathing
	  \item System plays sound to tell the user how easy it was to take medicines and that it is the user's turn
	  \item System plays sound to instruct user, telling the user to shake the medicine
	  \item System waits for interaction to make sure the user is ready
	  \item System plays sound to instruct user to put the mask on his/her face
	  \item System plays sound counting to 10. 
	  \item System plays sound to tell the user he/she has done a good job
	  \item System calculates reward based on health state
	  \item System plays sound to award user with the calculated amount of stars
	  \item System makes a HTTPGet call to the server to find the total amount of stars collected
	  \item System plays sound to inform the user about how many stars the user has collected totally
	\end{enumerate}
\\
\hline
	\textbf{Extensions} & 
		x.a Child does not interact with AsthmaBuddy when prompted
\\
\hline
\end{tabular}
\caption{Textual use case: By need treatment}
\label{tab:textual-use-case}
\end{table} 


\subsection{State Diagram}
When the \rpi{} boots, it retrieves the latest version of the source code from Git, compiles it, and starts running it. If a child approaches \buddy{} and registers a RFID-tag, a medication sequence is started.  

\begin{figure}[H] 
	\centering
		\includegraphics[width=0.6\paperwidth]{Pictures/statediagram.png}
	\caption{AsthmaBuddy State Diagram.}
	\label{fig:asthmabuddy_statediagram}
\end{figure}

\subsection{Sequence Diagram}
Figures \ref{fig:ab-sd-byneed} - \ref{fig:ab-sd-completing-treatment} shows sequence diagrams of how the system works internally. Some abstractions have been made, in order to reduce the cluster of arrows. 

\paragraph{By Need Treatment}
We were not able to find a reasonable easy way to let the \buddy{} automatically be aware of the medicine that was to be taken at the start of a treatment. As a result, we used ssh into the \rpi{}, and provided the color of the medicine and the sequence number for the interaction that was to be tested.

After inserting these parameters, the LogicHandler retrieves a LinkedList of Interactions that is to be played. After this sequence is ended, the system jumps to \ref{fig:ab-sd-instructions}.
 
\paragraph{Planned Treatment}
If AsthmaBuddy is started without any parameters, it starts looking through alarm files (see Section \ref{sec:node-server}) every 60 seconds. If no alarm is returned from UserDataManager, it waits. Once an alarm is found, it AsthmaBuddy is notified through onNotificationFired, which starts the treatment process. 
 
\paragraph{Playing Instructions}
Playing instructions is mainly a loop of running a playing a sound, and turning on and off the LED-lights. 
\paragraph{Finishing a Treatment}
When a treatment is finished, i.e. we are out of the while loop in \ref{fig:ab-sd-instructions}, we register the treatment in the database. This ensures that children are able to see the rewards in AsthmAPP. 


\begin{sidewaysfigure}
	\centering
		\includegraphics[scale=0.6]{Pictures/sd/sd-byneed.png}
	\caption{By Need Treatment - Sequence Diagram}
	\label{fig:ab-sd-byneed}
\end{sidewaysfigure}

\begin{sidewaysfigure}
	\centering
		\includegraphics[scale=0.6]{Pictures/sd/sd-planned-treatment.png}
	\caption{Planned Treatment - Sequence Diagram}
	\label{fig:ab-sd-planned-treatment}
\end{sidewaysfigure}

\begin{sidewaysfigure}
	\centering
		\includegraphics[scale=0.6]{Pictures/sd/sd-instructions.png}
	\caption{Playing Instructions - Sequence Diagram}
	\label{fig:ab-sd-instructions}
\end{sidewaysfigure}

\begin{sidewaysfigure}
	\centering
		\includegraphics[scale=0.6]{Pictures/sd/sd-complete-treatment.png}
	\caption{Finishing a treatment - Sequence Diagram}
	\label{fig:ab-sd-completing-treatment}
\end{sidewaysfigure}

\subsection{Node Server}
\label{sec:node-server}
In addition to the Java application running on the \rpi{}, we developed a Node.js server\fnurl{Node.js}{http://nodejs.org/}. This backend system was developed in order to easily visualize the rewards given to a child. The initial problem is that AsthmAPP stores data to a MySQL database, with childId as the primary key for most tables. Initially, \buddy{} has no way of knowing which childId to use in order to store the rewards. The current solution to our problem was to develop a Node.js server on AsthmaBuddy, which run as a background process. Whenever we want to switch users, AsthmAPP does an HTTP POST to this server, including the childId as a parameter. The server then retrives JSON-formatted data from a webservice, which includes the rewards a child has been given until now (for instance, by use of the smartphone), and the alarms set for this user. 
When \buddy{} starts running, it checks for alarms to be fired every 60 seconds. When a child has finished a treatment, AsthmaBuddy updates the database, with the childId previously retrieved, and the amount of stars a child gathered during his or her treatment. Additionally, with the data retrieved from the database, \buddy{} has the capability to tell the user how many stars a child has gathered\footnote{Since this is a prototype, this functionality only works until a child has gathered 20 stars. It became cumbersome to handle rewards totalling more than 20 stars}. 


 

\section{Prototype version 1}
\label{sec:proto1}
Our first prototype was a ``dumb'' version with the purpose of testing out different methods of interaction. We wanted to get as many different experiences as possible, in order to find what kids are thinking as ideal. Even though some of the interaction methods are infeasible to implement during the thesis, we wanted to know what children think of as fun interaction. We also found it important that children were motivated to give us honest results, so we wanted to give them freedom to be innovative and let their imagination play a part.  
%TODO: Skrive om
[TODO: Change to something useful, and actual\ldots] As a result, the first prototype was a simple state machine. The usability test was performed on NSEP's usability lab. Terje R\o sand sat in the backroom filming the sequence, while having a computer with ssh-connection to AsthmaBuddy. Whenever he observed that the children performed the method of interaction, he pressed enter in order for AsthmaBuddy to progress it's medication sequence. 

For the first prototype we tried out the following types of interaction with AsthmaBuddy:
\begin{enumerate}
	\item{Give AsthmaBuddy a ``High Five''}
	\item{Hold AsthmaBuddy's hand}
	\item{Hold smartphone close to AsthmaBuddy's belly}
	\item{Press AsthmaBuddy's nose}
	\item{Press \buddy{}'s belly}
	\item{Hold medicine close to AsthmaBuddy's mouth}
	\item{Hold RFID-tag close to \buddy{}'s nose}
	\item{Hold RFID-tag close to \buddy{}'s belly}
	\item{Clap your hands}
	\item{A variation of the above interactions}
\end{enumerate}

The prototype was not connected to the database. In order to show children their rewards, we simply added some stickers to a sheet of paper. 



\subsection{Information Gathering}

During the development phase we interviewed a couple of domain experts and parents in order to gather information about how they perform asthma treatments. 

\subsection{Evaluation of prototype version 1}

\paragraph{Evaluation of interaction methods}
\label{chp:interaction-methods}
\begin{table}[H]
\begin{tabular}{|p{5.0cm} | p{6.0cm} | p{3.0cm} |}
\hline 
\textbf{Interaction Method} & \textbf{Comments} & \textbf{Will it be in version 2?}\\
\hline
	Give AsthmaBuddy a ``High Five'' & & \\
\hline
	Hold AsthmaBuddy's hand & & \\
\hline
	Hold smartphone close to AsthmaBuddy's belly & & \\
\hline
	Press AsthmaBuddy's nose & & \\
\hline
	Press \buddy{}'s belly & & \\
\hline
	Hold medicine close to AsthmaBuddy's mouth & & \\
\hline
	Hold RFID-chip close to \buddy{}'s nose & The thickness of \buddy{}'s nose made it difficult for the RFID tag to communicate with our reader, this caused problems for the user. & No \\
\hline
	Hold RFID-chip close to \buddy{}'s belly & & \\
\hline
	Clap your hands & & \\
\hline
	A variation of the above interactions & & \\
\hline
\end{tabular}
\caption{Evaluation of interaction methods for AsthmaBuddy}
\label{tab:interactioneval}
\end{table}

\paragraph{Operating time}
The longest time for taking a planned treatment was about 2 minutes for an experienced user, and about 2.5 minutes for an inexperienced user. This treatment included the part where \buddy{} completes the treatment before the child. [SETT INN HVOR LANG TID DET TAR TIL VANLIG]. [SETT INN SAMMENLIGNING]. [VURDER OG KONKLUDER].     

A By Need treatment was registered to take about a minute for an experienced user. Usually, it takes approximately [XX] minutes to do a By Need treatment \footnote{Information gathered from parents}. [SETT INN SAMMENLIGNING]. [VURDER OG KONKLUDER].  


\paragraph{Feedback}
%Potential improvements
%New Interaction methods
%


\section{Prototype version 2}


\section{Evaluation}



\section{Further improvements}

 
%\subsection{Dealing with Bellotti's Challenges}
%\label{sec:handling-challenges}
%The main challenges Bellotti has identified (see Table \ref{tab:tuichallenges}) is \emph{``How to control or cancel system action in progress''} and \emph{``How to specify and select a possible object for action''}. We have previously stated that there are two cases in which a child will take their medicine, \emph{by need} and \emph{as a preventive measure}. An issue that could rise is how to detect when a child needs to take a medicine by need. Preventive medicines will fire alarms on AsthmaBuddy, and the child automatically knows that the system is ``awake''. However, when a child needs to take it by need, there is no other intuitive way to show that the AsthmaBuddy is awake than using LED-lights, which are the key for the system to work. Thus, the solution to the second challenge is usage of LED-lights. Cancelling operations, however, is difficult to find a ``good'' solution from a usability standpoint. 

%For instance, if AsthmaBuddy starts counting seconds on a treatment before a child is ready, there will be a need to reset this counter, or basically at any point during a treatment, go back to the last step. At the moment, the only solution we are able to see at this problem is to use RFID-chips to backtrack, modelling the treatment process as a statemachine. 
 