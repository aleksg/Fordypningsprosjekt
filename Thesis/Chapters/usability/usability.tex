\chapter{Usability}
\label{chp:usability}

This chapter will give a brief definition of what usability is, and how user tests can help us improve it. Since the applications are targeted towards both children and adults, we will give a description of how the usability tests for these groups will differ. We will also explain how the user testing was performed at \ldots

\section{What is usability?}
\label{sec:usability}
There are many ways to describe usability. 

The International Organization for Standardization(ISO) uses the following definition of the term usability \cite{isousability}:

\textit{Extent to which a system, product or service can be used by specified
users to achieve specified goals with effectiveness, efficiency
and satisfaction in a specified context of use.}

The same document defines the context of use as:

\textit{Users, tasks, equipment (hardware, software and materials), and
the physical and social environments in which a product is used.}

These definitions cover how the system is used, the user's thoughts about the use and the context of the system. This can be broken down further into several subgoals in order to achieve better usability, and to give a better insight as to what usability is. 
These subgoals are:

\begin{enumerate}
\item{How precisely is the user able to perform a task by using the application?}
\item{How much resources(for example time, or number of tries) was used to perform the given task using the application?}
\item{How many errors occurred?}
\item{Did the user find the use satisfactory?}
\end{enumerate}

User-centered design is a way of designing with the user in mind. By using this technique these goals are achievable. User-centered design is about getting feedback from the users during the design and development process. Always thinking about how the user would solve this problem, and consolidate the users when in doubt is a fundamental part of user-centered design. The user's opinion is the measure of how good the system performs and the user's feedback defines how you score on usability. [Should have reference]

\section{How to test usability}
\label{sec:howtotestusability}
There are many ways to create a good user experience. Having knowledge of expert opinions is always a good idea, and using user-centered design techniques is also a wise way to go. According to SOME PERSON [Insert Reference] developers should get feedback from users by users tests at different stages of development. According to SOME PERSON [Insert reference], having a user-centered approach will help the developers to address the weakest parts of their system, and give feedback on design decisions. 

A user-centered design can be done in many different ways and at different stages of the product life cycle \cite{abrasusercentereddesign}, as shown in Table \ref{table:designduringlifecycle}:

\begin{table}[H]
\begin{tabular}{|p{5cm} | p{5cm} | p{5cm} |}
\hline
\textbf{Method} & \textbf{Purpose} & \textbf{Phase of the project lifecycle} \\ \hline
Background interviews and questionnaires & To collect data and to understand the user better & When starting the project \\ \hline
Focus groups & Discover design issues and receive feedback & At an early stage \\ \hline
On-site observation & To both collect information of the context the system will be used in, and find the primary problems the users may have & At an early stage \\ \hline
Role playing / simulations & Will give a broader understanding of what the user expects from the system & Early to mid stage of the project \\ \hline
Automated evaluation & Gives feedback on deviations from standards or best practices. This method excludes actual users, but is based on well tested principles & Mid to end of the project \\ \hline
Usability testing & To measure the usability of the system and provide feedback on very specific elements that are badly designed & Abras \cite{abrasusercentereddesign} says it should be at the end of the project while others \cite{schneidermanusercentered} think it should be done in iterations throughout the project. \\ \hline
Interviews and questionnaires & Gives a qualitative measurement of how good or bad the system is & End of the project \\ \hline
\end{tabular}
\caption{Methods of user-centered feedback}
\label{table:designduringlifecycle}
\end{table}

The purpose of this project is to test an existing system, improve the existing product and plan an extensive testing of the improved product. We will focus mainly on WHAT WHAT WHAT?

\paragraph{Usability Testing}
The purpose of usability testing is to increase the usability of a system. At the same time, performing these usability tests may save the developers some time and reduce the cost of the project by removing errors and poor design at an early stage \cite{dumas1995practical}.

The usability testing can be performed in different ways \cite{schneidermanusercentered}. At the early stages of the project, low-fidelity prototypes are a good option since they will provide feedback and take proportionally little time to make, making it easier to have more iterations of testing. The different testing methods include a potential user of the system performing tasks to provide real data. Observing and recording each usability test may help the developers to analyze their system, and correct the flaws \cite{dumas1995practical}. 

Before starting the usability tests, the developers should set goals planning what they want to know about the system \cite{isosoftwareengineering}. This will ensure that the purpose of the test is fulfilled. The developers should then plan tasks according to the desired results. These tasks should allow the user to explore the system, or the parts the developers wish to test, giving the test person some time per task, in order to not stress the test person. 

After being planned, the test should be run on a number of different test persons. From figure %\ref{figure:numberoftests}
, you can see that as the number of participants increases, the number of undetected errors decrease. 

%%
%\begin{figure}[H]
%\begin{center}
%\includegraphics{Thesis/Pictures/numberoftests}
%\end{center}
%\caption{Number of users needed to find percentage of errors[Insert reference]}
%\label{numberoftests}
%\end{figure}

Nielsen states that after five user tests, 85\% of the errors have been found \cite{nielsennumberoftests}. Molich\cite{molich2008usable} states that six test persons is the ultimate number. Faulkner \cite{faulkner2003beyond} states that while six test persons may find 85 \% of the errors, they may also find considerably less. In Faulkner's research, a number of five test users found between 55 \% and 100 \% of the errors, while 20 test persons found between 95 \% and 100 \% of the errors. 

\paragraph{Testing environment}
\label{par:testingenvironment}
The next thing to consider when performing usability testing is the testing environment. It should resemble the environment in which the system will be used. To make the most of the tests, it is wise to perform videotaping of the tests. This will help when reviewing the results from the test[insert references]. If the test are being recorded, a consent from the test person or his/hers guardian will be required.

Before the test persons arrive, a test leader should be chosen, in order to have a person to guide the test persons through the process. The test leader should be in charge of testing and act as an interviewer to help the participant to ``think-aloud''\footnote{Reference to Thinking aloud}. The test leader should answer questions from the participant, but be careful not to give away information that may affect the results of the test.

After the tasks are done, it is necessary to gather loose ends and get answers to all the questions that might be unanswered. A system usability scale(SUS)\cite{sus} may be a good way to grade the usability of the system together with the observations made during the test. The SUS scale will reflect on how satisfying the usability is in the eyes of the users. Bangor et al \cite{susform} have made a scale based on the SUS-forms from different system usability tests, in order to make it possible to compare the mean score of a system with what is an acceptable level of usability. In our testing, we will make use of a Norwegian version, developed by Svan\ae s \ref{app:norsksus}.

\section{How to test usability on children and toddlers}
\label{sec:usabilitytestchildren}
While usability testing on children and toddlers have the same basic approach as testing on adults, there are many more precautions to be followed. 
Hanna et al. \cite{testingenvironmentforchildren} lays out some of these precautions. They recommend not using children that are skilled with computers since they may find the tasks too easy and will not produce useful data. 
Since children these days have a higher skill with computers thanks to the invasion of tablets and smart phones [insert reference?], this may not be as much of a concern. 

Since our application is targeted towards children with Asthma, we want to test the system on children suffering from Asthma in addition to children from the same age group, not suffering from Asthma. These children will most likely have a different approach to the system and may give different feedback.

Hanna, Risden and Alexander also point out changed that should be made to the testing environment as mentioned in \ref{par:testingenvironment}. They recommend making the testing environment more suitable for children by placing colourful posters on the walls.
Children of young age may be afraid of ``The Doctor's Office'' and we will need to make adjustments to avoid frightening the children upon their arrival at the test lab. 

As mentioned by Donker and Markopoulos \cite{TalkAloud} talk-aloud is very useful technique when doing usability testing with children. Talk-aloud is a technique were the children talk about what they are doing instead of what they are thinking.

%THIS NEEDS MORE WORK

\section{Usability testing on mobile devices}
\label{sec:usabilitytestonmobiledevices}
We plan on doing usability testing in testing lab or quiet testing office. The application's main environment for use will be at the user's home, which may be noisier and more hectic than our testing lab. Kaikkonen et al \cite{kallio2005usability} states that the similarity between testing environment and place of use is not too important, the test user will still be able to complete the tasks and find the same number of errors. This claim is supported by Beck et al \cite{beck2003experimental} who discovered that the test persons found more usability problems when sitting down, in difference to when walking on the street. 

Schusterich et al \cite{schusteritsch2007towards} published a guide on how to build the perfect infrastructure for usability testing on mobile devices, in 2007. They describe how generic infrastructure issues, mobile device-specific issues and usability study context issues should be taken into consideration. Shcusterich et al. recommends having a number of cameras recording from different angles in order to capture unbiased interaction patterns of the mobile device. 

\subsection{Emulator versus device}
When developing for mobile devices such as Android, it is possible to run an emulator on a pc, instead of running the application on an Android device. The emulator emulates the use of a mobile device on screen. Input must be given by mouse-clicks on buttons/screen elements. The Android emulator emulates use of system resources corresponding to a given Android device, in order to not act faster or slower that a real device. While this may be true in theory, it is not always true in practice. The emulator is often much slower than an actual device, a claim supported by Lin et al\cite{lin2011benchmark}, who found that native code will run up to 34.2\% faster than on the emulator.


\begin{figure}[H]
\begin{center}
\includegraphics{Thesis/Pictures/androidemulator}
\end{center}
\caption{The Android Emulator running an emulation of Android 4.1}
\label{fig:androidemulator}
\end{figure}

The question arises, should one do usability testing on an emulator or on a real device?
Using the emulator allows for easier capture of the interaction with the device, since it allows screen recording of user input and easier capture of the user when interacting with the emulator. Using a computer as a test object may be more positively perceived by the test user rather than installing the test application on their device or having them doing tests on our device. 

Using a real device has the benefit of being an actual device, and may lead to a more realistic interaction pattern when using the application. While there exists a number of screen recorders for Android devices, these require that the device is ``rooted'' \cite{androidrooting}, which is not an option for us. 
The use of a real device also allows the use of gestures, which is a benefit in contrast to the emulator which can simulate swiping. 


Beitol and Cybis \cite{betiol2005usability} compared doing usability testing on a tripod-mounted device to an emulator and having a in the field. They found that many users found the tripod-mounted device difficult and unnatural to operate. The users found 80\% of the usability issues on the emulator, but Beitol and Cybis points out that use of an emulator may depend on the similarity between the emulator and the device. 

Based on the research we have read we decided to do the usability testing on a real device.
