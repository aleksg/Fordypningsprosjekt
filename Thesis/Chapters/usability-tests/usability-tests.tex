\chapter{Usability Tests}
\label{chp:usabilitytests}


\section{Purpose}
\label{sec:usabilitypurpose}
The usability tests will be performed in order to provide usability feedback on the application. These tests will be created to test and discover the usability problems in the at that time current version of [INSERT APPNAME]. The tasks given to the participants will be created with routine use of the application in mind. Usability tests will be performed with the help of participants with no prior knowledge of the application. These participants will be chosen in order to get valuable feedback on usability problems with the current design and structure, and to prevent invalid feedback from users who already know how to perform the tasks. In addition, this situation mirrors everyday life of the users.


\section{Test Method}
The execution of the usability tests will be based on the theory described in Section \ref{sec:howtotestusability} and Section \ref{sec:usabilitytestchildren}. The usability tests are not conducted for a scientific research of the usability, but to ensure that errors in the application will not affect the use of the system and our research on gamification and tangible user interfaces.

We choose to not do usability testing exclusively on our target group, parents with children suffering from asthma. The main reason for this is time concern, and the goal of our application being suited and having good usability for all users, disregarding their knowledge of asthma. Our test persons will be mainly students selected from Bachelor or Master students at NTNU, with none computer science subjects. The computer science students will be excluded due to the fact that they are more used to graphical user interfaces and applications at a ``beta stage'', which may have given invalid data. The tests will be run in 2 iterations on 5-6 test persons on each iteration. 

Before of each usability test, we will perform a quick-and-dirty pilot test in order to discover critical errors that could make an impact on the result.

To ensure that the participants have the wanted background, we will ask them to fill in two forms when registering for the usability tests. These are added in \ref{app:questionnaire} and \ref{app:interviews-before-usability-testing}.


The participants will be given an Android mobile device to perform the tasks on. The different tasks will be given one by one in order to complete. The participants will be introduced to the ``think-aloud''-method, and will be told to ask questions during the process, even though the test leader is not allowed to answers these questions during the test. The main reason for gathering questions is for discussion afterwards or to facilitate the ``think-aloud''-method. 

Upon finishing the tasks, the participants will be asked to answer the forms in Appendix \ref{app:interviewafter} and \ref{app:norsksus}. The test leader will finish the test by asking questions regarding what the participants thought of the system and answer the questions that may have occurred during the test. 

The result will later be analyzed to work out the improvements needed to be done to the system. The errors will be rated after level of severity \cite{dumas1995practical}. 

\begin{itemize}
\item{Critical (Level 1) - Prevents the participant from completing the task.}
\item{Siginificant (Level 2) - Generates significant problems when trying to complete the task.}
\item{Minor (Level 3) - Have minor effect on the usability of the application.}
\item{Non-essential (Level 4) - Enchancements to the system. When a participant states that ``it would be nice to have this''.}
\end{itemize}



\subsection{Scenario and tasks given to the users}
We plan to use test users that speak fluent Norwegian, since the application has Norwegian as it's main language. Therefore the scenario and tasks will also be written in Norwegian. The exact scenario and tasks handed to the participants can be found in Appendix \ref{app:scenarioandtasks}, but for convenience the next paragraph gives a short summary of the scenario and tasks.

The scenario explained that the user was a guardian of a 4-year-old child with asthma. They have recently been to the doctor's office, and will now have to set up treatment plans according to advice given by the specialist. Since they have little experience with asthma, they would have to look up information about the medicines and how the treatment will be done. In order to motivate the child to continue taking his/hers medicine, they will have to add a reward via the application menu. Finally, they would have to look through the calendar log in order to find correlations between the child's health state and use of medicines. 