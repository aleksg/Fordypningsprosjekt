\chapter{Usability Tests}
\label{chp:usabilitytests}

This chapter will describe the purpose of the tests and how they were executed. Finally it will cover the observations and results of these tests.


\section{Purpose}
\label{sec:usabilitypurpose}
The usability tests were performed in order to provide usability feedback on the application. These tests were created to test and discover the usability problems in the at that time current version of [INSERT APPNAME]. The tasks given to the participants were created with routine use of the application in mind. User tests were performed on participants with no prior knowledge of the application. These participants where chosen in order to get valuable feedback on usability problems with the current design and structure, and to prevent invalid feedback from users who already knew how to perform the tasks. In addition, this situation mirrors everyday life of the users.


\section{Research Method}
The execution of the usability tests was based on the theory described in Section \ref{sec:howtotestusability} and Section \ref{sec:usabilitytestchildren}. We choose to not do usability testing exclusively on our target group, parents with children suffering from asthma. Main reason for this was time concern, and the goal of our application being suited and having good usability for all users, disregarding their knowledge of asthma. Our test persons were mainly students selected from Bachelor or Master students at NTNU, with none computer science subjects. The computer science students were excluded due to the fact that they are more used to graphical user interfaces and applications at a ``beta stage'', which may have given invalid data. The tests were run in XXX iterations on 5-6 test persons on each iteration. 

Before of each usability test, we performed a quick-and-dirty pilot test in order to discover critical errors that could make an impact on the result.

To ensure that the participants had the wanted background, they were asked to fill in two forms when registering for the usability tests. These are added in [INSERT REFERENCE].

%SOMETHING ABOUT IT BEING FILMED OR WHATEVER.


The participants were placed at a table and given an Android mobile device to perform the tasks on. The different tasks were given one by one in order to complete. The participants were introduced to the ``think-aloud''-method, and were told to ask questions during the process, even though the test leader was not allowed to answers these questions during the test. The main reason for gathering questions was for discussion afterwards or to facilitate the ``think-aloud''-method. 

Upon finishing the tasks, the participants were asked to answer the forms in [INSERT REFERENCE] and \ref{app:norsksus}. The test leader finished the test by asking questions regarding what the participants thought of the system and answered the questions that may have occurred during the test. 

The result was later analyzed to work out the improvements needed to be done to the system. The errors was rated after level of severity \cite{dumas1995practical}. 

\begin{itemize}
\item{Critical (Level 1) - Prevents the participant from completing the task.}
\item{Siginificant (Level 2) - Generates significant problems when trying to complete the task.}
\item{Minor (Level 3) - Have minor effect on the usability of the application.}
\item{Non-essential (Level 4) - Enchancements to the system. When a participant states that ``it would be nice to have this''.}
\end{itemize}


\section{Participants}
\label{sec:participants}
Table \ref{tab:participants-table} shows the answers to the forms in [INSERT REFERENCE], which covers information about them and the knowledge they had with computer usage and asthma.

\begin{sidewaystable}
	\label{tab:participants-table}
	\begin{tabular}{ | p{3.5cm} | p{3.0cm} | p{3.0cm} | p{3.0cm} | p{3.0cm} | p{3.0cm} | p{3.0cm} |}
	\hline
	\textbf{Participant} & P1 & P2 & P3 & P4 & P5 & P6 \\ \hline
	\textbf{Gender} & & & & & & \\ \hline
	\textbf{Age} & & & & & & & \\ \hline
	\textbf{Education} & & & & & & \\ \hline
	\textbf{Experience with computers} & & & & & & \\ \hline
	\textbf{Access to internet} & & & & & & \\ \hline
	\textbf{Time spent online per day} & & & & & & \\ \hline
	\textbf{Has a smartphone} & & & & & & \\ \hline
	\textbf{SMS Usage} & & & & & & \\ \hline
	\textbf{Has facebook account} & & & & & & \\ \hline
	\textbf{Uses electronic reminders (calendar, to-do list etc)} & & & & & & \\ \hline
	\end{tabular}
	\caption{Participants' experience with IT solutions}
\end{sidewaystable}


Table [INSERT REFERENE] contains answers the users wrote down after completing the test.

\begin{sidewaystable}
	\label{tab:participants-evaluation-table}
	\begin{tabular}{ | p{3.5cm} | p{3.0cm} | p{3.0cm} | p{3.0cm} | p{3.0cm} | p{3.0cm} | p{3.0cm} | }
	\hline
	\textbf{Participant} & P1 & P2 & P3 & P4 & P5 & P6 \\ \hline
	\textbf{What was difficult?} & & & & & & \\ \hline
	\textbf{Why was it difficult?} & & & & & & \\ \hline
	\textbf{Have you used similar systems?} & & & & & & \\ \hline
	\end{tabular}
	\caption{Participants's opinions upon finishing the test}
\end{sidewaystable}



\subsection{Scenario and tasks given to the users}
Since both the system and the test users spoke Norwegian, the scenario and tasks were given in Norwegian. The exact scenario and tasks handed to the participants can be found in Appendix [INSERT REFERENCE], but for convenience the next paragraph gives a short summary of the scenario and tasks.

The scenario explained that the 

\section{Observations and Results}

As described in Section \ref{sec:usability}, usability can be a combination of effectiveness, efficiency and satisfaction. The sections \ref{subsec:effectiveness}, \ref{subsec:efficiency} and \ref{subsec:satisfaction} will give answers to how well the system performed compared to these components. The most interesting observations are summarized in Section \ref{subsec:specificproblems}.

\subsection{Effectiveness}
\label{subsec:effectiveness}


\subsection{Efficiency}
\label{subsec:efficiency}


\subsection{Satisfaction}
\label{subsec:satisfaction}

The participants satisfaction when using the application was adressed through the System Usability Scale-forms. Table \ref{tab:participants-sus-score} shows the result from the participants SUS-forms after calculating the scores.

\begin{center}
	\begin{tabular}{| p{2.5cm} | p{1.5cm} | p{1.5cm} | p{1.5cm} | p{1.5cm} | p{1.5cm} | p{1.5cm} | p{1.5cm} | }
	\label{tab:participants-sus-score}
	\hline
	\textbf{Participant} & P1 & P2 & P3 & P4 & P5 & P6 & Mean \\ \hline
	\textbf{1} & & & & & & & \\ \hline
	\textbf{2} & & & & & & & \\ \hline
	\textbf{3} & & & & & & & \\ \hline
	\textbf{4} & & & & & & & \\ \hline
	\textbf{5} & & & & & & & \\ \hline
	\textbf{6} & & & & & & & \\ \hline
	\textbf{7} & & & & & & & \\ \hline
	\textbf{8} & & & & & & & \\ \hline
	\textbf{9} & & & & & & & \\ \hline
	\textbf{10} & & & & & & & \\ \hline
	\textbf{Score} & & & & & & & \\ \hline
	\end{tabular}
\end{center}


\subsection{Specific problems}
\label{subsec:specificproblems}