\section{Existing Research}
\label{sec:existing-research}

This section will give a foundation on some of the reserach performed on using technology in combination with deceases and children. 


\subsection{Monitoring your own decease with mobile technology}
There exists some research on self-management of monitoring your asthma condition. A lot of this research works used SMS (Short Messaging System) technology. In 2009, Andh\o j and M\o ldrup et. al.\cite{anhoj2004feasibility} did a feasability study to check how users would react to a SMS-reminder study. Their methodology were to send SMS a couple of times a day, and have the users respond to their peak flow and answer yes/no questions. Users could then access a web page to see different statistics on peak flows, how they've felt the last couple of days, etc.

They concluded that SMS is a feasible solution for collecting asthma diary data, mainly because the SMS technology was a big part of the participant's everyday life. Although SMS is a great technology to be used for this purpose, few children in our target group are able to use this technology, for obvious reasons. According to \emph{Senter for IKT i utdanningen}(Center for ICT in education), about 40\% of norwegian children below the age of 3 years old have used a tablet, and 6 out of 10 children below the age of 6 have used a touch screen device \cite{nrkchilduse}. Thus the technological background should be somewhat familiar for our target group. 

\subsection{Children and mobile devices}
In 2013, babies.co.uk posted results on a poll they had posted on how many toddlers are using smartphones or tablets each day\cite{babiesusageoftablets}. Over 1000 participants responded,  and according to the survey, 14\% of the responders allowed children to use smartphones or tablets more than 4 hours a day. Considering the normal awake time of a child between 9 and 12 months old is approximately 10 hours, they spend a considerable amount of their day on the smartphone.        


\subsection{Children and gestures}

Abdul Aziz et. al. \cite{aziz2013children} performed a study on which gestures children are able to comprehend when playing with an iPad. She tested 33 children's abililty to do gestures on a variety of applications suited for children. The children were in the range of 2-12 years old, 3 children per age. The study showed the following restrictions:

\begin{itemize}
  \item 2 year old children have difficulties with pinching, and are unable to drag and drop, spread and rotation of the device, and are not able to focus on the application. 
  \item 3 year old children have difficulties to drag and drop until they are told to do so, in addition to having problems with pinch and spread. 
  \item 4 year old children have difficulties to drag and drop. 
\end{itemize}
Children at age 5 and above are able to do all the normal gestures at a tablet. As CAPP is currently only available for mobile devices, this is reason for some discussion. The main part to notice is pinching and drag and drop. 

%TODO: Utdyp mer!
Now, are these difficulties only problems regarding the tablet size, or do they also arise on mobile phones? An iPad is fairly large relative to the size of these children's hands. The application [INSERT APPNAME] only uses ``swiping'' gestures and button presses for navigation.

