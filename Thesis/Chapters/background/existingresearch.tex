\section{Existing Research}
\label{sec:existing-research}

This section will give a foundation on some of the research performed on using technology in combination with diseases and children. 


\subsection{Monitoring Asthma with Mobile Technology}
Research on self-management of monitoring one's asthma condition has already been carried out. Much of this research was carried out by using SMS (Short Messaging System) technology. In 2009, Andh\o j  et. al.\cite{anhoj2004feasibility} did a feasability study to check how users would respond to a SMS-reminder. Their methodology was to send SMS a couple of times a day, and have the users respond to their peak flow and answer yes/no questions. Users could then access a web page to see different statistics on peak flows, how they've felt the last couple of days, etc.

Andh\o j \etal{} concluded that SMS is a feasible solution for collecting asthma diary data, mainly because the SMS technology was an important part of the participants' everyday life. Although SMS is a great technology to be used for this purpose, few children in our target group are old enough to use this technology. According to \emph{Senter for IKT i utdanningen} (Center for ICT in education), about 40\% of Norwegian children below the age of 3 years old have used a tablet, and 6 out of 10 children below the age of 6 have used a touch screen device\cite{nrkchilduse}. Thus our target group is likely to be familiar with the technology we plan to use.  


\subsection{Children and Mobile Devices}
In 2013, \url{www.babies.co.uk} posted results of a poll they had posted on how many toddlers are using smartphones or tablets each day\cite{babiesusageoftablets}. Over 1000 participants responded,  and according to the survey, 14\% of the responders allowed their children to use smartphones or tablets more than 4 hours a day. Considering the normal awake time of a small child, the children spent a considerable amount of their day playing with a smartphone. With this in mind, we aim to make an application that is used for a short period of time with each use. AsthmAPP and AsthmaBuddy will be tools for helping the children, not a toy or a game.


\subsection{Children and Gestures}
\label{sec:childrenandgestures}
Abdul Aziz et. al.\cite{aziz2013children} performed a study on which gestures children are able to comprehend when playing with an iPad. They tested children's ability to gesticulate on a variety of applications suited for children. The children were between the age of 2 - 12, three children in each age group. The study showed the following restrictions:

\begin{itemize}
  \item 2 year olds have difficulties with pinching and are unable to drag and drop, spread and rotation of the device, and are not able to focus on the application. 
  \item 3 year olds have difficulties with drag and drop until they are told to do so, in addition to having problems with pinch and spread. 
  \item 4 year olds have difficulties with drag and drop. 
\end{itemize}

In order to make AsthmAPP as child friendly as possible, it only uses ``swiping'' gestures and button presses for navigation.
[INSERT MORE CONCLUSIONS??]


\subsection{Assessment of Existing Asthma Applications}
In 2012, Huckvale et. al.\cite{huckvale2012apps} conducted an assesment on the existing asthma-related applications on both Google Play\fnurl{Google Play}{http://play.google.com} and App Store\fnurl{Apple App Store}{http://www.apple.com/itunes/features}. They assessed 103 different apps with english as the native language. Out of these applications, 

\emph{``No apps for people with asthma combined reliable, comprehensive information about the condition with supportive tools for self­management''}\cite{huckvale2012apps}. 

They concluded that doctors should be careful when recommending apps for patients with the purpose of self management of asthma.