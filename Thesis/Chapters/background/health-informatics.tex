\section{State of the art}
\label{sec:stateoftheart}
Mobile computing is evolving at a rapid pace, and finding new ways to use it in health care is a rising challenge for research. This section covers the state of the technological development of some of the areas in which mobile technology is being used with a combination of either gamification or tangible interfaces.   


\subsection{Get Up and Move (GUM)}
\label{sec:gum}
Penados et. al.\cite{penadosget} created GUM; an interactive toy to measure and stimulate physical activity. GUM is a small creature that needs to be taken care of by a child. The child's objective is to make his/her GUM healthier and happier by moving with it, feeding it and playing with it. GUM is healthy and happy when it has been through a minimal amount of daily physical activity, and since GUM can not move by itself, the child needs to do it. As GUM grows healthier, lighted stars will appear in its ears, until it reaches a maximum healthy state. To increase the number of stars, the child needs to progressively increase and later maintain the GUM's physical activity level. Penados et. al. argues that GUM had a positive effect on reducing sedentary behaviour and motivate physical activity with young children.


\subsection{Sisom}
\label{sec:sisom}
Sisom\fnurl{Sisom}{http://www.communicaretools.org/sisom/} is a software created to increase the communication level between physicians and children. It is an interactive game, where the user follows an avatar through different ``worlds'' of health care subjects. For instance, the avatar takes a boat to a hospital. The user can look around in the hospital and express how he/she feels when giving a blood sample. The results showed that when a child played the game before a consultation with his/her physician, the child was better prepared, the communcication had a better quality, and the child participated more during the consultation\cite{sisom-research}.


\subsection{Meassuring Blood Pressure}
\label{sec:bloodpressure}
iHealth\fnurl{iHealth}{http://www.ihealthlabs.com/wireless-blood-pressure-monitor-feature\_32.htm} , Withings\fnurl{Withings}{http://www.withings.com/en/bloodpressuremonitor/features} and other companies has created blood pressure monitors which are synchronized with mobile applications. By using a wrist monitor, heartbeat, blood pressure level and pulse wave is meassured and stored in the application. The application visually presents graphs of the user's historic blood pressure levels and tracks progress. The application also allows for sharing meassurements with friends, family and doctors. By sharing detailed information with a doctor, a more accurate treatment plan may be laid out.


\subsection{Controlling Your Diabetes}
\label{sec:controldiabetes}
Cellnovo\fnurl{Cellnovo}{http://www.cellnovo.com/} has created a system that helps to control diabetes. It consists of a handheld device for meassuring blood sugar levels, a pump that controls the flow of insulin and a web interface that allows one to access the information. The interface helps users to check for trends and patterns in their blood sugar level, which again motivates users to continue applying the correct treatment. It also allows users to send information to physicians, which helps them make decisions regarding patient care.
      


\subsection{Quit Smoking}
\label{sec:quitsmoking}
There are lots of mobile applications that help people to quit smoking. For instance, \emph{The Norwegian Heart and Lung Patient Organization} has developed an app called ``'R\o ykeslutt''\fnurl{R\o ykeslutt}{https://play.google.com/store/apps/details?id=no.lhl.roykeslutt}. The application shows what the body is going through after a specific amount of days after the user has stopped smoking, which according to the reviews of the application, is a huge motivational factor. Additionally, they show how much money a user has saved at any given time, which can be seen as ``gamifying'' the element of money in order to motivate users.  


\subsection{TUIs and Multimodal Interfaces for Safety-Critical Applications}
Cohen \etal{} proposed the use of TUIs and Multimodal Interfaces(MMUI) for safety-critical applications \cite{cohen2004tangible}. He presents and example with making strategic military planning of a battlefield. By combining special pens with cameras, CPUs and communication units, the lines drawn on a physical map would easily translate to a digital one. These tools made it easier to collaborate on making strategies and sharing them between officers. Cohen argues that the combination of TUIs and MMUIs may make a suitable improvement for traditionally paper-heavy work.

\subsection{Wii Fit Plus}
\label{sec:wiifitplus}
Nintendo Wii has gamified the way people train at home with Wii Fit and later Wii Fit Plus\fnurl{Wii Fit}{http://wiifit.com}. It gives a user the ability to choose his/her own training programme, including Yoga, Strength and Aerobics. The user can easily track his/her progress over several months. Additionally, it contributes to keeping children healthy, by having games that depend on the child's movement. For instance, if a child flaps his/her arms up and down, a bird on the screen is able to fly.    
 