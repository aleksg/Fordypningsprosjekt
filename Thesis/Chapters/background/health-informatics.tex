\section{State of the art}
Mobile computing is evolving at a rapid pace, and finding new ways to use it in health care is a rising research problem. This section covers the state of the art of some of the areas in which mobile technology is being used with a combination of either gamification or tangible interfaces.   


\subsection{Get Up and Move (GUM)}
Penados et. al.\cite{penadosget} created GUM an interactive toy to measure and stimulate physical activity. GUM is a small creature that needs to be taken care of by a child. The child's objective is to make his/her GUM healthier and happier by moving with it, feeding it and playing with it. GUM is healthy and happy when it has been through a minimal amount of daily physical activity, and since GUM can't move by itself, the child needs to do it. As GUM grows healthier, lighted stars will appear in its ears, until it reaches a maximum healthy state. To increase the number of stars, the child needs to progressively increase and later maintain its physical activity level. Penados et. al. argues that GUM had a positive effect on reducing sedentary behaviour and motivate physical activity with young children. The findings presented by Penados et. al. \cite{penadosget} gives us reason to believe that \buddy{} will show positive results when tested on children with asthma. 


\subsection{Sisom}
Sisom\fnurl{Sisom}{http://www.communicaretools.org/sisom/} is a software created to increase the communication level between physicians and children. It is an interactive game, where the user follows an avatar through different ``worlds'' of health care subjects. For instance, the avatar takes a boat to a hospital. Here the users can look around in the room and express how they feel when they are giving a blood sample. The results showed that when children played the game before a consultation with his/her physician, children were better prepared and the communcication had a better quality, and children participated more during the consultation \cite{sisom-research}.


\subsection{Meassuring Blood Preassure}
iHealth\fnurl{iHealth}{http://www.ihealthlabs.com/wireless-blood-pressure-monitor-feature\_32.htm} , Withings\fnurl{Withings}{http://www.withings.com/en/bloodpressuremonitor/features} and other companies has created blood preassure monitors which are synchronized towards mobile applications. They allow easy monitoring over periods of time and make it possible to share meassurements to both friends, family and doctors.


\subsection{Controlling Your Diabetes}
Cellnovo\fnurl{Cellnovo}{http://www.cellnovo.com/} has created a system that helps controlling diabetes. It consists of a handheld device for meassuring blood sugar level, a pump that controls the flow of insulin and a web interface that allows one to access the information. The interface helps users to check for trends and patterns in their blood sugar level, which motivates users to continue applying the correct treatment. It also allows users to send information to physicians, which helps them make decisions regarding how patients are managed.
      


\subsection{Quit Smoking}
There are lots of mobile applications that are helping people to quit smoking. For instance, \emph{The Norwegian Heart and Lung Patient Organization} has developed an app called ``'R\o ykeslutt''\fnurl{R\o ykeslutt}{https://play.google.com/store/apps/details?id=no.lhl.roykeslutt}. The application shows what the body is going through after a specific amount of days, which is a huge motivational factor, considering what we have read in the reviews of the application. Additionally, they show how much money a user has saved at a particular time, which can be seen as ``gamifying'' the element of money in order to motivate users.  


\subsection{Wii Fit Plus}
Nintendo Wii has gamified the way people train at home with Wii Fit and later Wii Fit Plus\fnurl{Wii Fit}{http://wiifit.com}. It gives a user the ability to choose their own training programme, including Yoga, Strength and Aerobics. Users can easily track their progress over several months. Additionally, it allows children to stay healthy, by having games that depend on their movement. For instance, if a child flaps their arms up and down, they fly a bird on the screen.    
 