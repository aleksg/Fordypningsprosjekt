\chapter{Background}
\label{chp:background}

This chapter will give a brief introduction to the history behind the BLOPP project (Section \ref{sec:bloppproject}). Section \ref{sec:about-asthma} will give an overview of asthma and how it affects people. Section \ref{sec:cappgappkapp} will go into details of the applications that were developed by Aaberg, Aarseth, Dale, Gisvold and Svalestuen during the Autumn 2012.     
Section \ref{sec:existing-research} will give an introduction to some of the current research that has been performed on mobile technology in combination with children and health.   


\section{BLOPP Project}
\label{sec:bloppproject}
The BLOPP project aimed to explore how design and technology can motivate children with respiratory diseases to take prescribed medication and to promote positive interactions between children and caregivers, thereby increasing adherence to medical treatment. They have previously worked with [FYLL INN].


\section{About Asthma}
\label{sec:about-asthma}
Asthma is a disease that affects the lungs. Asthma causes wheezing, breathlessness, chest tightness and coughing. It is a chronic disease, but asthma attacks will only occur when something is bothering the lungs. It may be hard to tell if someone has asthma, especially in children under the age five. 


An asthma attack may include coughing, chest tightness, wheezing and trouble breathing. The attack happens in the body's lungs, the airways tighten, letting less oxygen pass through.
According to the Norwegian Minestry of Health and Care Services, acute asthma attacks was the most common reason for hospitalization of children during 2008 \cite{NationalStrategy}. About 20 per cent of children are suffering from the disease, which causes economical consequences for the society \footnote{Costs of having parents at home instead of working, hospitalization costs, medicine costs, etc.}. However, asthma can be controlled and asthma attacks avoided by taking medicine at regular intervals. Some of the medicines are taken as a preventive measure to avoid asthma attacks from occuring. These medicines are Seretide and Flutide. Ventoline is taken before exercise or when an asthma attack occurs, in order to stop/shorten the length of the attack. 


A treatment may be done in different ways, mostly differentiated by the use of a breathing chamber or not. Before use the asthma medicine must be shaken, in order to stir the particles. If a breathing chamber is used, the medicine protection cap is removed and the medicine is mounted on the breathing chamber. The chamber is pressed towards the user's face, covering nose and mouth. The medicine is then pressed, to release the particles into the breathing chamber, and the user breaths calmly for ten seconds. 
The breathing chamber is most often used by small children, while others may use a disk-formed medicine (see Figure and Figure respectively).

\begin{figure}
	\begin{minipage}[b]{0.4\linewidth}
		\centering
			\includegraphics[width=0.20\paperwidth]{Pictures/asthmamask.png}
		\caption{Main menu of partent partition}
		\label{fig:parent_main_menu}
	\end{minipage}
	\begin{minipage}[b]{0.4\linewidth}
		\centering
			\includegraphics[width=0.20\paperwidth]{Pictures/ventolin.jpg}
		\caption{Starting a treatment}
		\label{fig:capp_start_treatment}
	\end{minipage}
\end{figure}

People suffering from asthma are often given an asthma control plan, which tells them how often they should take their medication and what to do if an attack occurs. These plans are often parted into three separate health zones, corresponding with how the user feels. In order to make these health zones understandable, a traffic light system is often used (see Appendix \ref{chp:traffic-light}). A green light tells what the user should do when all is normal. A yellow light indicates what to do when the user is feeling a bit ill, there may be a lot of pollen in the air or otherwise poor air quality or that the user is recovering from a cold. A red light indicates what to do when the user is feeling ill, or there is an extreme amount of pollen or extremely poor air quality.  


\section{CAPP, KAPP and GAPP}
\label{sec:cappgappkapp}
In the autumn of 2012 Aaberg, Aarseth, Dale, Gisvold and Svalestuen were engaged by the BLOPP Project group through the course ``TDT4290 - Customer Driven Project'' at NTNU\fnurl{Course Description of TDT4290 - Customer Driven Project}{http://www.idi.ntnu.no/emner/tdt4290/}. During the period of August 2012 to December 2012 they developed a prototype of a mobile information system consisting of two Android applications and a TUI. One application was developed for parents of a child (GAPP) and one application were developed for children (CAPP). Additionally, they created a Karotz Application (KAPP) targeted at children. In this section, we elaborate on these applications, while a full report of their work is available at\cite{CustomerDriven}. 

Their prototype is the foundation for our work in this project. 


\subsection{CAPP}
\label{sec:description-capp}
CAPP is an Android application targeted at children\footnote{All applications have norwegian as their main language}. It's main purpose is to guide children through the medication process. Figure \ref{fig:capp-main-menu} shows the main page of CAPP.  
As the target group for the application is children below the age of 8, it is reasonable to assume that not all of them are able to read, this application consists mainly of pictures and animations.


In CAPP, it is possible to start a medication in one of two ways. A parent can either set alarms in GAPP (See Section \ref{sec:description-gapp}) for preventive medicines, or a child can access the medication process directly by pressing the Karotz showed in Figure \ref{fig:capp-main-menu}, which is the way to start a by-need-treatment. 


One of the objectives towards CAPP was to introduce a gamification experience to the medication process. Accordingly, the child gets a golden star in his/her treasure chest once the child is done. However, these stars are not useful for anything else but showing them off.
  

By clicking the treasure chest, the child is able to see how many stars he/she has aquired. A screenshot showing the inside of the treasure chest is included in Figure \ref{fig:capp_stars} 


The last part of this application is an Information-section, where children has a quick reference as to how to take a medicine. A part of the functionality that has not been implemented is voice over for these instructions. Thus, a parent should be close by in order to read the information contained in this functionality.     
\ref{fig:instructions-1}-\ref{fig:instructions-7} shows the information-part of this application.

%CAPP MAIN MENU


%CAPP STARS & START TREATMENT
\begin{figure}[H]
	\begin{minipage}[b]{0.3\linewidth}
		\centering
			\includegraphics[width=0.20\paperwidth]{Pictures/app-screenshots/capp_start_treatment.png}
		\caption{Starting a treatment}
		\label{fig:capp_start_treatment}
	\end{minipage}
	\begin{minipage}[b]{0.3\linewidth}
		\centering
			\includegraphics[width=0.20\paperwidth]{Pictures/app-screenshots/capp_stars.png}
		\caption{Inside the treasure chest}
		\label{fig:capp_stars}
	\end{minipage}
	\begin{minipage}[b]{0.3\linewidth}	
		\centering
			\includegraphics[width=0.20\paperwidth]{Pictures/app-screenshots/capp_main_menu.png}
		\caption{CAPP main menu}
		\label{fig:capp-main-menu}
	\end{minipage} 
\end{figure}

%INSTRUKSJONER
\begin{figure}[H]
	\begin{minipage}[b]{0.3\linewidth}
		\centering
		\includegraphics[width=0.20\paperwidth]{Pictures/app-screenshots/instructions-1.png}
		\caption{Instructions 1}
		\label{fig:instructions-1}
	\end{minipage}
	\begin{minipage}[b]{0.3\linewidth}
		\centering
		\includegraphics[width=0.20\paperwidth]{Pictures/app-screenshots/instructions-2.png}
		\caption{Instructions 2}
		\label{fig:instructions-2}
	\end{minipage}
	\begin{minipage}[b]{0.3\linewidth}
		\centering
		\includegraphics[width=0.20\paperwidth]{Pictures/app-screenshots/instructions-3.png}
		\caption{Instructions 3}
		\label{fig:instructions-3}
	\end{minipage}
	
	\begin{minipage}[b]{0.3\linewidth}
		\centering
		\includegraphics[width=0.20\paperwidth]{Pictures/app-screenshots/instructions-4.png}
		\caption{Instructions 4}
		\label{fig:instructions-4}
	\end{minipage}
	\begin{minipage}[b]{0.3\linewidth}
		\centering
		\includegraphics[width=0.20\paperwidth]{Pictures/app-screenshots/instructions-5.png}
		\caption{Instructions 5}
		\label{fig:instructions-5}
	\end{minipage}
	\begin{minipage}[b]{0.3\linewidth}
		\centering
		\includegraphics[width=0.20\paperwidth]{Pictures/app-screenshots/instructions-6.png}
		\caption{Instructions 6}
		\label{fig:instructions-6}
	\end{minipage}
	
	\begin{minipage}[b]{0.3\linewidth}
		\centering
		\includegraphics[width=0.20\paperwidth]{Pictures/app-screenshots/instructions-7.png}
		\caption{Instructions 7}
		\label{fig:instructions-7}
	\end{minipage}
\end{figure}


\subsection{KAPP}
\label{sec:description-kapp}
KAPP is the TUI-application targeted at children. The application runs on a Karotz\fnurl{Karotz}{www.karotz.com}, which is a small robot bunny (see Figure \ref{fig:karotz}). The purpose of the KAPP is similar to CAPP, namely to remind children when it is time to take their asthma medicine and give instructions during treatment. In order to interact with the Karotz, children may use either a Nanoz (a small bunny with an integrated RFID) or by pressing a button on the top of the Karotz' head. It is not possible to do a by-need treatment with a Karotz as a companion. 

A basic breakdown of the CAPP and KAPP manuscript is included in Appendix \ref{chp:anuscript}. 


\begin{figure}[H]
	\begin{minipage}[b]{0.4\linewidth}
		\centering
			\includegraphics[width=0.20\paperwidth]{Pictures/karotz.jpg}
		\caption{Karotz. \emph{Image source: http://karotz.com}}
		\label{fig:karotz}
	\end{minipage}
	\hspace{3cm}
	\begin{minipage}[b]{0.4\linewidth}
	\centering
		\includegraphics[width=0.20\paperwidth]{Pictures/app-screenshots/gapp_main_menu.png}
		\caption{GAPP main menu}
		\label{fig:gapp-main-menu1}
	\end{minipage}
\end{figure}


\subsection{GAPP}
\label{sec:description-gapp}
GAPP is an Android application targeted at the guardians or parents of the children. 
Some parents have problems with remembering how often their children have taken their medication the last couple of days, when they should take them and how their children's disease has evolved over the a period of time. Thus, GAPP's main puropose is to make parents more aware of their child's disease.   


Figure \ref{fig:gapp-main-menu1} shows a screenshot of the main menu of GAPP. The main functionality is separated into 
\emph{Medical Plan}, \emph{Register Treatment}, \emph{Medicine Log}, \emph{Medical Information} and \emph{Manual}. 

\paragraph{Medical Plan}
\emph{Medical Plan} gives parents the option to set up reminders at particular times. It is divided according to the Traffic-Light system (See Appendix \ref{chp:traffic-light}). A child has three separate plans, such that an alarm that is set on the \emph{Healthy}-plan is not automatically set on the \emph{Sick}-plan.   

\paragraph{Register Treatment}
The \emph{Register Treatment}-option gives parents the possibility to register a treatment that is taken in case the child for some reason did not go through the process in CAPP or KAPP. This way, children will be rewarded with stars accordingly. Figure \ref{fig:gapp-register-treatment} shows a screen shot of this process.  

\paragraph{Medical Information}
\emph{Medical Information} gives general information about different medicines, what they do and what they are used for. The three medicines that are currently in the system is Flutide, Seretide and Ventoline. Figures \ref{fig:information-1} and \ref{fig:information-2} shows screenshots from this functionality.

\paragraph{Medicine Log}
\emph{Medicine Log} shows how many times a child has taken his/her medicine the last couple of days. Figure \ref{fig:medicine-log} shows a screen shot of this functionality. A red circle marks the current day. A child's health state is displayed by the Green/Yellow/Red bar at the top of each day. In the bottom left corner, it is possible to show how much medicine was taken on a given day.
In the bottom right corner, Aaberg et. al. intended to show the pollen distribution for a given day. However, the pollen distribution data is only available during spring and summer, and thus Aaberg et. al created an artificial pollen distribution for demonstration purposes. 

\paragraph{Manual}
The \emph{Manual} is to help ``newcomers'' to medicate children. For instance, if a relative is watching children with asthma, she could use the application as a reference on how to do the process. At the time being, the manual shows Figures \ref{fig:instructions-1}-\ref{fig:instructions-7}. 
        
\begin{figure}[H]
	\begin{minipage}[b]{0.3\linewidth}
		\centering
		\includegraphics[width=0.20\paperwidth]{Pictures/app-screenshots/register_treatment_old.png}
		\caption{Register treatment}
		\label{fig:gapp-register-treatment}
	\end{minipage}
	\begin{minipage}[b]{0.3\linewidth}
		\centering
		\includegraphics[width=0.20\paperwidth]{Pictures/app-screenshots/gapp_view_plans.png}
		\caption{View plans}
		\label{fig:gapp-view-plans}
	\end{minipage}
	\begin{minipage}[b]{0.3\linewidth}
		\centering
		\includegraphics[width=0.20\paperwidth]{Pictures/app-screenshots/information-1.png}
		\caption{Information 1}
		\label{fig:information-1}
	\end{minipage}
	\begin{minipage}[b]{0.4\linewidth}
		\centering
		\includegraphics[width=0.20\paperwidth]{Pictures/app-screenshots/information-2.png}
		\caption{Information 2}
		\label{fig:information-2}
	\end{minipage}
	\hspace{3cm}
	\begin{minipage}[b]{0.4\linewidth}
		\centering
		\includegraphics[width=0.20\paperwidth]{Pictures/app-screenshots/logg.png}
		\caption{Medicine log}
		\label{fig:medicine-log}
	\end{minipage}
	
\end{figure}

\subsection{Known areas for improvement}
\label{sec:improvements}
As Aaberg, Aarseth, Dale, Gisvold and Svalestuen finished their work, they commented on several areas of potential improvement for CAPP, GAPP and KAPP. This document is reprinted in its entirety in Appendix \ref{app:furtherWork} (after permission from Aaberg, Aarseth, Dale, Gisvold and Svalestuen). The main topics for improvement were
\begin{itemize}
\item{Reward System}
\item{Distraction sequence for children}
\item{Web application}
\end{itemize}


These comments are used as a basis when we decide what to improve in this project. 

\section{Existing Research}
\label{sec:existing-research}

This section will give a foundation on some of the reserach performed on using technology in combination with deceases and children. 


\subsection{Monitoring your own decease with mobile technology}
There exists some research on self-management of monitoring your asthma condition. A lot of this research works used SMS (Short Messaging System) technology. In 2009, Andh\o j and M\o ldrup et. al.\cite{anhoj2004feasibility} did a feasability study to check how users would react to a SMS-reminder study. Their methodology were to send SMS a couple of times a day, and have the users respond to their peak flow and answer yes/no questions. Users could then access a web page to see different statistics on peak flows, how they've felt the last couple of days, etc.

They concluded that SMS is a feasible solution for collecting asthma diary data, mainly because the SMS technology was a big part of the participant's everyday life. Although SMS is a great technology to be used for this purpose, few children in our target group are able to use this technology, for obvious reasons. According to \emph{Senter for IKT i utdanningen}(Center for ICT in education), about 40\% of norwegian children below the age of 3 years old have used a tablet, and 6 out of 10 children below the age of 6 have used a touch screen device \cite{nrkchilduse}. Thus the technological background should be somewhat familiar for our target group. 

\subsection{Children and mobile devices}
In 2013, babies.co.uk posted results on a poll they had posted on how many toddlers are using smartphones or tablets each day\cite{babiesusageoftablets}. Over 1000 participants responded,  and according to the survey, 14\% of the responders allowed children to use smartphones or tablets more than 4 hours a day. Considering the normal awake time of a child between 9 and 12 months old is approximately 10 hours, they spend a considerable amount of their day on the smartphone.        


\subsection{Children and gestures}

Abdul Aziz et. al. \cite{aziz2013children} performed a study on which gestures children are able to comprehend when playing with an iPad. She tested 33 children's abililty to do gestures on a variety of applications suited for children. The children were in the range of 2-12 years old, 3 children per age. The study showed the following restrictions:

\begin{itemize}
  \item 2 year old children have difficulties with pinching, and are unable to drag and drop, spread and rotation of the device, and are not able to focus on the application. 
  \item 3 year old children have difficulties to drag and drop until they are told to do so, in addition to having problems with pinch and spread. 
  \item 4 year old children have difficulties to drag and drop. 
\end{itemize}
Children at age 5 and above are able to do all the normal gestures at a tablet. As CAPP is currently only available for mobile devices, this is reason for some discussion. The main part to notice is pinching and drag and drop. 

An iPad is fairly large relative to the size of these children's hands. There is reason to believe that gestures may be more difficult on smaller screens due to the screen size, however, we were unable to find research supporting this claim. In order to make [INSERT APPNAME] as child friendly as possible, it only uses ``swiping'' gestures and button presses for navigation.



\section{State of the art}
Mobile computing is evolving at a rapid pace, and finding new ways to use it in health care is a rising research problem. This section covers the state of the art of some of the areas in which mobile technology is being used with a combination of either gamification or tangible interfaces.   


\subsection{Get Up and Move (GUM)}
Penados et. al.\cite{penadosget} created GUM an interactive toy to measure and stimulate physical activity. GUM is a small creature that needs to be taken care of by a child. The child's objective is to make his/her GUM healthier and happier by moving with it, feeding it and playing with it. GUM is healthy and happy when it has been through a minimal amount of daily physical activity, and since GUM can't move by itself, the child needs to do it. As GUM grows healthier, lighted stars will appear in its ears, until it reaches a maximum healthy state. To increase the number of stars, the child needs to progressively increase and later maintain its physical activity level. Penados et. al. argues that GUM had a positive effect on reducing sedentary behaviour and motivate physical activity with young children. The findings presented by Penados et. al. \cite{penadosget} gives us reason to believe that \buddy{} will show positive results when tested on children with asthma. 


\subsection{Sisom}
Sisom\fnurl{Sisom}{http://www.communicaretools.org/sisom/} is a software created to increase the communication level between physicians and children. It is an interactive game, where the user follows an avatar through different ``worlds'' of health care subjects. For instance, the avatar takes a boat to a hospital. Here the users can look around in the room and express how they feel when they are giving a blood sample. The results showed that when children played the game before a consultation with his/her physician, children were better prepared and the communcication had a better quality, and children participated more during the consultation \cite{sisom-research}.


\subsection{Meassuring Blood Preassure}
iHealth\fnurl{iHealth}{http://www.ihealthlabs.com/wireless-blood-pressure-monitor-feature\_32.htm} , Withings\fnurl{Withings}{http://www.withings.com/en/bloodpressuremonitor/features} and other companies has created blood preassure monitors which are synchronized towards mobile applications. They allow easy monitoring over periods of time and make it possible to share meassurements to both friends, family and doctors.


\subsection{Controlling Your Diabetes}
Cellnovo\fnurl{Cellnovo}{http://www.cellnovo.com/} has created a system that helps controlling diabetes. It consists of a handheld device for meassuring blood sugar level, a pump that controls the flow of insulin and a web interface that allows one to access the information. The interface helps users to check for trends and patterns in their blood sugar level, which motivates users to continue applying the correct treatment. It also allows users to send information to physicians, which helps them make decisions regarding how patients are managed.
      


\subsection{Quit Smoking}
There are lots of mobile applications that are helping people to quit smoking. For instance, \emph{The Norwegian Heart and Lung Patient Organization} has developed an app called ``'R\o ykeslutt''\fnurl{R\o ykeslutt}{https://play.google.com/store/apps/details?id=no.lhl.roykeslutt}. The application shows what the body is going through after a specific amount of days, which is a huge motivational factor, considering what we have read in the reviews of the application. Additionally, they show how much money a user has saved at a particular time, which can be seen as ``gamifying'' the element of money in order to motivate users.  


\subsection{Wii Fit Plus}
Nintendo Wii has gamified the way people train at home with Wii Fit and later Wii Fit Plus\fnurl{Wii Fit}{http://wiifit.com}. It gives a user the ability to choose their own training programme, including Yoga, Strength and Aerobics. Users can easily track their progress over several months. Additionally, it allows children to stay healthy, by having games that depend on their movement. For instance, if a child flaps their arms up and down, they fly a bird on the screen.    
 
