\section{Existing products}
\label{sec:exisiting-products}

On the two largest application stores, Google Play and iOS AppStore, we have found a couple of applications similar to the one we have in mind. Among the ones we have looked into, is Huff and Puff \fnurl{Google Play : Huff And Puff}{https://play.google.com/store/apps/details?id=com.healthnutsmedia.huffandpuffsd.free}, Asthma Logger
\fnurl{Google Play : Asthma Logger}{https://play.google.com/store/apps/details?id=org.androworks.asthmalog}, Kids Beating Asthma \fnurl{Google Play : Kids Beating Asthma}{https://play.google.com/store/apps/details?id=es.medianet.hcsc01} and Asthma Monitor \fnurl{Google Play : Asthma Monitor}{https://play.google.com/store/apps/details?id=ch.imvs.unibas.asthma}. Common for all these applications is that they have one specific aim. For instance, Huff and Puff wants to teach children in general about asthma. Asthma Logger logs treatments, and Kids Beating Asthma have some game elements, but the games are not available for playing during treatment. 

\begin{sidewaystable}
	\label{tab:existing-product-table}
	\begin{tabular}{ | p{4.0cm} | p{5.5cm} | p{5.5cm} | p{4cm}|}
	\hline
	\textbf{Application} & \textbf{Positive} & \textbf{Negative} & \textbf{Target Audience} \\ \hline
	
   Huff And Puff 
	& 
	\begin{itemize}
	  \item Relevant quizzes from introduction to more experienced users
	  \item Can play sounds if children cannot read
	  \item Has asthma-specific word games, puzzles, etc.  
	\end{itemize}
	&
	\begin{itemize}
	  \item Poor navigation models
	  \item Quiz is too generic, for instance asks what doctors call this and that.
	  \item The games are not exactly what we look for, as they cannot be played while undergoing a treatment  
	\end{itemize}
	&
	Children
	\\ \hline
	Asthma Logger
	& 
	\begin{itemize}
	  \item Possibility to send journal on email specified by user. May forward the journal to doctor.  
	  \item Very intuitive application
	  \item Shows doses taken the last couple of days
	\end{itemize}
	& 
	\begin{itemize}
	  \item Only has one generic medicine (does not state which medicine, for instance Ventoline) or dosage (?) 
	\end{itemize}
	& 
	Adults
	\\ \hline
	Kids Beating Asthma
	& 
	\begin{itemize}
	  \item Informative and simple
	\end{itemize}
	&
	\begin{itemize}
	  \item Suffers from software bugs and crashes regularly
	\end{itemize}
	& Children
	\\ \hline
	Asthma Monitor
	&
	\begin{itemize}
	  \item Ability connect Peak Flow to activities
	  \item Thorough and ``advanced'' statistics
	  \item Can input symptoms like Cough, Sputum, Wheezing breath and Dyspnsea 
	  \item Can send records via email
	\end{itemize}
	&
	\begin{itemize}
	  \item Old fashioned GUI
	\end{itemize}
	& 
	Adults
	\\ \hline
	\end{tabular}
	\caption{Evaluation of existing products on the market}
\end{sidewaystable}

\subsection{Conclusion and evaluation}
\label{sec:existingconcl}

%TODO: Fjerne hele dette avsnittet?
%TODO: Skrive nytt avsnitt om noe vi fant?
%The main ideas we want to take further in our application are the email-sending system of Asthma Logger and the quiz-aspect of Huff And Puff. In general, it is a good idea to be able to send your journal on email, for instance to yourself. If we combine this with possibility to send the journal to a doctor, we have a great time saving tool. To give an example: Ole has been feeling ill for a while, and has been logging when he takes his medicine. He can then schedule an appointment with his doctor, and send his journal on email to the doctor. When he arrives to his appointment, the doctor already knows how many times he has taken his medicine the last days and can give advice based upon these facts. 

Asthma Monitor seems like a great application once you get used to it, and it is developed by researchers, which implies that they know what they're doing. However, it seems a bit too complex for the following reasons:
\begin{enumerate}
  \item If an adult who have no other experience of asthma other than through his/her child, the application contains terminology which they might not be very used to
  \item The user interface is not very appealing
  \item Forcing information from a child regarding how much they cough once a day seems rather hard 
\end{enumerate}   

As for the quiz, we have concluded that this is a great way to inform children. Namely by letting them playing around with the application and gathering knowledge on this basis. 

\subsection{Assessment of existing applications}
In 2012, Huckvale et. al. \cite{huckvale2012apps} conducted an assesment on the existing applications on both Google Play and AppStore. They assessed 103 different apps with english as the native language. Out of these applications, \emph{No apps for people with asthma combined reliable, comprehensive information about the condition with supportive tools for self­management.}(Huckvale et. al., 2012). They concluded that doctors should be careful when recommending apps for people with the purpose of self management. 




