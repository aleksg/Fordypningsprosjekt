\chapter{Further Work}
\label{chp:futurework}

AsthmAPP and AsthmaBuddy are early prototypes, and need further work and changes before finalization. This chapter presents different solutions to previously mentioned evaluation and other changes based on user feedback.

\section{Future Work on \ab{}}
\label{sec:futureworkab}
During our research, \ab{} was operated using the Wizard-of-Oz technique. A future version of \ab{} should have different methods for interaction, such as touch sensors, separate actions for replaying instructions and feedback on user interaction. \ab{} now relies on a couple of configurations that has to be made before running the application. The amount of configurations that needs to performed should be reduced to a minimum.  

There are many possibilities for functionality that may be included in \ab{}. These are listed in Chapter \ref{sec:resultstui}. This list is not exhaustive, and there is still much research that may be done in this area.


\section{Future Work on \app{}}
\label{sec:futureworkapp}
\app{} has some issues regarding usability which we have already discovered. However, thorough iterations of user testing and improvements will be necessary in order for \app{} to become a success. Low usability will frustrate the user and may bias the research, and if \app{} should be used for future research it will need these improvements. 

There are several possibilities for functionality that may be added to \app{}. Examples of this are keeping track of dosages left in a container, with automatic user warning if there are less than X amounts left. Functionality for telling how many dosages was planned and how many were taken may give a better overview for the use over time. Since the amount of dosages varies with the treatment plan, such functionality would make it easier to check how well the treatment plan was followed over time. 

The calendar/log in \app{} now only supports showing a month at a time. Viewing one week or one day at the time would be useful to get a more closer view of the use of medicine. Logging when a medicine was taken, down to the minute may also be interesting for users. 

\app{} only supports one child. There are no way to track more than one child per app, which may cause problems for parents who was two or more children suffering for asthma. In a future version of \app{} there should be support for more than one child. \ab{} has support for being used by different users, but the alarm functionality will only run for one user at the time, and will block other users for starting a treatment while it's waiting for an alarm. Future \ab{} should be able to keep track of more than one child at the time. 

\textbf{Connection between health state and rewards}
\app{} has a gamification system where rewards is tied to the health state of the user. We have received contradicting arguments as to whether this solution is positive or not. One of our interviewees offered a possible solution for how to link health states and rewards.
\emph{``...In fact, one star for completing the plan might even be a better approach than one star per treatment (though the gifts can be adjusted in ``price''). Maybe even have two types of stars? For example a silver one for taking the medicine as planned, followed by a gold star once the treatment for the day has been completed.''}
We advice further research on how to make the connection between health states and rewards. 

\textbf{Rewards and shop}
While \app{}'s gamification system is built around the use of real-world awards, this might not be suitable or even necessary for all users. Some children might motivated by the simple stars on their own. The user should have the possibility of choosing only to use the stars as rewards. When setting up the system for first use the user could be asked the question of which system they would like to use. If they chose to not use real-world awards, the menu options for rewards should be hidden in order to not confuse the children. 

\section{Testing}
\label{sec:furtherworktesting}

We did a validation test at the end of our project to discover whether or not our prototypes have any potential in the future. The test results suffered from a low sample size, and more tests on children should be performed in order to discover potential improvements and usability errors. 

As stated in Section \ref{sec:gamificationdiscussion}, the motivational effect of gamification withers over time. Because of lack of test users and resources, we have not been able to test our prototypes over a longer period of time. As such, there is an existing risk that using \buddy{} and AsthmAPP may loose its effect when used over a longer period of time. Testing which rewards, and when they should be given, should undergo further research.   

\section{Future Research}
\label{sec:futureresearch}

During our research we have discovered a few research areas that could be explored further. This section aims to cover some of those. 

\textbf{Other Treatments}

The underlying concept behind our prototypes, i.e. use of gamification and tangible user interfaces, could be explored further for other diseases than asthma. For instance, testing our approach on diabetics or childen with heart defects. We have faith in the motivational factors that gamification introduces in the treatment of sick children. 

\textbf{Gamification Elements to Treat Ashmatic Children}

We have introduced a broad specter of gamification elements we considered using in \app{} and \ab{}. We can not guarantee that the set of gamification elements used in \app{} and \ab{} are the most effective, in terms of providing motivation, information or enjoyment, but it is a start. 

Researching the different possibilities of combining these elements will become a key for the eventual success a product such as ours could have.    


\textbf{Tools for Kindergarten Personell}

During our interview with a kindergarten teacher (see Appendix \ref{sec:parent2interview}), we discovered the possibility for creating a tool to help kindergarten and school teachers with treatment of asthmatic children. This could be extended to concern other diseases as well. Guiding adults through treatments they are not used to perform could be important and potentially life saving for children, as they are not around their parents at all time at any given day.    

\section{Future Vision}
\label{sec:futurevision}

In order to tie together loose ends, we have created a scenario for the use of \app{} and \buddy{}.

John is a 5 year old child who has recently been diagnosed with asthma. He is the oldest child in his family, and his parents do not have any prior knowledge of asthma. After consulting with John's pediatrician, his family has acquired \buddy{} and \app{} to help them make the transition easier. 

Johns parents wake him up at 7:00 AM a cold winter morning. They get dressed and start to make their way to the kitchen. On the way, they stop by AsthmaBuddy, who greets them with the morning status regarding asthma. \buddy{} informs the parents that the air quality is poor outside, due to heavy traffic and the cold air. He also mentions that there is currently no pollen in the air today. \buddy{} asks whether or not he should add an additional treatment to the plan for today, in order to comply with these conditions. His parents shakes \buddy{}'s hand in order to indicate that this is wanted. John eats his breakfast, and goes back to his room. 

While John is getting ready for kindergarten, an alarm is fired on \buddy{}, indicating that John is due to take his medicine before leaving. John calls for his mother, and together they are guided safely through the treatment. John is now ready for another cold day, and leaves for kindergarten, where they have planned an activity day outside. John wants to take \ab{} with him to kindergarten, since the new version of \ab{} is portable. John's parents does not think this is a good idea, since it may be lost or broken if all the children want to play with \ab{}, and thus \ab{} must stay at home.

At home, \buddy{} senses that there is a high amount of dust in John's bedroom. \buddy{} contacts the family's Roomba, who starts dusting and mopping the floors. \buddy{} turns on his red light and sends a push notification to \app{}, indicating to the parents that the floor has been cleaned. 

At noon, John gets an asthma attack after playing some serious rounds of tag. The kindergarten teacher, Lucy, is the closest grown-up around and runs over to help John. She's new on the job, and has little experience in how to handle the situation. Luckily, she has a kindergarten version of \app{}, where she can press the emergency button in order to receive guidance on how to help John. With the help of AsthmaBro, \buddy{}'s digital brother, they are guided safely through the treatment. John's parents receives a call about the incident, in order to let them know that everything is fine. They register the medicine on \app{}, in order to ensure that John gets his stars.

During the day, John's parents meet at the pediatrician's office for a follow-up meeting. They display \app{}'s log for the doctor. The doctor recommends to follow the yellow treatment plan, given the recent high pollution and cold air. He also tells them that if John keeps getting asthma attacks, he should change to the red treatment plan.

Before picking John up at the kindergarten, John's parents have added a reward, which allows John to choose today's dinner. The reward is received when John has gained a total of two stars, which he currently has accomplished. On his way home, John cashes in this reward; deciding to order pizza for the family. When they arrive at home, his parents starts dusting John's room. During dusting, John's father turns his attention to \buddy{}. He is uncertain about the number of dosages left in the inhaler, as he does not know how often he needs to purchase a new inhaler. He holds the medicine towards \buddy{}'s belly, who informs that there are a large number of dosages left. \buddy{} tells him that he will schedule an alarm when there are 15 dosages left. 

Before John goes to bed, he needs to take his preventive medicine, which is scheduled for 8:00 PM. However, John has had a long and eventful day, and starts to get tired. He takes his medicine at 7:45 PM, and \buddy{} removes the alarm scheduled at 8:00 PM. After taking his treatment, \buddy{} starts reading a chapter from a book about how the dragon Seath once conquered the magical land of Ooo, after gaining control of his asthma. 