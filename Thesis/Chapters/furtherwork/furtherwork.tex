\chapter{Further Work}
\label{chp:futurework}

AsthmAPP and AsthmaBuddy are early prototypes, and need further work and changes before finalization. This chapter presents different solutions to previously mentioned evaluation and other changes based on user feedback.

As stated in Section \ref{sec:gamificationdiscussion}, the motivational effect of gamification withers over time. We have not been able to test our prototypes over a longer period of time, and there is an existing risk that \buddy{} and AsthmAPP will become less interesting over time. 


[Oversettbarhet til andre medikamenter?]

\section{Future Work on \ab{}}
\label{sec:futureworkab}
During our research, \ab{} was operated using the Wizard-of-Oz technique. A future version of \ab{} should have different methods for interaction, such as touch sensors, separate actions for replaying instructions and feedback on user interaction. 

There are many possibilities for functionality that may be included in \ab{}. These are listed in Chapter \ref{sec:resultstui}. This list is not exhaustive, and there is still much research that may be done in this field of expertise. 

\section{Future Work on \app{}}
\label{sec:futureworkapp}
\app{} has some issues regarding usability which we have already discovered. However, thorough iterations of user testing and improvements will be necessary in order for \app{} to become a success. Low usability will frustrate the user and may bias the research, and if \app{} should be used for future research it will need these improvements. 

There are several possibilities for functionality that may be added to \app{}. Examples of this are keeping track of dosages left in a container, with automatic user warning if there are less than X amounts left. Functionality for telling how many dosages was planned and how many were taken may give a better overview for the use over time. Since the amount of dosages varies with the treatment plan, such functionality would make it easier to check how well the treatment plan was followed over time. 

The calendar/log in \app{} now only supports showing a month at a time. Viewing one week or one day at the time would be useful to get a more closer view of the use of medicine. Logging when a medicine was taken, down to the minute may also be interesting for users. 

\app{} only supports one child. There are no way to track more than one child per app, which may cause problems for parents who was two or more children suffering for asthma. In a future version of \app{} there should be support for more than one child. \ab{} has support for being used by different users, but the alarm functionality will only run for one user at the time, and will block other users for starting a treatment while it's waiting for an alarm. Future \ab{} should be able to keep track of more than one child at the time. 


\section{Testing}

\section{Future Research}


\subsection{Gamification to Treat Other Diseases}

\section{Future vision}
\label{sec:futurevision}

In order to tie together loose ends, we have created a scenario for the use of \app{} and \buddy{}.

John is a 5 year old kid who have recently been diagnosed with asthma. He's the oldest child in his family, and his parents do not have any prior knowledge about asthma. After consulting with his pediatrician, his family has acquired \buddy{} and \app{} to help them make the transition easier. 

Johns parents wake him up at 7:00 AM a cold winter morning. They get clothed and starts to make their way to the kitchen. On the way, they stop by AsthmaBuddy, who greets them with the morning status regarding asthma. \buddy{} informs the parents that the air quality is poor outside, due to heavy traffic and the cold air. He also notifies that there are currently no pollen in the air today. \buddy{} asks whether or not he should add an additional treatment to the plan for today, in order to comply with these conditions. His parents shakes \buddy{}'s hand in order to indicate that this is wanted. John eats his breakfast, and goes back to his room. 

While John is getting ready for kindergarten, an alarm is fired on \buddy{}, indicating that John is due to take his medicine before leaving. John calls for his mother, and together they are guided safely through the treatment. John is now ready for another cold day, and leaves for kindergarten, where they have planned an activity day outside. 

At home, \buddy{} senses that there is a high amount of dust on the room. \buddy{} contacts the family's Roomba, which starts dusting and mopping the floors. \buddy{} also turns on his red light, indicating to the parents that the floor has been cleaned. \buddy{} also sends a push notification to \app{}, telling the parents about the cleaning, and telling them to wipe John's room. 

At noon, John gets an asthma attack after playing some serious rounds of tag. The kindergarten teacher, Lucy, is the closest grownup around and runs over to help John. She's new on the job, and has little knowledge on how to handle the situation. Luckily, she has a kindergarten version of \app{}, where she can press the emergency button in order to receive guidance on how to help John. With the help of AsthmaBro, \buddy{}'s digital brother, they are guided safely through the treatment. John's parents receives a call about the incident, in order to let them know that everything is fine. They register the medicine on \app{}, in order to ensure that John gets his stars.

During the day, John's parents meet at the pediatrician's office for a follow-up meeting. They display \app{}'s log for the doctor. The doctor recommends to follow the yellow treatment plan, given the recent high pollution and cold air. He also says that they should switch to the yellow plan, if John keeps getting asthma attacks.

Before picking John up at the kindergarten, John's parents have added a reward, which allows John to choose today's dinner. The reward is received when John has gained a total of two stars, which he currently has accomplished. On his way home, John cashes in this reward; deciding to order pizza for the family. When they arrive at home, his parents starts dusting John's room. During dusting, John's father turns his attention to \buddy{}. He is uncertain about the number of dosages left in the inhaler, as he does not know how often he needs to purchase a new inhaler. He holds the medicine towards \buddy{}'s belly, who informs that there are a large number of dosages left. \buddy{} tells him that he will schedule an alarm when there is 15 dosages left. 

Before John goes to bed, he needs to take his preventive medicine, which is scheduled for 8:00 PM. However, John has had a long and eventful day, and starts to get tired. He takes his medicine at 7:45 PM, and \buddy{} removes the alarm scheduled at 8:00 PM. After taking his treatment, \buddy{} starts reading a chapter from a book about how a dragon, Seath, once conquered the magical land of Ooo, after gaining control of his asthma. 