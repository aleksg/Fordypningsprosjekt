\chapter{Introduction}
\label{chp:introduction}

This chapter will give an introduction to our thesis. It will state the purpose, motivation, research questions and the research method for the study. 

\section{Purpose}
\label{sec:purpose}
%TODO: Should change this to something more appropriate
The goal of this project is to explore the use of mobile technology and tangible user interfaces in the treatment of asthmatic children. The project is based on a system made by Aaberg, Aarseth, Dale, Gisvold and Svalestuen in 2012\cite{CustomerDriven}. During their project Aaberg et. al. made an Android application and a Karotz-program in order to motivate, inform and reward children suffering from asthma. Their main focus was to develop the applications, rather than do research.
We intend to improve their versions of CAPP and GAPP (see Chapter \ref{chp:background}) and in addition create a Tangible User Interface from scratch. The new and improved version of CAPP and GAPP will be combined to one application, called \emph{AsthmAPP}. The tangible user interface will be called \emph{AsthmaBuddy}. The planned functionality for the tangible user interface is to play sounds, blink with lights and sense interaction through RFID. 

The evaluation of the project will be done through co-design sessions and usability tests of different versions of the system. 
 

\section{Motivation}
\label{sec:motivation}
%TODO: Should change this to give more ``meat to the bone''
%Better now?
\subsection{Asthma Among Children}
According to NAAF, 20\% of the Norwegian population has or has had asthma by the age of 10, and 8\% of the adult population suffers from asthma\cite{NAAF}. Many of the children find it unpleasant to use their medicine as they often do not understand why the medicine must be taken. Children suffering from asthma may have to attend numerous appointments with an asthma specialist. This requires time and effort from the parents, and some may have to take time off work. In this project we will solely focus on treatment done by the use of an inhaler (with or without the mask) and disk formed medication. We chose to exclude the use of the nebulizer, since the nebulizer treatments differs so much from the use of inhalers.

We hope to motivate children suffering from asthma, since following a treatment plan can lead to a more controlled form of asthma, where attacks occur less often\cite{ginasthma}. 
Research done by \r{A}sheim showed that children suffering from HRS-virus\fnurl{Center for Disease Control : HSRV}{http://www.cdc.gov/rsv/} were easily distracted and motivated to finish treatments when shown a non-interactive flash-video during the treatment\cite{Asheim610877}. We have seen other projects using gamification elements have positive results, such as Get Up and Move made by Penados \etal{}\cite{penadosget}\footnote{A short summary of the relevance of Penados' research is given in Section \ref{sec:gum}}. 
By using mobile technology, tangible user interfaces and may make the children more aware of their disease and thus make them better understand why they must take their medicine on a daily basis. 
By using gamification elements we hope to motivate the children to use the system more often, and by making a dynamic and user-centered reward system, we hope to make a system that will be interesting for children for a longer period of time.  


\section{Research Questions}
\label{sec:researchquestions}
The main goal for this study is to figure out in which ways technology can help children taking their medication. The objective has been composed into the following research questions: 

\paragraph{RQ1:}
\textbf{How can gamification be used for motivating children to take their asthma medicine?}


\paragraph{RQ2:}
\textbf{How can tangible user interfaces be used to help children with asthma?}


\section{Research Method}
\label{sec:researchmethod}
We will expand on the Android application Aaberg \etal{} with focus on gamification elements. We will also create a prototype for a tangible interface, called \buddy{}. We will build these prototypes on information gained in the following steps:

\begin{enumerate}
  \item We will start out by interviewing parents and domain experts to find out more about children's medication habits.
  \item The prototypes will then be tested by potential users and parents of asthmatic children. 
  \item Based on feedback from the user tests, we will design new prototypes.
  \item Step 2 and 3 will be repeated once. 
  \item The final prototype will be user tested by children with asthma. 
\end{enumerate}
  
%Hva vi ønsker å få ut av de ulike stegene
The rationale for doing Step 1 is to create a foundation to build upon on the later steps. 
Collecting information about how children take their medicine under normal conditions gives us insight into the main problems in relation to taking medication. Having insight to the problems our users meet is a useful way to make sure the final product will be suited to meet the users needs. We also want to gather ideas for how the TUI should be designed in terms of functionality and look.

The mobile application was already built on the start of this project. The main focus for the mobile application will be to test the gamification elements of the mobile application. 

We will build the TUI using a Raspberry Pi, with additional components in order to play sounds, display lights and different I/O. A detailed description of the TUI is given in Chapter \ref{chp:our-solution}. 


Based on the feedback gathered in our first round of user testing and focus groups, we will make a second version of the prototype. This prototype will then again be brought back to our test persons and domain experts in order to receive feedback on the improved design.

The final prototype will undergo user testing by children in order to determine if the system is understood by the children and if the children enjoys using the system. We will not necessarily measure the usability in terms of a SUS-score\cite{brooke1996sus}, but rather by using the methods presented in Section \ref{sec:usabilitytestchildren}, since we believe these methods to be more appropriate.

\subsection{Theoretical Motivation Behind Interviews}

When we interviewed domain experts and parents, we did it in a \emph{semi structured} manner. By conducting semi structured interviews, we let our interviewees put more weight upon their own views and opinions around the subject, and we could explore more in depth on interesting answers, which could lead to findings we could not have obtained otherwise. It also allowed us to use some pre-determined questions, which provided uniformity among our subjects.

The purpose of conducting these interviews, was to ensure that the end product was not limited by our own imagination. We wanted feedback on the work we had done so far, in addition to exploring new functionality and elements we should keep in mind. 

The problems of using this approach, is that it limits creativity of our subjects. For instance, people might be shocked over the research we're doing, and thus might not be in the correct mindset to give as good as possible feedback as far as creativity goes. We tried to minimize this threat by giving our subjects a brief understanding of our work beforehand, such that they could be able to explore their own opinions before meeting with us.

We collected data by having one interviewer, while the other was taking notes. We also recorded the interviews, in order to be able to play back these. The people we collected data from was

[TODO: Should we mention people by name?]
\begin{itemize}
  \item Nanna S\o nnichsen Kayed. PhD in psychology, with experiences in reward systems while raising children.  
  \item A couple of parents with children suffering from asthma. 
  \item X domain experts, including people working for the Blopp project and NAAF (The Norwegian Asthma and Allergy Association).  
  \item X asthma nurses
\end{itemize} 

\section{Thesis Outline}
Chapter \ref{chp:background} provides the reader with backround information around Asthma, and some of the projects that have previously been developed by Blopp, in addition to introduce the reader on the state of the art. 
Chapter \ref{chp:gamification} will give the reader an introduction to gamification, with discussion around some of the principles that are being used. 
Chapter \ref{chp:tangibleinterfaces} dicusses the origins and use of tangible interfaces.
Chapter \ref{chp:usability} gives an introduction to the principles behind usability. Though it is not a primary part of our thesis, we consider it important to keep in mind, especially when designing for children.
Chapter \ref{chp:description} provides a product description of AsthmAPP, our prototype for gamifying children's experience with a smartphone, while Chapter \ref{chp:our-solution} provides a description of AsthmaBuddy, our tangible interface.
Chapter \ref{chp:results} provides the results we have found during our research.
Chapter \ref{chp:evaluation} provides an evaluation of our process, what could have been done better, etc. 
Chapter \ref{chp:masterconclusion} provides the final conclusions for our thesis, with discussions around the results given in Chapter \ref{chp:results}.          
