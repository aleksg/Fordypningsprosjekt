\chapter{Introduction}
\label{chp:introduction}

This chapter will give an introduction to our thesis. It will state the purpose, motivation, research questions and the research method for the study. 
%introduction to our thesis. 

\section{Purpose}
\label{sec:purpose}
The goal of this project is to evaluate the use of mobile technologies in the treatment of asthmatic children. The project is based on a system made by Aaberg, Aarseth, Dale, Gisvold and Svalestuen in 2012 \cite{CustomerDriven}. We intend to improve their versions of CAPP and GAPP (see Chapter \ref{chp:background}), in addition to create a Tangible User Interface from scratch. 

The evaluation of the project will be done through usability testing and diary studies of different versions of the system, augmenting the functionality from cycle to cycle. 

%We plan on using gamification as a motivational factor for the children. The goal of using gamification is to research whether gamification is a suitable concept, rather than finding ``the best'' use of gamification in this setting. Bland inn med TUI lengre nede.


\section{Motivation}
\label{sec:motivation}

\subsection{Asthma among children}
According to NAAF, 20\% of the Norwegian population has or has had asthma at the age of 10, and 8\% of the adult population suffers from asthma \cite{NAAF}. Many of the children find it unpleasant to use their medicine as they often do not understand why the medicine must be taken. Research done by \r{A}sheim showed that children suffering from HRS-virus\fnurl{Center for Disease Control : HSRV}{http://www.cdc.gov/rsv/} were easily distracted and motivated to finish treatments when shown a non-interactive flash-video during the treatment \cite{Asheim610877}. We aim to research whether use of mobile technology may make the children more aware of their asthma and thus make them better understand why they must take their medicine on a daily basis. 


\subsection{Ways asthma affect the guardians}
In an already hectic everyday life, remembering to give their children medication may be cumbersome. Often the children do not enjoy taking their medicine, and the children may start an argument not wanting to finish their treatment. This may result in guardians applying the medication incorrectly, applying the wrong treatment, or even forgetting to give the medicine, which in turn has negative effect on the overall treatment. [INSERT REFERENCE]. We aim to find out if the use of mobile technology will make guardians more aware of their childrens' disease, thus increasing the effect of the treatment.  


\section{Research Questions}
\label{sec:researchquestions}
The main goal for this study is to figure out ways technology can help children taking their medication. During the prephase, we will build upon the work of Aaberg, et. al.  \cite{CustomerDriven}, trying to identify problems with the system as it is today. The improved system will then undergo user testing over a longer period of time, in order to test different concepts.


The objective has been composed into the following research questions: 

\paragraph{RQ1:}
\textbf{Is gamification a feasible solution for motivating children to take their asthma medicine?}


\paragraph{RQ2:}
\textbf{How will the presence of a Tangible User Interface affect children's medicational habits?}


This evaluation should be done through user testing and feedback from potential users of the applications. Hopefully, thorough testing will give information on how the interaction between children and the system is, and whether these systems helped during the treatment process.

\section{Research Method}
\label{sec:researchmethod}

\subsection{RQ1}
\label{sec:RQ2-methodology}
%TODO: Find a suitable number
We want to test the system on NUMBER children. In order to test this using a systematic approach, we plan to take the following steps, which will be explained in further detail below. 
\begin{enumerate}
  \item Give guardians a diary, where they take note of how things are working on a regular basis. Expected duration: 1 week.
  \item Give guardians a mobile application which instructs children during their medication. This will be a simplified version of our final application. Expected duration: 1 week. 
  \item Give guardians a mobile application which has gamification elements to it, in addition to the instructions in Step 2. Expected duration 1 week.   
  \item Give the test persons a custom built TUI. Expected duration: 1 week. 
\end{enumerate}
During the testing phase, we want the guardians to fill out a diary, in addition to undergo an interview at the end of test period. Answers we want to find from the diary study, is whether the presence of technology had an effect on their medical habits. 

The rationale for doing Step 1 is to create a foundation to build upon on the later steps. Having children taking their medicine under ``normal'' conditions gives us a set of control data which we may use for comparison in order to discover trends and how the application and TUI affected the children and guardians. 

The rationale for doing Step 2 is to see if it is actually enough to have a minor avatar system who tells the child what to do, and when to do it. This might give some ideas for further research from BLOPP. 


The rationale for doing Step 3 is to answer whether gamification have a motivational effect on children, i.e. motivating children to take their medicine on a continuous basis. 

The rationale for doing Step 4 is to see if a relational artifact can give children proper motivation for taking their medicine.

A key point during all these steps is that guardians are actually on place, observing the children's behavior. We will not dive further into children's behavioural patterns or psychology, other than whether or not the system had an effect on their children opinions with regards to taking their medicine.       
 
\subsection{RQ2}
\label{sec: RQ3-methodology}

Through the conducted studies we will evaluate and compare the usage and opinions the guardians and children had regarding use of our system. Through the data gathering in the diaries we will be able to directly compare how the children and guardians reacted to the arrival and use of a TUI. We will also conduct interviews to get feedback on how the use of the TUI affected the children. The interviews will mainly be conducted on the guardians, since small children may not give reliable data.