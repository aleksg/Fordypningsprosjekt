\chapter{Introduction}
\label{chp:introduction}

This chapter will give an introduction to our thesis. It will describe the purpose, motivation, research questions and the research method for the study. 

\section{Purpose}
\label{sec:purpose}
%TODO: Should change this to something more appropriate
The goal of this project was to explore the use of mobile technology and tangible user interfaces in the treatment of asthmatic children. The project was based on a system made by Aaberg, Aarseth, Dale, Gisvold and Svalestuen in 2012\cite{CustomerDriven}. During their project Aaberg et. al. made an Android application and a Karotz-program in order to motivate, instruct, inform and reward children who suffer from asthma. Their main focus was on the development process of their developed prototypes, rather than researching the different solutions to help children with asthma.
We intended to improve their versions of CAPP and GAPP (see Chapter \ref{chp:background}), in addition to create a tangible user interface from scratch. The new and improved version of CAPP and GAPP was combined to one application, called \emph{AsthmAPP}. The tangible user interface was named \emph{AsthmaBuddy}. 

In this project we focused solely on treatment by the use of an inhaler (with or without the mask) and disk formed medication. We chose to exclude the use of the nebulizer, since the nebulizer treatments differs too much from the use of inhalers.
The evaluation of our project was done through interviews and usability tests of different versions of the system. 
 

\section{Motivation}
\label{sec:motivation}

\subsection{Asthma Among Children}
According to NAAF, 20\% of the Norwegian population has or has had asthma by the age of 10, and 8\% of the adult population suffers from asthma\cite{NAAFStat}. Many children find taking their medicine unpleasant, and they often do not understand what the medicine is good for. Children suffering from asthma may have to have several appointments with  asthma specialists. This requires time and effort from the parents, and many parents have to take time off work. 

We hoped to motivate children suffering from asthma to follow their treatment plan, since following their plan may lead to a more controlled form of asthma, where attacks occur less frequently\cite{ginasthma}. 
Research done by Asheim showed that children suffering from HRS-virus\fnurl{Center for Disease Control : HSRV}{http://www.cdc.gov/rsv/} were easily distracted and motivated to finish treatments when shown a non-interactive flash-video during the treatment\cite{asheim2012konsept}. We have seen other projects where using gamification elements has provided positive results, such as Get Up and Move made by Penados \etal{}\cite{penadosget}\footnote{A short summary of the relevance of Penados' research is given in Section \ref{sec:gum}}. 
By using mobile technology and tangible user interfaces we wanted to make the children more aware of their disease and thus make them better understand why they need to take their medicine on a daily basis. 
By using gamification elements we hoped to motivate the children to use the system regularly, and by making a dynamic and user-centered reward system, we hoped to make a system that the children will find interesting for a longer period of time.  



\section{Research Questions}
\label{sec:researchquestions}
The main goal for this study was to discover in ways technology can help and encourage children to take their medication. The objective was composed into following research questions: 

\paragraph{RQ1:}
\textbf{How can gamification be used to motivate children to take their asthma medicine?}


\paragraph{RQ2:}
\textbf{How can tangible user interfaces be used to help children with asthma?}


\section{Research Method}
\label{sec:researchmethod}

The first step of our research was to do find recent trends in our research area. We did a literature study that gave us some principles to start with. 
 
Based on the literature study, we started developing prototypes. The Android application Aaberg \etal{} built was modified, introducing gamification elements to \app{}. Starting from scratch, we created a tangible user interface by developing an application running on a \rpi{}, and placing the \rpi{} inside a teddy bear. This tangible user interface was given the name \ab{}.       

After the first prototypes were developed, we interviewed domain experts, by using a ``semi-structured'' interviewing-approach. The goal for the interviews was to receive feedback on the prototypes and it's reward system and to explore new functionality we could incorporate in \ab{} and \app{}. By interviewing domain experts we were able to receive feedback that we could not have obtained by continuous user testing on children, as children may be nervous or excited and have difficulties in explaining their thoughts and opinions.  

Our interview subjects consisted of parents of two different children with asthma, a person with a PhD in psychology, two nurses with asthma as their field of expertise, a PhD candidate in industrial design, a senior advisor at NAAF and a industrial designer previously involved in the BLOPP project. 

After finishing the interviews we did a round of user tests of \ab{} on fellow students, to test the ``Token+Constraint'' approach (see Section \ref{sec:tokenandconstraint}) of \ab{}. As the students were considered inexperienced with regards to asthma, these user tests were also able to uncover whether or not \ab{} was able to explain the treatment process in an easily understandable manner. Based on the feedback gathered from the domain experts and the user tests, we did an additional development cycle of \ab{} and \app{}.

After the last development cycle, we tested the prototypes on N children \iref{}. The purpose of the user tests was to hear children's opinion on the prototypes, discover difficulties when they interacted with \ab{}, and to explore whether or not the gamification system worked in the short run.  



\subsection{Theoretical Motivation Behind Interviews}
\label{sec:motivationbehindinterviews}

When we interviewed domain experts and parents, we did it in a \emph{semi structured} manner. By conducting semi structured interviews, we let our interviewees put more weight upon their own views and opinions around the subject, and we could explore interesting answers in more depth, which could lead to findings we could not have obtained otherwise. It also allowed us to use some pre-determined questions, which provided uniformity among our subjects.

The purpose of conducting these interviews was to ensure that the end-product was not limited by our own imagination. We wanted feedback on the work we had done so far, in addition to exploring new functionality and elements we should keep in mind. 

The problems of using this approach was that it limits creativity of our subjects. For instance, people may be intimidated by the research we were doing, and thus may not have been in the correct mindset to give as good as possible feedback as far as creativity goes. We tried to minimize this risk by giving our subjects a brief summary of our work beforehand, so that they would be able to make up their own opinions about the project before meeting for the interview.


\section{Thesis Outline}
Chapter \ref{chp:background} provides the reader with background information around Asthma, and some of the projects that have previously been developed by BLOPP, and introduces the latest developments in the use of mobile technology and tangible user interfaces for medical purposes. 
Chapter \ref{chp:gamification} will give the reader an introduction to gamification, with discussion around some of the principles that are being used. 
Chapter \ref{chp:tangibleinterfaces} dicusses the origins and use of tangible interfaces.
Chapter \ref{chp:usability} gives an introduction to the principles behind usability. Though it is not a primary part of our thesis, we consider it important to keep in mind, especially when designing applications for children.
Chapter \ref{chp:description} provides a product description of AsthmAPP, our prototype for gamifying children's experience with a smartphone, while Chapter \ref{chp:our-solution} provides a description of AsthmaBuddy, our tangible interface.
Chapter \ref{chp:results} provides the results we have discovered during our research.
Chapter \ref{chp:masterconclusion} provides the final conclusions of our thesis, with discussions around the results given in Chapter \ref{chp:results}.          
