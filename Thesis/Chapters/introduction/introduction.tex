\chapter{Introduction}
\label{chp:introduction}

This chapter will give an introduction to our thesis. It will state the purpose, motivation, research questions and the research method for the study. 

\section{Purpose}
\label{sec:purpose}
%TODO: Should change this to something more appropriate
The goal of this project is to explore the use of mobile technology and tangible user interfaces in the treatment of asthmatic children. The project is based on a system made by Aaberg, Aarseth, Dale, Gisvold and Svalestuen in 2012\cite{CustomerDriven}. During their project Aaberg et. al. made an Android application and a Karotz-program in order to motivate, inform and reward children suffering from asthma. Their main focus was to develop the applications, rather than do research.
We intend to improve their versions of CAPP and GAPP (see Chapter \ref{chp:background}) and in addition create a Tangible User Interface from scratch. The new and improved version of CAPP and GAPP will be combined to one application, called \emph{AsthmAPP}. The tangible user interface will be called \emph{AsthmaBuddy}. The planned functionality for the tangible user interface is to play sounds, blink with lights and sense interaction through RFID. 

The evaluation of the project will be done through co-design sessions and usability tests of different versions of the system. 
 

\section{Motivation}
\label{sec:motivation}
%TODO: Should change this to give more ``meat to the bone''
%Better now?
\subsection{Asthma Among Children}
According to NAAF, 20\% of the Norwegian population has or has had asthma by the age of 10, and 8\% of the adult population suffers from asthma\cite{NAAFStat}. Many of the children find it unpleasant to use their medicine as they often do not understand why the medicine must be taken. Children suffering from asthma may have to attend numerous appointments with an asthma specialist. This requires time and effort from the parents, and some may have to take time off work. In this project we will solely focus on treatment done by the use of an inhaler (with or without the mask) and disk formed medication. We chose to exclude the use of the nebulizer, since the nebulizer treatments differs so much from the use of inhalers.

We hope to motivate children suffering from asthma to follow their treatment plan, as following their plan can lead to a more controlled form of asthma, where attacks occur less frequently\cite{ginasthma}. 
Research done by Asheim showed that children suffering from HRS-virus\fnurl{Center for Disease Control : HSRV}{http://www.cdc.gov/rsv/} were easily distracted and motivated to finish treatments when shown a non-interactive flash-video during the treatment\cite{asheim2012konsept}. We have seen other projects using gamification elements have positive results, such as Get Up and Move made by Penados \etal{}\cite{penadosget}\footnote{A short summary of the relevance of Penados' research is given in Section \ref{sec:gum}}. 
By using mobile technology, tangible user interfaces and may make the children more aware of their disease and thus make them better understand why they must take their medicine on a daily basis. 
By using gamification elements we hope to motivate the children to use the system more often, and by making a dynamic and user-centered reward system, we hope to make a system that will be interesting for children for a longer period of time.  



\section{Research Questions}
\label{sec:researchquestions}
The main goal for this study is to figure out in which ways technology can help children taking their medication. The objective has been composed into the following research questions: 

\paragraph{RQ1:}
\textbf{How can gamification be used for motivating children to take their asthma medicine?}


\paragraph{RQ2:}
\textbf{How can tangible user interfaces be used to help children with asthma?}


\section{Research Method}
\label{sec:researchmethod}

% Literature study
% Developed prototypes based on our litterature study
% Interviewed semistructured domain-experts
% Improved prototypes
% Validation tests

The first step of our research was to do find recent trends in our research area. We did a literature study that gave us some principles to work from, especially regarding tangible interfaces and gamification. 
 
Based on the literature study, we started developing prototypes. The Android application Aaberg \etal{} built was modified, giving \app{} more focus on introducing gamification elements to the treatment of asthmatic children. Starting from scratch, we created a Tangible User Interface by developing a \rpi{} application, and putting the \rpi{} inside a teddy bear. This Tangible User Interface was given the name \ab{}.       

After the first prototypes was developed, we interviewed domain experts, by using a ``semi-structured'' interviewing-approach. The goal for the interviews was to receive feedback on the prototypes and it's reward system, in addition to explore new functionality we could incorporate in \ab{} and \app{}. By interviewing domain experts, we were able to receive feedback that we could not have gotten by continuous user testing on children, as they might be nervous or have difficulties explaining their thoughts and opinions.  

Our interview subjects consisted of two parents of children with asthma, one PhD in psychology, two nurses with expertise on asthma, one PhD candidate in industrial design, one senior advisor at NAAF and one industrial designer previously involved in the BLOPP project. 

After finishing the interviews, we did a round of user tests of \ab{} on fellow students, to test the Token+Constraint approach of \ab{}. Additionally, we tested \ab{}'s ability to explain the treatment process to inexperienced users. Based on the feedback gathered from the domain experts and the user tests, we did one more development cycle of \ab{} and \app{}.

After the last development cycle, we tested the prototypes on N children \iref{}. The purpose of the user tests was to gather children's thoughts around the prototypes, discover difficulties when they interacted with \ab{}, and to explore whether or not the gamification system implemented worked in the short run.  



\subsection{Theoretical Motivation Behind Interviews}

When we interviewed domain experts and parents, we did it in a \emph{semi structured} manner. By conducting semi structured interviews, we let our interviewees put more weight upon their own views and opinions around the subject, and we could explore more in depth on interesting answers, which could lead to findings we could not have obtained otherwise. It also allowed us to use some pre-determined questions, which provided uniformity among our subjects.

The purpose of conducting these interviews, was to ensure that the end product was not limited by our own imagination. We wanted feedback on the work we had done so far, in addition to exploring new functionality and elements we should keep in mind. 

The problems of using this approach, is that it limits creativity of our subjects. For instance, people might be shocked over the research we're doing, and thus might not be in the correct mindset to give as good as possible feedback as far as creativity goes. We tried to minimize this threat by giving our subjects a brief understanding of our work beforehand, such that they could be able to explore their own opinions before meeting with us.


\section{Thesis Outline}
Chapter \ref{chp:background} provides the reader with backround information around Asthma, and some of the projects that have previously been developed by Blopp, in addition to introduce the reader on the state of the art. 
Chapter \ref{chp:gamification} will give the reader an introduction to gamification, with discussion around some of the principles that are being used. 
Chapter \ref{chp:tangibleinterfaces} dicusses the origins and use of tangible interfaces.
Chapter \ref{chp:usability} gives an introduction to the principles behind usability. Though it is not a primary part of our thesis, we consider it important to keep in mind, especially when designing for children.
Chapter \ref{chp:description} provides a product description of AsthmAPP, our prototype for gamifying children's experience with a smartphone, while Chapter \ref{chp:our-solution} provides a description of AsthmaBuddy, our tangible interface.
Chapter \ref{chp:results} provides the results we have found during our research.
Chapter \ref{chp:masterconclusion} provides the final conclusions for our thesis, with discussions around the results given in Chapter \ref{chp:results}.          
