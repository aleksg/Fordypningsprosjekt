\chapter{Introduction}
\label{chp:introduction}

This chapter will give an introduction to our thesis. It will state the purpose, motivation, research questions and the research method for the study. 

\section{Purpose}
\label{sec:purpose}

%TODO: Should change this to something more appropriate
The goal of this project is to evaluate the use of mobile applications and a tangible user interface in the treatment of asthmatic children. The project is based on a system made by Aaberg, Aarseth, Dale, Gisvold and Svalestuen in 2012\cite{CustomerDriven}. We intend to improve their versions of CAPP and GAPP (see Chapter \ref{chp:background}) and in addition create a Tangible User Interface from scratch. The new and improved version of CAPP and GAPP will be combined to one application, called \emph{AsthmAPP}. The tangible user interface will be called \emph{AsthmaBuddy}.

The evaluation of the project will be done through co-design sessions and usability tests of different versions of the system. 
 

\section{Motivation}
\label{sec:motivation}
%TODO: Should change this to give more ``meat to the bone''
\subsection{Asthma among children}
According to NAAF, 20\% of the Norwegian population has or has had asthma by the age of 10, and 8\% of the adult population suffers from asthma\cite{NAAF}. Many of the children find it unpleasant to use their medicine as they often do not understand why the medicine must be taken. Research done by \r{A}sheim showed that children suffering from HRS-virus\fnurl{Center for Disease Control : HSRV}{http://www.cdc.gov/rsv/} were easily distracted and motivated to finish treatments when shown a non-interactive flash-video during the treatment\cite{Asheim610877}. We aim to research whether use of mobile technology may make the children more aware of their disease and thus make them better understand why they must take their medicine on a daily basis. 


\section{Research Questions}
\label{sec:researchquestions}
The main goal for this study is to figure out in which ways technology can help children taking their medication. The objective has been composed into the following research questions: 

\paragraph{RQ1:}
\textbf{How can gamification be used for motivating children to take their asthma medicine?}


\paragraph{RQ2:}
\textbf{How can Tangible Interfaces be used for motivating children to take their asthma medicine?}

\section{Research Method}
\label{sec:researchmethod}

We want to test the system on 5-7 asthmatic children at the age of 3-7 years. In order to test this by using a systematic approach, we plan to take the following steps, which will be explained in further detail below. 

\begin{enumerate}
  \item We will start out by interviewing parents and domain experts to find out more about children's medication habits. Additionally, we will arrange focus groups for brainstorming purposes.
  \item Next we will develop a TUI and a mobile application prototype based on the knowledge we have gathered.  
  \item The prototypes will then be tested by potential users and parents of asthmatic children. 
  \item Based on feedback from the user tests, we will design new prototypes.
  \item Step 3 is repeated once.
\end{enumerate}
 

The rationale for doing Step 1 is to create a foundation to build upon on the later steps. 
Collecting information about how children take their medicine under normal conditions gives us insight into the main problems in relation to taking medication. We also want to gather ideas for how the TUI should be designed in terms of functionality and look. 

The mobile application was already built on the start of this project. The main focus for the mobile application will be to test the gamification elements and general usability testing of the mobile application. The initial plan for the TUI is to build a system using a Raspberry Pi and different elements to play sounds and emit lights. A detailed description of the TUI is given in Chapter \ref{chp:our-solution}.

Our first prototype of AsthmaBuddy will be a ``dumb'' version, where we can test out different forms of interaction without actually programming the TUI for them. A list of the different forms of interaction is given in Chapter \ref{chp:our-solution}. 

The prototypes will then be user tested through usability testing and co-dessign sessions. Both users with knowledge of the project and users without prior knowledge of the system should participate in the system in order to not bias results. 

Based on the feedback we will make prototypes iteratively, in a total of three iterations.