\chapter{Introduction}
\label{chp:introduction}

This chapter will give an introduction to our thesis. It will state the purpose, motivation, research questions and the research method for the study. 
%introduction to our thesis. 

\section{Purpose}
\label{sec:purpose}
The goal of this project is to evaluate the use of mobile technologies in the treatment of asthmatic children. The project is based on a system made by Aaberg, Aarseth, Dale, Gisvold and Svalestuen in 2012 \cite{CustomerDriven}. We intend to improve their versions of CAPP and GAPP (see Chapter \ref{chp:background}), in addition to create a Tangible User Interface from scratch. 

The evaluation of the project will be done through usability testing and diary studies of different versions of the system, augmenting the functionality from cycle to cycle. 



\section{Motivation}
\label{sec:motivation}

\subsection{Asthma among children}
According to NAAF, 20\% of the Norwegian population has or has had asthma at the age of 10, and 8\% of the adult population suffers from asthma \cite{NAAF}. Many of the children find it unpleasant to use their medicine as they often do not understand why the medicine must be taken. Research done by \r{A}sheim showed that children suffering from HRS-virus\fnurl{Center for Disease Control : HSRV}{http://www.cdc.gov/rsv/} were easily distracted and motivated to finish treatments when shown a non-interactive flash-video during the treatment \cite{Asheim610877}. We aim to research whether use of mobile technology may make the children more aware of their asthma and thus make them better understand why they must take their medicine on a daily basis. 


\subsection{Ways asthma affect the parents}
In an already hectic everyday life, remembering to give their children medication may be cumbersome. Often the children do not enjoy taking their medicine, and the children may start an argument not wanting to finish their treatment. This may result in parents applying the medication incorrectly, applying the wrong treatment, or even forgetting to give the medicine, which in turn have a negative effect on the overall treatment.  



\section{Research Questions}
\label{sec:researchquestions}
The main goal for this study is to figure out ways technology can help children taking their medication. The objective has been composed into the following research questions: 

\paragraph{RQ1:}
\textbf{How can gamification be used for motivating children to take their asthma medicine?}


\paragraph{RQ2:}
\textbf{How will the presence of a Tangible User Interface affect children's medication habits?}

%Eventuell rationale her

\section{Research Method}
\label{sec:researchmethod}


\subsection{RQ1}
\label{sec:RQ2-methodology}
We want to test the system on 5-7 asthmatic children at the age of 3-7 year old. In order to test this using a systematic approach, we plan to take the following steps, which will be explained in further detail below. 

\begin{enumerate}
  \item We will start out by interviewing parents and domain experts to find out more about children's medication habits. 
  \item Give parents a mobile application which instructs children during their medication. This will be a simplified version of our final application.  
  \item Give parents a mobile application which has gamification elements to it, in addition to the instructions in Step 2. 
  \item Give the test persons a custom built TUI. 
\end{enumerate}
 

The rationale for doing Step 1 is to create a foundation to build upon on the later steps. 
Collecting information about how children take their medicine under normal conditions give us a set of control data which we may use for comparison with the developed technology.

The rationale for doing Step 2 is to see if it is actually enough to have a minor avatar system where the avatar tells the child what to do, and when to do it. This might give some ideas for further research from BLOPP. 


The rationale for doing Step 3 is to answer whether gamification have a motivational effect on children, i.e. motivating children to take their medicine on a continuous basis. 

The rationale for doing Step 4 is to see if a relational artifact can give children proper motivation for taking their medicine.

During Step 2-4, we want parents to fill out a diary. We will prepare a set of control questions which should be answered in this diary. The answers to these control questions might give us additional insight into how the process went.  


An example of such a question is \emph{``Did the child have problems using the TUI?''}. The answer to this question might improve over time, once the child has gotten used to the applications or the TUI.  


Through the conducted studies we will evaluate and compare the usage and opinions the parents and children had regarding use of our system. Through the data gathering in the diaries we will be able to directly compare how the children and parents reacted to the arrival and use of a TUI. We will also conduct interviews to get feedback on how the use of the TUI affected the children. The interviews will mainly be conducted on the parents, since small children may not give reliable data.
