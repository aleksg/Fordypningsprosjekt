\chapter{Introduction}
\label{chp:introduction}

This chapter will give an introduction to the study. It will state the purpose, motivation, research questions and the research method for this study. 

\section{Purpose}
\label{sec:purpose}
The goal of this study is to evaluate the CAPP, GAPP and Karotz Applications created by Aaberg, Aarseth, Dale, Gisvold and Svalestuen \cite{CustomerDriven}.
The evaluation will be done through usability testing carried out on all three applications. The results of these initial tests will thereafter be used to improve the applications for a newer version. 
We will also plan a thorough testing of the applications.


\section{Motivation}
\label{sec:motivation}
According to NAAF, 20\% \cite{NAAF} of the Norwegian population has or has had asthma at the age of 10, and 8\% of the adult population suffers from asthma. Many of the children find it unpleasant to use their medicine as they often do not understand why the medicine must be taken [Should have a reference]. This may result in parents applying the medication incorrectly, applying the wrong treatment, or even forgetting to give the medicine to their children. 


\section{Research Questions}
\label{sec:researchquestions}
The main goal for this study is to evaluate the CAPP, GAPP and Karotz application, and identify the usability problems in these systems. Structuring the goal into different research questions will help this study with the evaluation of the goal. The goal has been composed into these questions:

\paragraph{RQ1:}
\textbf{What are the usability problems of the current system?}


\paragraph{RQ2:}
\textbf{How will the guardian(s) react on having a system monitoring the childs use of medicines?}


\paragraph{RQ3:}
\textbf{Will the physicians benefit from having detailed logs and information sent by email?}

This evaluation should be done through user testing and feedback from future users of the applications. The testing will give information on how well the 
\ldots

\section{Research Method}
\label{sec:researchmethod}

\subsection{RQ1}
\label{sec: RQ1-methodology}
%Naatid vs fortid??
We will perform usability testing on the NSEP laboratory located at St. Olavs hospital. We will ask the participants to fill out SUS-schemes \ref{app:norsksus}, in addition to noting problems users may experience. 


\subsection{RQ2}
\label{sec: RQ2-methodology}
We will try to get a holding of more Karotz, on which we will launch the application. Then we will find participants through BLOPP's network. We will then see how parents react on having such a tangible interface at their home, and how the parents react to it. The central question in mind is whether or not they feel ``monitored''. Through this process, the parents are required to fill out a diary on a daily basis. 


\subsection{RQ3}
\label{sec: RQ3-methodology}
We will use some of the data collected in \ref{sec: RQ2-methodology} and present it to asthma physicians. We will then interview them, trying to identify what sort of information is useful, and whether or not it is feasible to send this medical information to physicians.    

