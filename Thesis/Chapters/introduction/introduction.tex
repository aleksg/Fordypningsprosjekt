\chapter{Introduction}
\label{chp:introduction}

This chapter will give an introduction to our thesis. It will describe the purpose, motivation, research questions and the research method for the study. 

\section{Purpose}
\label{sec:purpose}
%TODO: Should change this to something more appropriate
The goal of this project was to explore the use of gamification and tangible user interfaces in the treatment of asthmatic children. The project was based on a system made by Aaberg, Aarseth, Dale, Gisvold and Svalestuen in 2012\cite{CustomerDriven}. During Customer Driven Project 2012 (TDT4290) at NTNU, Aaberg et. al. made an Android application and a Karotz-program in order to motivate, instruct, inform and reward children who suffer from asthma. Their main focus was on the development process of their prototypes, rather than researching the different possibilities to help children with asthma.
We intended to improve their versions of CAPP and GAPP (see Chapter \ref{chp:background}), in addition to create a tangible user interface from scratch. The new and improved version of CAPP and GAPP was combined into one application, called \emph{AsthmAPP}. The tangible user interface was named \emph{AsthmaBuddy}.  
 

\section{Motivation}
\label{sec:motivation}

According to NAAF, 20\% of the Norwegian population has or has had asthma by the age of 10, and 8\% of the adult population suffers from asthma\cite{NAAFStat}. Many children find taking their medicine unpleasant, and they often do not understand what the medicine is good for. Children suffering from asthma may have to attend several appointments with asthma specialists. This requires time and effort from the parents\footnote{While not all child do not live with their parents, we chose to use the terms parents instead of guardian.}, and many parents have to take time off work. 

We hoped to motivate children suffering from asthma to follow their treatment plan, since following their plan may lead to a more controlled form of asthma, where attacks occur less frequently\cite{ginasthma}. 
Research done by Asheim showed that children suffering from HRS-virus\fnurl{Center for Disease Control : HSRV}{http://www.cdc.gov/rsv/} were easily distracted and motivated to finish treatments when shown a non-interactive flash-video during the treatment\cite{asheim2012konsept}. We have seen other projects where using gamification elements has provided positive results, such as Get Up and Move made by Penados \etal{}\cite{penadosget}\footnote{A short summary of the relevance of Penados' research is given in Section \ref{sec:gum}}. 

By using mobile technology and tangible user interfaces we wanted to make the children more aware of their disease and thus make them better understand why they need to take their medicine on a daily basis. 
By using gamification elements we hoped to motivate the children to use the take their medicine according to plan. By making a dynamic and user-centered reward system, we hoped to make a system that the children will find interesting for a longer period of time.  



\section{Research Questions}
\label{sec:researchquestions}
The goal of this project was to explore the use of gamification and tangible user interfaces in the treatment of asthmatic children. The objective was composed into the following research questions: 

\paragraph{RQ1:}
\textbf{How can gamification be used to motivate children to take their asthma medicine?}


\paragraph{RQ2:}
\textbf{How can tangible user interfaces be used to help children with asthma?}


\section{Research Method}
\label{sec:researchmethod}
In order to answer the research questions, we have used qualitiative research methods. 
We started off by conducting a literature study, in order to gain knowledge about the state of the art, find arguments discussing gamification and tangible user interfaces and exploring development frameworks for building tangible user interfaces. We then developed a tangible user interface prototype, using a \rpi{}, which was placed inside a stuffed toy animal. Additionally, we continued development of Aaberg \etal{}'s smartphone application prototypes; CAPP and GAPP. 

In parallell with the development process, we conducted a series of semi-structured interviews, in order to retrieve information and feedback from a wide area of perspectives. Our interview subjects consisted of two parents of children with asthma, a researcher with a PhD in psychology, two nurses with asthma within their field of expertise, a PhD candidate in industrial design, a senior advisor at NAAF and an industrial designer previously involved in the BLOPP project. 

At the end of the project, we did a validation test of our prototypes on three children and two parents, by following the usability testing approach. 

A more thorough explanation of the research methods employed can be found in Chapter \ref{chp:researchmethod}. 


\section{Thesis Scope}
\label{sec:thesis scope}
In this project we focused solely on treatment by the use of an inhaler (with or without the mask) and disk formed medication. We chose to exclude the use of the nebulizer, since the nebulizer treatments differs too much from the use of inhalers\footnote{Nebulizer treatments often lasts for 10-15 minutes, while treatments performed through an inhaler or a disk formed medication are completed after one minute, including preparations. This makes the nebulizer treatments a completely different area of research regarding the use of gamification}.
The evaluation of our project was done through semi-structured interviews and usability tests of different versions of the system.


\section{Thesis Outline}
\label{sec:thesisoutline}
Chapter \ref{chp:background} provides the reader with background information around Asthma, some of the previous projects BLOPP have been involved with, and introduces the latest developments in the use of mobile technology and tangible user interfaces for medical purposes.
Chapter \ref{chp:researchmethod} gives an overview of the research methods we have employed.  
Chapter \ref{chp:gamification} will give the reader an introduction to gamification, with discussion around some of the principles that are being used. 
Chapter \ref{chp:tangibleinterfaces} dicusses the origins, usage and development frameworks of tangible interfaces.
Chapter \ref{chp:description} provides a product description of AsthmAPP, our prototype for gamifying children's experience with a smartphone, while Chapter \ref{chp:our-solution} provides a description of AsthmaBuddy, our tangible interface.
Chapter \ref{chp:results} provides the results we have discovered during our research.
Chapter \ref{chp:masterconclusion} provides discussions and conclusions of our thesis.
Chapter \ref{chp:futurework} lays out further work that can be based on our thesis, and gives a scenario of how \ab{} and \app{} could be used in the future.            
