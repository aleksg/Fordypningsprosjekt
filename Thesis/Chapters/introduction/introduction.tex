\chapter{Introduction}
\label{chp:introduction}

This chapter will give an introduction to our thesis. It will state the purpose, motivation, research questions and the research method for the study. 

\section{Purpose}
\label{sec:purpose}

%TODO: Should change this to something more appropriate
The goal of this project is to explore the use of mobile technology and tangible user interfaces in the treatment of asthmatic children. The project is based on a system made by Aaberg, Aarseth, Dale, Gisvold and Svalestuen in 2012\cite{CustomerDriven}. During their project Aaberg et. al. made an Android application and a Karotz-program in order to motivate, inform and reward children suffering from asthma. Their main focus was to develop the applications, rather than do research.
We intend to improve their versions of CAPP and GAPP (see Chapter \ref{chp:background}) and in addition create a Tangible User Interface from scratch. The new and improved version of CAPP and GAPP will be combined to one application, called \emph{AsthmAPP}. The tangible user interface will be called \emph{AsthmaBuddy}. The planned functionality for the tangible user interface is to play sounds, blink with lights and sense interaction through RFID. 

The evaluation of the project will be done through co-design sessions and usability tests of different versions of the system. 
 

\section{Motivation}
\label{sec:motivation}
%TODO: Should change this to give more ``meat to the bone''
%Better now?
\subsection{Asthma among children}
According to NAAF, 20\% of the Norwegian population has or has had asthma by the age of 10, and 8\% of the adult population suffers from asthma\cite{NAAF}. Many of the children find it unpleasant to use their medicine as they often do not understand why the medicine must be taken. Children suffering from asthma may have to attend numerous appointments with an asthma specialist. This requires time and effort from the parents, and some may have to take time off work. 
We hope to motivate children suffering from asthma, since following a treatment plan can lead to a more controlled form of asthma, where attacks occur less often \cite{ginasthma}. 
Research done by \r{A}sheim showed that children suffering from HRS-virus\fnurl{Center for Disease Control : HSRV}{http://www.cdc.gov/rsv/} were easily distracted and motivated to finish treatments when shown a non-interactive flash-video during the treatment\cite{Asheim610877}. We have seen other projects using gamification elements have positive results, such as Get Up and Move made by Penados et. al\cite{penadosget}\footnote{A short summary of the relevance of Penados' research is given in Section \ref{sec:gum}}. 
By using mobile technology, tangible user interfaces and may make the children more aware of their disease and thus make them better understand why they must take their medicine on a daily basis. 
By using gamification elements we hope to motivate the children to use the system more often, and by making a dynamic and user-centered reward system, we hope to make a system that will be interesting for children for a longer period of time.  


\section{Research Questions}
\label{sec:researchquestions}
The main goal for this study is to figure out in which ways technology can help children taking their medication. The objective has been composed into the following research questions: 

\paragraph{RQ1:}
\textbf{How can gamification be used for motivating children to take their asthma medicine?}


\paragraph{RQ2:}
\textbf{How can Tangible Interfaces be used for motivating children to take their asthma medicine?}


\section{Research Method}

We will expand on the Android application Aaberg \etal{} with focus on gamification elements. We will also create a prototype for a tangible interface, called \buddy{}. We will build these prototypes on information gained in the following steps:

\begin{enumerate}
  \item We will start out by interviewing parents and domain experts to find out more about children's medication habits. Additionally, we will arrange focus groups for brainstorming purposes.
  \item We will then develop prototypes based on the knowledge gathered.   
  \item The prototypes will then be tested by potential users and parents of asthmatic children. 
  \item Based on feedback from the user tests, we will design new prototypes.
  \item Step 3 is repeated once.
\end{enumerate}
   

\section{Research Method}
\label{sec:researchmethod}

The following list is the plan for our research project:

\begin{enumerate}
  \item We will start out by interviewing parents and domain experts to find out more about children's medication habits. Additionally, we will arrange focus groups for brainstorming purposes.
  \item Next we will improve on our TUI and mobile application prototype based on the knowledge we have gathered.  
  \item The prototypes will then be tested by potential users and parents of asthmatic children. 
  \item Based on feedback from the user tests, we will design new prototypes.
  \item Step 3 and 4 is repeated once.
  \item The final prototype will be user tested by children with asthma.
\end{enumerate}
 
%Hva vi ønsker å få ut av de ulike stegene
The rationale for doing Step 1 is to create a foundation to build upon on the later steps. 
Collecting information about how children take their medicine under normal conditions gives us insight into the main problems in relation to taking medication. Having insight to the problems our users meet is a useful way to make sure the final product will be suited to meet the users needs. We also want to gather ideas for how the TUI should be designed in terms of functionality and look.

The mobile application was already built on the start of this project. The main focus for the mobile application will be to test the gamification elements and general usability testing of the mobile application. The initial plan for the TUI is to build a system using a Raspberry Pi and different elements to play sounds and emit lights. A detailed description of the TUI is given in Chapter \ref{chp:our-solution}. 

Our first prototype of AsthmaBuddy will be a ``dumb'' version, where we can test out different forms of interaction without actually programming the TUI for them. A list of the different forms of interaction is given in Chapter \ref{chp:our-solution}. By testing the different types of interaction, we can eliminate methods of interaction that are not well suited for our system on an early stage. This may give useful feedback on what we should focus on in the further development. 

Based on the feedback gathered in our first round of user testing and focus groups, we will make a version 2 of the prototype. This prototype will then again be brought back to our test persons and domain experts in order to receive feedback on the improved design.

The final prototype will undergo user testing by children in order to measure if the system is understood by the children and if the children enjoys using the system. We will not necessarily measure the usability in terms of a SUS-score\cite{brooke1996sus} but rather by using the methods presented in \ref{sec:usabilitytestchildren}.
