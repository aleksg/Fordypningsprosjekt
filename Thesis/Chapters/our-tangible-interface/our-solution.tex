
\chapter{Our intended solution}
\label{chp:our-solution}

This chapter will give a brief introduction into the TUI we are planning to create. 

\section{Introduction}
\label{sec:our-solution-introduction}
From our experiences with Karotz \cite{karotz}, we have found that we do not want to go any further with it. The thought behind Karotz is great, but these thoughts are, in our subjective opinion, poorly executed. If we were to test the concept at people's homes, we would need to preconfigure the units down to the smallest details. Just the fact that we would have to ask families for their wifi-credentials before testing a system, should be reasons enough.
Additionally, the Karotz API is only documented in french, which makes it considerable harder for us. Maybe the largest challenge by using Karotz is it's price. A Karotz starters kit costs \$200, not including customs to Norway. Making such a large investment on a product parents have no experience with is going to be quite difficult from a commercial standpoint.  


We have also taken a brief look into Arduino, which is an open source electronics prototyping platform \cite{arduino}. We have seen some of the projects that have been done in Ardunio, and have found that they require competence and knowledge we simply do not have, i.e. digital design and circuit boards. 


The solution we have found most exciting, both for ourselves and as a solution, is Rasperry Pi \cite{rasperrypi}. It is a cheap computer with the size of a credit card, with the original intention of teaching british school children about computer programming \cite{rasperrypi-about}. Since the release early in 2012, it has sold more than 1 million units, and is highly popular among computer enthusiasts \cite{pimillion}. 


\section{What do we want to create?}
On the highest level of abstraction, what we want to create here is a descrete artifact children can have at their rooms, which can remind them to take their medicine, and help them through the process. Our plan for the time being is basically to wrap the Rasperry Pi inside a toy, i.e. teddy bear, a doll, or some other popular, medium sized toy.

We consider the following as the minimum set of abilities for our TUI:

\begin{itemize}
  \item Ability to connect to a network, either through ethernet or Wifi, though both are preferred. 
  \item Play sounds through speakers
  \item Ability to read RFID-chips
  \item Display information about the color of the asthma medicine through LED-lights
\end{itemize} 

\subsection{A basic scenario}
In order to get a basic understanding of what we are trying to achieve here, we have included a basic scenario below. Assume that we have our Rasperry Pi inside a teddy bear.  

\begin{enumerate}
  \item (Bear): Cough, cough. Come over here kiddo 
  \item (Kid approaches)
  \item (Bear): I need my asthma medicine now. Could you please help me? The color of the medicine is displayed at my belly.
  \item (Kid finds the medicine)
  \item (When the child holds the medicine towards the mouth of the bear, the RFID chip on the medicine is read, which instructs the bear to continue). 
  \item (Bear): Thanks. Now it is your turn to take it. 
  \item (Bear): Now, find your mask and put it over your head. You need to press the button at the top of your medicine before you start breathing. 
  \item (Bear): (Counts for the amount of seconds needed)
  \item (Bear): Great job!
\end{enumerate}

The idea behind having a voice to the teddy bear is to establish a two-way communcation between children and the artifact. We want children to actually care for our product, which may be achieved through giving the artifact a sense of personality. 


\subsection{Usage of components}

\paragraph{RFID}
We will be using RFID-technology to let children interact with the TUI. Our initial vision is to attach an RFID-chip to each of the childrens' medicines, which in turn can be read by our TUI. For instance, during step 5 in the above scenario, a child can hold the medicine towards the mouth of the bear. The type of the medicine is read by the bear, who can notify the child if he/she is starting to take the wrong medicine. 

\paragraph{Speakers}
The speakers will be used to play prerecorded messages from the teddybear.

\paragraph{Connect to network}
Ideally, we want to attach a wifi shield to our Rasperry Pi, making it able to connect wirelessly. The benefits of this is that it makes the TUI portable. However, this might include having to preconfigure the system as with Karotz as mentioned in section \ref{sec:our-solution-introduction}. This makes it a problem we will have to research further before deciding on a solution for our prototype. Information we want to send through the network consists mainly of alarm notifications set by parents, i.e. when the toy should giving signals that it needs medication. 

\paragraph{LED-lights}
Through LED-lights, we intend to show the color of the medicine that is to be taken. As we cannot assume children are able to read, or to remember the name of the medicine, Aaberg et. al. have found that showing the color is a feasible solution to showing which medicine should be taken \cite{CustomerDriven}. 

\section{Programming our Rasperry Pi}
Rasperry Pi's main programming language is C. As we are not especially good at programming in C, we will use Pi4J as a bridge between Java and the Rasperry Pi's native libraries, which makes us able to manipulate IO-channels \cite{pi4j}.
  

 