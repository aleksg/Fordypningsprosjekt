\chapter{Conclusions}
\label{conclusions}

20\% of the Norwegian population has or has had asthma before the age of 10. Treating children for asthma is often a cumbersome task which may lead to arguing and sub-optimal treatment. Research has shown that TUIs and mobile applications have been useful in medical care in a number of different settings. By combining gamification with TUIs and mobile technology, we belive that children will be motivated to take their medicines and get greater awareness of their own disease. While there is much research on the effect of TUI's in treatment of patients, there is still very few systems that have seen a mass-market release. We believe that the development of AsthmaBuddy, through a codesign approach will give an indicative result as to how well TUI's can be used for treatment of asthmatic children.


Serious games has given us an idea of how to balance gamification elements properly. The many different uses and approaches of how to develop such games, has given us valuable insight and knowledge. We have learned that when developing a system targeted for children, it is important to use non-obtrusive gamification elements. Applications with have focus on money-spending and in-app-purchases have also recieved much critique for trying to tirck money out of the user. By letting parents and their children decide upon what the rewards are going to be, we give an opportunity for better motivation factors for both children and their parents. 


We have chosen to develop a TUI in addition the Android application. The rationale behind this decision is that TUIs often are percieved as more fun to operate when performing a repetetive task \ref{sec:aretuisfun}. We believe that a friendly looking TUI will be percieved in a more positive manner than a smartphone application by the children, while the parents will have more flexibility and easier keep of control with the smartphone application. Inspired by Ullmer \cite{ullmer2002tangible} we have chosen a token+constraint approach for our TUI; AsthmaBuddy. With the help of domain experts and potential end users, we are sure that the iterative design and development of AsthmaBuddy will in turn lead to a successful TUI. 

We are eager to continue our research and the development of AsthmaBuddy. 


