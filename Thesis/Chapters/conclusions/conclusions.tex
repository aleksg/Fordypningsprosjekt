\chapter{Conclusions}
\label{conclusions}

20\% of the Norwegian population has or has had asthma before the age of 10. Treating children for asthma is often a cumbersome task. Research has shown that TUIs and mobile applications have been useful in medical care in a number of different settings. By combining gamification with TUIs and mobile technology, we belive that children will be motivated to take their medicines and get greater awareness of their own disease.    


Serious games has given us an idea of how to balance gamification elements properly. When developing a system targeted for children, it is important to use non-obtrusive gamification elements. By letting parents and their children decide upon what the rewards are going to be, we give an opportunity for better motivation factors for both children and their guardians.   


We are eager to continue our research.


