\section{Existing products}
\label{exisiting-products}

On the two biggest application stores, Google Play and iOS AppStore, it exists a couple of similar applications to the one we have in mind. Among those we have looked into, is Huff and Puff, Asthma Logger, Kids Beating Asthma and Asthma Monitor. Common for all applications is that they have one specific aim. For instance, Huff and Puff wants to teach children in general about asthma. Asthma Logger logs treatments, and Kids Beating Asthma have some game elements, but none of these games are able to play during medication. 

\begin{center}
	\begin{tabular}{ | p{2.5cm} | p{5cm} | p{5cm} | p{2cm}|}
	\hline
	Application & Positive & Negative & Gamification elements \\ \hline
	
	Huff and Puff & 
	\begin{itemize}
	  \item Decent quizzes from introduction to more experienced users
	  \item Can play sounds if children cannot read
	  \item Has asthma-specific word games, puzzles, etc.  
	\end{itemize}
	&
	\begin{itemize}
	  \item Poor navigation models
	  \item Quiz is too generic, for instance asks what doctors call this and that.
	  \item The games is not exactly what we look for, as they cannot be played while undergoing a treatment  
	\end{itemize}
	&
	YES
	\\ \hline
	Asthma Logger & 
	\begin{itemize}
	  \item Possibility to send journal on email specified by user. May forward this to doctor.  
	  \item Really intuitive application
	  \item Shows dozes taken the last couple of days
	\end{itemize}
	& 
	\begin{itemize}
	  \item Only has one generic medicine (does not state which medicine, for instance Ventoline) or dosage (?) 
	\end{itemize}
	& 
	NO
	\\ \hline
	Kids Beating Asthma
	& 
	\begin{itemize}
	  \item Informative and simple
	\end{itemize}
	&
	\begin{itemize}
	  \item Many software bugs and crashes regularly
	\end{itemize}
	& NO
	\\ \hline
	\end{tabular}
\end{center}

\subsection{Conclusion and evaluation}
\label{existingconcl}

The main ideas we want to take further in our application is the email-sending system of Asthma Logger and the quiz-aspect of Huff And Puff. In general, it is a really good idea to be able to send your own journal on email, for instance to yourself. If we combine this with possibility to send this journal to the doctor, we have a great time saving tool. Let's say that Ole has been feeling bad for a while, and has been good at making journal for when he has taken his medicine. He can then schedule an appointment with his doctor, and send his journal on email to the doctor. When he arrives to his appointment, the doctor already knows how many times he has taken medicine the last days and can easier give advice based upon these facts. 
 

As for the quiz, we have concluded that this is a great way to inform children. Namely by letting them playing around with the application and gathering knowledge on this basis. 



