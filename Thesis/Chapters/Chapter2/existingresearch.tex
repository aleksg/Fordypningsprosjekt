\section{Existing Research}
\label{sec:existing-research}

\subsection{Monitoring your own decease}
There exists some research on self-management of monitoring your asthma condition. A lot of this research does however work with SMS (Short Messaging System) technology. In 2009, Andh\øj and M\øldrup et. al.\cite{anhoj2004feasibility} did a feasability study to check how users would react to a SMS-reminder study. Their methodology were to send SMS a couple of times a day, and have the users respond to their peak flow and answer yes/no questions. Users could then access a web page to see different statistics on peak flows, how they've felt the last couple of days, etc.


Whether the system actually improved the user's awareness of their decease was unanswered (TLDR?).. 

Although SMS is a great technology to be used for this purpose, few children in our target group are able to use this technology. 


\subsection{Children and gestures}

Abdul Aziz et. al. \cite{aziz2013children} made a study on what gestures children are able to comprehend when playing with an iPad. He/She tested 33 children's abililty to do gestures on a variety of applications suited for children. The children were in the range of 2-12 years old, 3 children per age. The study showed the following restrictions:

\begin{itemize}
  \item 2 year old children have difficulties with pinching, and are unable to drag-and-drop, spread and rotation of the device, and are not able to focus on the application. 
  \item 3 year old children have difficulties to drag \& drop until they are told to do so, in addition to having problems with pinch and spread. 
  \item 4 year old children have difficulties to drag and drop. 
\end{itemize}
Children at age 5 and above are able to do all the normal gestures at a tablet. As CAPP is currently only available for mobile devices, this is reason for some discussion. The main part to notice is pinching and drag and drop. Now, are these difficulties only problems regarding the tablet size, or do they also arise on mobile phones? An iPad is fairly large relative to the size of these children's hands. 