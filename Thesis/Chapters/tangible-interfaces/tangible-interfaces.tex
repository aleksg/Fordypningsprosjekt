\chapter{Tangible User Interfaces}
\label{chp:tangibleinterfaces}

This chapter will introduce the reader to Tangible User Interfaces, and elaborate on some existing research that has been done on the concept.   

\section{About Tangible User Interfaces}
\label{sec:abouttuis}

In 1997, Ishii et. al. presented an article called ``Tangible Bits: Towards Seamless Interfaces between People, Bits and Atoms''. They established the term ``Tangible User Interface'' (TUI) as a way to move beyond the dominant model of Graphical User Interfaces. 
While GUIs show information (bits) in the form of pixels mapped to a display, Ishii meant for TUIs to represent the bits in form of physical objects. The objective of TUI was explained to \emph{augment the real physical world by coupling digital information to everyday physical objects and environments}\cite{ishii1997tangible}. 

Urp\cite{underkoffler1999urp} is an example of an first-generation TUI. Urp is a workbench architects used to determine shadow patterns for models of buildings. By moving a ``clock tool'' the lighting on the workbench would move according to what time of day was choosen. Instead of interacting with the lights directly, the TUI factor of the workbench was the clock tool.
What makes this a first-generation TUI is the fairly simple and state-determined operations.

Sandscape\cite{ishii2004bringing} is an example of a second-generation TUI. Sandscape uses clay, sand, cameras, digital software and lighting to give an overview of a Geographical Information System(GIS). The users could interact with the clay, forming dunes or dig holes and the software would calculate landscape analysis based on the interaction. Sandscape is dynamical user interface since it may change to several different non-predefined states, also named a ``Continuous TUI''.


\section{Examples of Use}
\label{sec:tuiexamples}
Using TUIs instead of GUIs has been proven to work in several different settings. In this section, we will give an overview on some of the domains in which the concept has been proven to work. 

\paragraph{Learning}
Terrenghi et al. designed a cube for learning, giving children quizzes where answers had the shape of text or images\cite{terrenghi2006cube}. Children could then rotate the cube in order to get the correct answer pointing upwards, like a dice. They concluded that the TUI gave children a different set of affordances that prompted a great initial engagement\cite{terrenghi2006cube}. 

\paragraph{Collaborative Learning}
Scarlatos et. al. created a system called TICLE (Tangible Interface in Collaborative Learning), which are used to help children solve a Tangram\cite{scarlatos1999ticle}.  

\paragraph{Interactive Storytelling}
Zhou et. al. designed a cube for storytelling, using a head mounted display and a ``Magic story cube'' in order to let children explore the world while being told a story\cite{zhou2004magic}. Stanton et. al. created a ``Magic Carpet'', giving children possiblity to influence a story in the classroom\cite{stanton2001classroom}. 
 
\paragraph{Social Context}
Marble Answering Machine is an invention by Durrell Bishop, dating back to 1992\cite{crampton1995hand}. The interface allows users to drop marbles into a play-back indent on the system, which plays a recorded message.   

\section{TUIs Used in Health Care}
\label{sec:effectofrobots}
In 2003, Wada et. al. conducted a study on how the introduction of robotics affected the elderly\cite{wada2004effects}. They conducted a study at a day service center in Japan, where they placed a robotic seal, named Paro, together with the elderly. It had recently been found that animals have a positive effects on blood preassure, depression and loneliness. They placed a robotic seal in the care center, and analyzed the reactions from the elderly. 

The results showed that their mood was better after interacting with Paro over five weeks, and became worse once Paro was removed. In addition, nurses burnout rate decreased during the experiment, which implied that the subjects had easier days when Paro was there. The study showed that their quality of life improved after Paro was introduced.           

Farr \etal{} did a study on children with Autisitic Spectrum Condition (ACS), when playing with a TUI called Topobo\fnurl{Topobo}{http://www.topobo.com/}\cite{farr2010social}. The study compared the level of social interaction when playing with Topobo, compared to playing with LEGO. Their findings showed that ACS children were playing more cooperatively with the TUI than LEGO. Additionally, children with traditional development were able to play more cooperative, solitary and parallell when using a TUI, suggesting that 

\textit{``(\ldots) programmable digital technology may support more pathways to social interaction.''}

 
\section{Are Tangible Interfaces more fun?}
\label{sec:aretuisfun}
In 2008, Xie et. al. performed a study on how children reacted using different interfaces in order to solve a jigsaw puzzle\cite{xie2008tangibles}. The different interfaces were a physical interface (i.e. a standard jigsaw puzzle), a TUI and a GUI. Their findings were mainly that children enjoyed playing with the different interfaces equally. However, the children were more likely to start a puzzle over again if the interface were physical or tangible, which implies that a repeated task is more likely to be performed if they're playing with a tangible or physical interface, while it becomes boring to do the same task over and over again on a graphical user interface. It is worth mentioning that the puzzles were being solved by groups of two, and considering the GUI was a computer with one mouse, they didn't get the same sharing experience as with the other interfaces. 


\section{How to create a TUI}
Ullmer states that a tangible user interface should embody the following four properties\cite{ullmer2002tangible}:

\begin{enumerate}
	\item{Physically Embodied}
	\item{Physically Representational}
	\item{Physically Manipulable}
	\item{Spatially Reconfigurable}
\end{enumerate}

Ullmer states that these four properties describe physical artifacts as representations and controls for digital information. 

Ullmer proposes three different classes of TUI's: Interactive Surfaces, Token and Constraints, and Constructive Assemblies. These classes are partly based on varying degree of support for continuous and discrete forms of interaction\cite{ullmer2002tangible}.

\subsection{Token + Constraint Approach}
The ``Token + Constraint'' approach centers on a hierarchical relationship between tokens and constraints. Tokens may be placed within or removed from the compatible constraints. The physical shape of the tokens and constraints display whether or not the tokens are compatible or not. This approach support a combination of continuous and discrete interactions. 

\paragraph{Strenghts of token + constraint approach}
Ullmer states that interpretive constraints will help to express which of the physical tokens can take part within a given interpretive constraint, which physical configurations these physical tokens can take and the demarcation between interaction regions with different computational interpretation.

These interpretive constraints may help to simplify the human perception since humans are good at comparing shapes and forms. 
It may help human manipulation since interpretive constraints provide an increased sense of kinesthetic feedback from the manipulation of tokens. 



\subsection{Design Approach to TUI's}
A lot of material exists on the potential benefits of TUI's, but few exists on how to actually create them. Champoux proposes a mechanism to design TUI's based achieving fitness between the form and its context\cite{subramaniandesign}.
He proposes three classes of questions, which corresponds to the different development phases of TUIs:
\begin{itemize}
  \item Defining the boundaries
  \item Orienting the components
  \item Fitting the components
\end{itemize} 


By answering the questions in Table \ref{tab:tuidesign}, his mechanism will ease the development phase.   


\begin{table}[h]
	\begin{tabular}{| p{5.0cm} | p{5.0cm} | p{5.0cm} |}
	\hline
	\textbf{Defining the Boundaries} & \textbf{Orienting the Concepts} & \textbf{Fitting the Components} \\
	\hline
	\textbf{BO1}: What should the user experience? \newline
	\textbf{BO2}: What are the human task? \newline
	\textbf{BO3}: What would the artefact represent and control? \newline 
	\textbf{BO4}: What are the conventions? (Physical ergonomics vs electromechanical) \newline 
	&
	\textbf{OC5a}: What is the nature of the interaction for each sub task? (Continuous vs Discrete vs Assembly) \newline
	\textbf{OC5b}: What are the electromechanical and physical ergonomic constraints for this task? \newline
	\textbf{OC6}: Does the sub-task need any relational interaction? \newline
	&
	\textbf{FC7}: What are the relations between the objects and the actions? \newline 
	\textbf{FC8}: What is the task order when using the artefact? \\ 
	\hline
	
	\end{tabular}
	\caption{The eight questions stated by Champoux\cite{subramaniandesign}}
	\label{tab:tuidesign}
\end{table}  

We consider Question \textbf{OC5b} as irrelevant for our purposes, as we will have a stationary artefact without any electromechanical and phyical ergonomic properties, i.e. moving arms, waving ears, etc. The other questions are relevant to the development of AsthmaBuddy.

\newpage

\section{Challenges when creating TUI}
\label{sec:challenges-with-TUI}

Working with GUI's is quite easy from a usability point of view. Assuming that every potential user has used some sort of GUI-based application before, there should not be any fundemental affordance problems when creating a desktop application. What a user expects from a computer mouse is simply given beforehand. However, with TUIs, no such expectations from the user exist. Bellotti et. al.\cite{bellotti2002making} has found several research challenges with creating usable TUIs and ubiquitous systems. This section will elaborate on some of the questions we have to ask ourselves when we are creating our TUI.


Bellotti et. al. found five basic issues considering communication between a human and a system; Address, Attention, Action, Alignment and Accident.  
These issues were then asked as questions, which exposed several challenges. 
Some of the challenges they found considered ubiquitous computing, assuming there are several possible target systems at the same location. Table \ref{tab:tuichallenges} shows an appropriate subset of their findings, with our project in mind.   

\begin{table}[H]
	\begin{tabular}{| p{6.0cm} | p{7.0cm} |}
	\hline
	\textbf{Basic question} & \textbf{Exposed Challenges} \\
	\hline
	\textbf{Address:} How do I address one of many possible devices? & How do disambiguate intended target system. \newline How to not address the system. \\ 
	\hline
	\textbf{Attention:} How do I know the system is ready and attending to my actions? & How to embody appropriate feedback, so that the user can be aware of the system's attention.\newline How to direct feedback to zone of user attention. \\
	\hline
	\textbf{Action:} How do I effect a meaningful action, control its extent and possibly specify a target or targets for my action? & How to identify and select a possible object for action \\
	\hline
	\textbf{Alignment:} How do I know the system is doing (has done) the right thing? & How to make system state perceivable and persistent or query-able. \newline How to direct timely and appropriate feedback. \newline How to provide distinctive feedback on results and state. \\ 
	\hline 
	\textbf{Accident:} How do I avoid mistakes & How to control or cancel system action in progress. \newline How to disambiguate what to undo in time. \newline How to intervene when user makes obvious error \\
	\hline
	\end{tabular}
	\caption{The challenges of interacting with a TUI\cite{bellotti2002making}}
	\label{tab:tuichallenges}
\end{table}



% Some of the challenges we identified as particularly hard to encounter is summarized below. 
% 
% \begin{itemize}
%   \item How to control or cancel system action in progress.
%   \item How to make system state perceivable and persistent. 
%   \item How to embody appropriate feedback, so that the user can be aware of the system's attention. 
% \end{itemize} 


\section{Co-design}
\label{sec:codesign}
This section will describe co-design and how we plan to use a co-design approach when developing our TUI.

\subsection{What is co-design?}
Co-design is a product, service, or organization development process where design professionals empower, encourage, and guide users to develop solutions for themselves. Co-design encourages the blurring of the role between user and designer, focusing on the process by which the design objective is created \cite{sanders2008co}. By encouraging the trained designer and the end user to create ideas and solutions together, the final result will be more appropriate and acceptable to the end user \cite{albinsson2007co}. 
Albinsson et al \cite{albinsson2007co} discovered that having a co-design approach to development of a e-mail sorting system lead to the ability of easier conflict management, ability to centre innovation on the client/customer and made it easier to manage a project with an unknown outcome. While co-design has been around for over 40 years, it draws it's roots from user-centered design and participatory design. Co-design differs from user-centered design approaches in that it acknowledges that the client or beneficiary of the design may not be using the artifact itself \cite{norman1986user}.



\subsection{Using co-design to build AsthmaBuddy}
Based on the research we have read, and recommendation from domain experts, we have chosen to try a co-design approach to develop our TUI; AsthmaBuddy. We will build a low-fidelity prototype which we can show to a group of test persons and domain experts. These sessions will take place at a usability lab in order to make sure we record all answers we get during the sessions. A session will consist of us showing the functionality and design of the current prototype, a test where the test persons may use the prototype in order to get to know it better and an interview/brainstorming where the test persons may give feedback. In order to achieve useful feedback we will answer the questions stated in Table \ref{tab:tuidesign} beforehand, thus limiting the options of the participants to a reasonable level. 


Based on the feedback and ideas from the co-design session we will build a new prototype which we will bring back to the next co-design session. These sessions will be arranged at regular short intervals, so we may get much feedback and ensure that we do not go off track when developing AsthmaBuddy.


\section{Summary}


Tangible Interfaces is about giving digital information to physical objects. There are a lot of examples on Tangible Interfaces, and we have taken a brief look into some of them (see \ref{sec:tuiexamples}). There seems to be a lot of activity going on in this research field. However, few are commercially available. 

Champoux proposed a design mechanism to create Tangible Interfaces. The design mechanism is somewhat abstract, and misses out on obvious points like how to actually display information to a user, which Bellotti \etal{} (see \ref{sec:challenges-with-TUI}) exposes as a challenge when creating ubiquitous systems and TUI's. We will use Champoux' approach to a certain extent, and keep Bellotti's challenges in mind when designing our TUI.   

 
