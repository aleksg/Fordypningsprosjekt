\chapter{Tangible User Interfaces}
\label{chp:tangibleinterfaces}

This chapter will introduce the reader to Tangible User Interfaces, and elaborate on some existing research that has been done on the concept.   

\section{About Tangible User Interfaces}

In 1997, Ishii et. al. presented an article called ``Tangible Bits: Towards Seamless Interfaces between People, Bits and Atoms''. They established the term ``Tangible User Interface'' (TUI) as a way to move beyond the dominant model of Graphical User Interfaces. The objective of TUI was explained to \emph{augment the real physical world by coupling digital information to everyday physical objects and environments} \cite{ishii1997tangible}. Thus, TUIs are about giving physical objects a digital meaning. 

Additionally, combining augmented reality (AR) with objects have been proven to improve children's cognitive learning \cite{zhou2004magic}.   

The Karotz is an example of a tangible user interface. It lets the user interact with a rabbit instead of a desktop or tablet, which contains digital information about whether it is time to take medicine, and can send digital information when a child has taken his/her medicine.      


\section{Examples of Use}
Using TUIs instead of GUIs have been proven to work in several different settings. In this section, we will give an overview on some of the domains in which the concept has been proven to work. 

\paragraph{Learning}
Terrenghi et al. designed a cube for learning, giving children quizzes where answers had the shape of text or images \cite{terrenghi2006cube}. Children could then rotate the cube in order to get the correct answer pointing upwards, sort of like a dice. They concluded that the TUI gave children a different set of affordances that prompted a great initial engagement \cite{terrenghi2006cube}. 

\paragraph{Collaborative Learning}
Scarlatos et. al. created a system called TICLE (Tangible Interface in Collaborative Learning), which are used to help children solve a Tangram \cite{scarlatos1999ticle}.  

\paragraph{Interactive Storytelling}
Zhou et. al. designed a cube for storytelling, using a head mounted display and a ``Magic story cube'' in order to let children explore the world while being told a story \cite{zhou2004magic}. Stanton et. al. created a ``Magic Carpet'', giving children possiblity to influence a story in the classroom \cite{stanton2001classroom}. 
 
\paragraph{Social Context}
Marble Answering Machine is an invention by Durrell Bishop, dating back to 1992\cite{crampton1995hand}. The interface allows users to drop marbles into a play-back indent on the system, which plays a recorded message.   

\section{Effects of Robots}

In 2003, Wada et. al. conducted a study on how the introduction of robotics affected the elderly. \cite{wada2004effects}. They conducted a study at a day service center in Japan, where they placed a robotic seal, named Paro, together with the elderly. It had recently been found that animals have a positive effects on blood preassure, depression and loneliness. They placed a robotic seal in the care center, and analyzed the reactions from the elderly. 

The results showed that their mood was better after interacting with Paro over five weeks, and became worse once Paro was no longer there. In addition, nurses burnout rate decreased during the experiment, which implied that the subjects had easier days whenever Paro was there. The study shows that their quality of life was improved after Paro was introduced.           


\section{Are Tangible Interfaces more fun?}
In 2008, Xie et. al. performed a study on how children reacted using different interfaces in order to solve a jigsaw puzzle \cite{xie2008tangibles}. The different interfaces were a physical interface (i.e. a standard jigsaw puzzle), a TUI and a GUI. Their findings were mainly that children enjoyed playing with the different interfaces similarly. However, the children were more likely to start a puzzle over again if the interface were physical or tangible, which implies that a repeated task is more likely to be performed if they're playing with a tangible or physical interface, while it becomes boring to do the same task over and over again on a graphical user interface. It is worth mentioning that the puzzles were being solved by groups of two, and considering the GUI was a computer with one mouse, they didn't get the same sharing experience as with the other interfaces. 


\section{How to make a TUI}
Ullmer states four different properties a system should have in order to be a tangible user interfaces\cite{ullmer2002tangible}:

\begin{enumerate}
	\item{Physically embodied}
	\item{Physically representational}
	\item{Physically manipulable}
	\item{Spatially Reconfigurable}
\end{enumerate}

Ullmer states that these four properties describe systems that use spatially reconfigurable physical artifacts as representations and controls for digital information\cite{ullmer2002tangible}. 


Ullmer proposes three different classes of TUI's: Interactive Surfaces, Token and Constraints, and Constructive Assemblies. These classes are partly based on varying degree of support for continuous and discrete forms of interaction \cite{ullmer2002tangible}.


\subsection{Token + Constraint Approach}
The ``Token + Constraint'' approach centers on a hierarchical relationship between tokens and constraints. Tokens may be placed within or removed from the compatible constraints. The physical shape of the tokens and constraints display whether or not the tokens are compatible or not. This approach support a combination of continuous and discrete interactions. 

\paragraph{Strenghts of token + constraint approach}
Ullmer states that interpretive constraints will help to express which of the physical tokens that can take part within a given interpretive constraint, which physical configurations these physical tokens can take and the demarcation between interaction regions with different computational interpretation.

These interpretive constraints may help to simplify the human perception since humans are good at compare shapes and forms. 
It may help human manipulation since interpretive constraints provide an increased sense of kinesthetic feedback from the manipulation of tokens. 



\subsection{Design Approach to TUI's}
A lot of material exists on the potential benefits of TUI's, but few exists on how to actually create them. Champoux proposes a mechanism to design TUI's based achieving fitness between the form and its context \cite{subramaniandesign}.
He proposes three classes of questions, which corresponds to the different development phases of TUIs:
\begin{itemize}
  \item Defining the boundaries
  \item Orienting the components
  \item Fitting the components
\end{itemize} 


By answering the questions in Table \ref{tab:tuidesign}, his mechanism will ease the development phase.   


\begin{table}[h]
	\begin{tabular}{| p{5.0cm} | p{5.0cm} | p{5.0cm} |}
	\hline
	\textbf{Defining the Boundaries} & \textbf{Orienting the Concepts} & \textbf{Fitting the Components} \\
	\hline
	\textbf{BO1}: What should the user experience? \newline
	\textbf{BO2}: What are the human task? \newline
	\textbf{BO3}: What would the artefact represent and control? \newline 
	\textbf{BO4}: What are the conventions? (Physical ergonomics vs electromechanical) \newline 
	&
	\textbf{OC5a}: What is the nature of the interaction for each sub task? (Continuous vs Discrete vs Assembly) \newline
	\textbf{OC5b}: What are the electromechanical and physical ergonomic constraints for this task? \newline
	\textbf{OC6}: Does the sub-task need any relational interaction? \newline
	&
	\textbf{FC7}: What are the relations between the objects and the actions? \newline 
	\textbf{FC8}: What is the task order when using the artefact? \\ 
	\hline
	
	\end{tabular}
	\caption{The eight questions stated by Champoux \cite{subramaniandesign}}
	\label{tab:tuidesign}
\end{table}  

