\chapter{Tangible Interfaces}
\label{chp:tangibleinterfaces}

This chapter will introduce the reader to Tangible Interfaces, and elaborate on some existing research that has been done on the concept.   

\section{About tangible interfaces}

In 1997, Ishii et. al. presented an article called ``Tangible Bits: Towards Seamless Interfaces between People, Bits and Atoms''. They established the term ``Tangible User Interface'' (TUI) as a way to move beyond the dominant model of Graphical User Interfaces. The objective of TUI was explained to \emph{augment the real physical world by coupling digital information to everyday physical objects and environments} \cite{ishii1997tangible}. Thus, TUIs are all about giving physical objects a digital meaning. 

Additionally, combining augmented reality (AR) with objects have been proven to improve children's cognitive learning \cite{zhou2004magic}.   


In our case, a Karotz is an example of a tangible user interface. It lets the user interact with a rabbit instead of a desktop or tablet, which contains digital information about whether it is time to take medicine, and can send digital messages to notify our database that a child has taken their medicine.      


\section{Usage}
Tangible User Interfaces have been proven to work in several different settings. In this section, we will discuss some of these approaches, giving an overview on which domains the concept has been proven to work. 

\paragraph{Learning}
Terrenghi et al. designed a cube for learning, giving children quizzes where answers had the shape of text or images \cite{terrenghi2006cube}. Children could then rotate the cube in order to get the correct answer pointing upwards, sort of like a dice.

\paragraph{Collaborative learning}
Scarlatos et. al. created a system called TICLE (Tangible Interface in Collaborative Learning), which are used to help children solve a Tangram \cite{scarlatos1999ticle}.  

\paragraph{Interactive storytelling}
Zhou et. al. designed a cube for storytelling, using a head mounted display and a ``Magic story cube'' in order to let children explore the world while being told a story \cite{zhou2004magic}. Stanton et. al. created a ``Magic Carpet'', giving children possiblity to influence a story in the classroom \cite{stanton2001classroom}. 
 
\paragraph{Social Context}
Marble Answering Machine is an invention by Durrell Bishop, dating back to 1992. The interface allows users to drop marbles into a play-back indent on the system, which plays a recorded message. Today, we have Karotz who have the ability to read Twitter or Facebook posts, streaming music, etc.   

\section{Effects of robots}

In 2003, Wada et. al. conducted a study on how robotics affected elderly \cite{wada2004effects}. They conducted a study at a day service center in Japan, where they placed a robotic seal, named Paro, together with the elderly. The argument supporting this study was that it has been found that animals have a positive effects on blood preassure, depression and loneliness. (Omskriving?). The problems is that animals are not allowed in a lot of hosiptals and care centers, because people may have allergic reactions or get scratch marks from it. They placed a robotic seal in the care center, and anylzed the reactions from the elderly. 

The results showed that their mood was better after interacting with Paro in five weeks, and became worse once Paro was no longer there. In addition, nurses burnout rate decreased during the experiment, which implies that they had easier days whenever Paro was there.          

\section{Are Tangible Interfaces more fun?}
In 2008, Xie et. al. performed a study on how children reacted using different interfaces in order to solve a jigsaw puzzle \cite{xie2008tangibles}. The different interfaces were a physical interface (i.e. a standard jigsaw puzzle), a TUI and a GUI. Their findings were mainly that children enjoyed playing with the different interfaces similarly. However, the children were more likely to start a puzzle over again if the interface were physical or tangible, which implies that a repeated task is more likely to be performed if they're playing with a tangible or physical interface, while it becomes boring to do the same task over and over again on a graphical user interface.   


\chapter{Gamification}
\label{chp:gamification}

This chapter will give a description of the term ``gamification'', describe some of the uses of gamification and how we plan to use gamification in our solution.

\section{What is Gamification?}
\label{sec:whatisgamification}
``Gamification'' as a term was first mentioned by Currier in 2008\cite{gamificationcurrier}, but did not become a wide-spread term until 2010. 

Huotari and Hamari\cite{huotari2012defining} define gamification as:

\textit{``Gamification is a process of enhancing a service with affordances for gameful experiences in order to support user's overall value creation.''}

There are many different ways of describing gamification. Deterding, Dixon, Khaled and Nacke\cite{Deterding:2011:GDE:2181037.2181040} define Gamification as:

\textit{``Gamification is the use of game design elements in non-game
contexts.''}

Deterding, Dixon, Khaled and Nacke's definition is often commonly referred to, because of its simplicity and understandability for people who have little or no connection to traditional video games or game consoles.

Today gamification is a much used term both in programming and in the spoken language. Smartphone applications and manufacturers have helped make the term gamification a widespread notion. Examples of this is the application Foursquare, which is built around gamifying ``checking in'' at restaurants, historical sites and similar places\fnurl{Foursquare}{www.foursquare.com}. Apple developed a Game Center for iOS in 2010, giving every iPhone/iPod and iPad user a hub for challenges, awards and other gamelike activities\fnurl{Apple Game Center}{http://support.apple.com/kb/HT4314}, which made every iOS user a potential target for gamification. Lately there have been many games built singularily around gamification, such as Cookie Clicker\fnurl{Cookieclicker}{http://orteil.dashnet.org/cookieclicker/} or Farmville\fnurl{Farmville}{www.farmville.com}. Even game consoles like Playstation 3 and Xbox contain gamification support per default, with their achievement/trophy systems\fnurl{Xbox}{http://xbox.com}\fnurl{Playstation}{http://playstation.com}. While there are many users of such games, they are often criticised for using gamification to lure players into playing. 


\section{What are Serious Games?}
\label{sec:seriousgames}

The term ``serious game'' became a concept with the emergence of the Serious Game Initiative in 2002\cite{seriousgamesinitative}. Their website defines serious games as: 

\textit{``The Serious Games Initiative is focused on uses for games in exploring management and leadership challenges facing the public sector. Part of its overall charter is to help forge productive links between the electronic game industry and projects involving the use of games in education, training, health, and public policy.''} 

This definition has been criticised for being too narrow, and for not including any reason as to why businesses should care. An anonymous author\footnote{The essay is only signed with the name 'Danc'. Still, we regard this essay interesting and relevant, and it has been mentioned in several scientific publications.} posted an essay on \url{www.lostgarden.com} criticizing the definition and suggesting the following definition:

\textit{``Serious Games: The application of gaming technology, process, and design to the solution of problems faced by businesses and other organizations. Serious games promote the transfer and cross fertilization of game development knowledge and techniques in traditionally non-game markets such as training, product design, sales, marketing, etc.''}


Since it's debut in 2002, serious games have later grown to become a multi-billion industry.
Pilots are being trained in simulators, lecturers make lecture quizzes for students\cite{wang2007lecture}, Swedish firefighters have used serious games for training\cite{lebram2009design} and persons suffering from diabetes can use serious games for learning about the illness. These are just a few of the many ways of using serious games. 

Foldit is a very interesting example of how a serious game may lead to solving bigger problems than the game itself\cite{cooper2010predicting}. Foldit is a massive multiplayer online game (MMO). The objective for the player is to fold protein by following a set of rules. The system records how players fold protein and learns patterns for interaction. Humans have much higher skills at interacting with 3D objects than computers, and the system learns patterns and techniques from the players. By playing Foldit, researchers were able to solve the crystal structure of the M-PMV retroviral protease\fnurl{Mason Pfizer Monkey Virus}{http://microbewiki.kenyon.edu/index.php/Mason\_pfizer\_monkey\_virus}\cite{khatib2011crystal}.

Serious games and gamification have many similarities; whereas serious games are mainly targeted towards making education or learning more fun, gamification is used in a number of different ways. 


\section{Discussion about Gamification}
\label{sec:gamificationdiscussion}

Gamification is a much discussed theme, and no agreement seems to have been reached as to whether gamification is useful or not. 
Antin and Churchill argues that gamification may be used for goal setting or instruction\cite{antin2011badges}. Goal setting challenges the users to meet the mark that is set for them, and is known to be an effective motivator\cite{ling2005using}. 

Bogost goes as far as naming gamification as ``marketing bullshit'', used as a way of moneytizing bad business\cite{gamificationbullshit}\footnote{While this is not a scientific publication, we found it interesting and relevant to the discussion}.

McGonigal's studies on how rewards are perceived over time show that: 

\textit{``After three hours of consecutive online play, gamers receive 50 percent fewer rewards (and half the fiero\footnote{Fiero is an italian term for personal triumph\cite{ekman2007emotions}}) for accomplishing the same amount of work.''}\cite{jane2011reality}

Steinung argues that gamification is not powerful enough to make a task interesting\cite{steinung2012interessante}. Simply adding points, badges, a leveling system or similiar, will not make a task interesting on its own. Since gamification is based on behavioural pshychology, poor design may be perceived as interesting, for a shorter period of time\cite{steinung2012interessante}. Zichermann makes a similar statement, saying gamification needs to take ethical precautions\cite{zichermann2011gamification}.

While McGonigal's research focuses on how rewards are percieved when playing over a longer consecutive time, our intent was to make the user spend only small amounts of time using the application. AsthmAPP is a tool, not a pastime.

In order to achieve a meaningful use of gamification Nicholson\cite{nicholson2012user} suggests using a user-centered design approach\cite{usercentereddesign} when developing systems with elements of gamification. Since AsthmaBuddy is a computer supported learning system\cite{stahl2006computer} it was important for us to maintain our focus on the learning and awareness created by our system, making gamification a tool and not the key feature.

\section{Game Elements}
\label{sec:gameelements}

This section will take a brief look into the different classifications of players that exists, and will introduce the reader to the mechanisms commonly used to gamify users' experiences. 

\subsection{Bartle's Four Player Types}
\label{sec:bartlesplayertypes}
People have different preferences when it comes to playing a game. Richard Bartle proposes a classication of four different player types\cite{bartle-gamers}. These types are \emph{Achievers},  \emph{Explorers},  \emph{Socialisers} and \emph{Killers}. We'll take a brief look on each of these in this section. 

\subsubsection{Achievers}
\textit{``Achievers regard points-gathering and rising in levels as their main goal, and all is ultimately subserviant to this''}\cite{bartle-gamers}. 

Most young children will fall under this category. Achievers mostly play games just for the fun of it, and do not necessarily need other incentives to the game than being able to finish the challenge imposed by the game. Most children like to see progress in terms of points, clearing a level or a similar sense of progression. 

\subsubsection{Explorers}
\textit{``Explorers delight in having the game expose its internal machinations to them''}\cite{bartle-gamers}.

Explorers are thus the players who easily enjoy a game more than once, and potentially want to find every secret embedded in the game. Children will in some cases fall under this category, but with our target group, it is hard to separate between achievers and explorers. 
[Children play the game for the fun of playing, not necessarily to find discover secrets][Find something that supports this claim] 

\subsubsection{Socialisers}
\textit{``Socialicers are interested in people, and what they have to say. The game is merely a backdrop, a common ground where things happen to players''}\cite{bartle-gamers}. 

This implies that socialisers play games in order to connect with new people or hang out with their friends. The youngest children in our target group will probably noy fall under this category, as they will not comprehend that there is someone ``on the other side of the screen''.    

\subsubsection{Killers}
\label{sec:killers}
\textit{``Killers get their kicks from imposing themselves on others''}\cite{bartle-gamers}.

``Killers'' thrive upon destroying other people's game experience. Hopefully, no children fall into this category, at least not in our target group. [Something about social skills not developed yet?] 

\subsection{Game Mechanisms Used to Achieve Gamification} 
\label{sec:gamemechanismsusedtoachievegamification}
There are some game mechanisms that are widely used for gamifying every day tasks. This section will explain some of them. We will use a stick figure to examplify each game mechanism. 

\subsubsection{Avatar Systems}
\label{sec:avatarsystems}
Avatars are commonly used in children's games. It gives a player a virtual character, which can be upgraded with different clothing and equipment when players reach certain points in the game. Such equipment can usually be bought for either points awarded or through \emph{In-app purchases}. Players can then show their avatar to other users, compare, and have fun with them. This approach may be seen as giving the avatar a piece of the player's personality. For instance, some players would want their avatar to look as ridiculous as possible, while others would prefer that it looked as cool as possible. Showing off ``expensive'' gear may also give the player a feeling of accomplishment (\emph{``I'm so good at this game that I could afford this golden armour. Have you managed to get it yet?''}). 

\textbf{Example:} The stick figure will be a player's avatar, which can be modified to have different pieces of clothing or equipment.  

\subsubsection{Achievements and Badges}
\label{sec:achievementsandbadges}
Achievements and badges are systems well incorporated into Microsoft's Xbox\fnurl{Xbox}{www.xbox.com} and Sony's Playstation\fnurl{Playstation}{www.playstation.com}. Such achievements and badges are typically given if the player reaches a certain point or level in the game. For instance, on Foursquare, the players gets a badge called ``Adventurer'' if he/she checks in at 10 different venues. 

\textbf{Example:} If we combine this mechanism with avatar systems, we can give out a badge when the stick figure has obtained a complete sets of clothes or a specific set (eg. has purchased all the green clothing).   

\subsubsection{Real-world Rewards}
\label{sec:realworldrewards}
Used together with leaderboards, real-world awards may be given to some of the best players of the game. For instance, they could be rewarded with exclusive tickets to concerts. These real-world awards are often given during marketing campaigns, for instance ``Invite your friends to use this system, and recieve one ticket in the lottery to win a brand new computer''.  

\textbf{Example:} Players may have a real-life stick figure, and the stick figure is rewarded with equipment sent to the player by mail. These rewards could be, for example, different clothing or equipment the player could apply to the figure. 

\subsubsection{Social networking}
\label{sec:socialnetworking}
During the last few years, Facebook feeds has a tendency to be flooded by updates from third-party applications, like Runkeeper\fnurl{Runkeeper}{http://runkeeper.com/}, who updates everyone on your friend list that you have been working out. The idea here is to have a common platform, where users may brag about their accomplishments.      

\textbf{Example:} Social networking may be used to upload images of a player's stick figure, and show it to his/her friends. 

\subsubsection{Mirroring User Behaviour}
\label{sec:mirroringuserbehaviour}
This is most commonly used for children, where an animation or a character shows how to go forward with a procedure. For instance, there are a lot of apps on App Store mirroring the process of brushing a child's teeth. A child may use this app as a reference that indicates how long he/she should brush on the same side.   

\textbf{Example:} The stick figure mirrors the player's intended behavior. 

\subsubsection{Experience Points}
\label{sec:experiencepoints}
Experience points is an indicator of how much experience the player has gaines within a game or setting. These points may be awarded from completing tasks, exploring areas and features or other similar activities. Experience points are usually combined with a leveling system, where the player ``climbs a ladder'' using these experience points, for example by unlocking new levels, new rewards or new features. A player with many experience points is considered an experienced user, and is percieved as higher ranking than a player with less experience points. Experience points are also often combined with leaderboards. 

\textbf{Example:} One experience point may be represented as a stick figure, and the goal is to gather as many stick figures as possible. Another example is that leveling up, based on experience points, may be represented by the size or attributes of your stick figure.

\subsubsection{Leaderboards}
\label{sec:leaderboards}
A leaderboard is a list of the players ordered by their collected points, completed activites or any other predefined system. Each user has a score defined by rules set before a competition started. The score is compared and the players are ranked based on the scores. Leaderboards may be fully dynamic, changing when a player has scored points, or state based, where the new order is determined after a certain period of time.

\textbf{Example:} If the stick figure gathers enough experience points, it may find itself on a regional leaderboard, ranking players in your area (neighbourhood, town, country, etc). 

\subsubsection{Progress Bar}
\label{sec:progressbar}
A progress bar is used to indicate how far a user has come towards a given goal. When the player completes a task or an activity, the progress bar is filled to indicate the progress of getting closer to a goal. How much the progress bar is moved is often determined by the severity of a task or by using points. The progress bar may often be combined with experience points, where the experience points collected determines the movement of the progress bar. 

\textbf{Example:} The stick figure is placed on a road. The figure's position on that road, mirrors the progress a player has made. When completing a task, the figure will be moved closer to its goal.  


\subsubsection{Contests}
\label{sec:contests}
Gamification can be done through having contests either with players in a duel-like head-to-head contest or a free-for-all contest with no limit on the capacity of players. A duel-like contest may be a knockout style of competition where players compete to get the highest amount of points within a given time period or a similar type of a goal. A player may make progress in the tournament without being the player with the highest score, but being better than his/hers opponent. In a free-for-all contest, the winner is whoever fulfills a specific goal to the best degree. The goal which players try to achieve is set be a specific set of rules determined before the start of the contest.

\textbf{Example:} A player's stick figure may enter a voting contest, where votes are given to the best looking one.

\subsection{Combining Game Mechanisms in AsthmAPP}
\label{sec:combininggamemechanismsinasthmapp}
\begin{table}[H]
\begin{tabular}{| p{2.5cm} | p{2.1cm} | p{9.5cm} | }
	\hline
	\textbf{Mechanism} & \textbf{Included in AsthmAPP} & \textbf{Rationale} \\
	\hline
	Avatar systems & No & We did not have time to implement this during our thesis, but we do believe this could be a good feature if it was implemented in a right manner.    
	 \\
	\hline
	Achievements and badges & No & We believe our target group would not enjoy this feature as much as older children, those of 12 - 16 years of age.  \\
	\hline 
	Real-World awards & Yes & Children enjoy the feeling of being rewarded with something real.
	 \\
	\hline
	Mirroring user behavior & Yes & Demonstration has a positive effect on children.
	\\
	\hline
	Leaderboards & No & There are no way to implement this in a realistic and legal way. Children would have to share their data, which consists of points based on medicine doses. Parents could be blamed if their child were on the bottom of the list. It would also be negative for small children's motivation if they have been really good at taking their medicine, and perform poorly on a leaderboard. 
	\\
	\hline
	Social networking & No & Much of the same reasons as why we did not choose leaderboards. We also believe that children in our target group would not understand this aspect.  
	\\
	\hline
	Progress bar & No & In an ideal world, we could have shown how close the child was to becoming fully treated for asthma. However, identifying how close a patient is to becoming healthy, is virtually impossible. Another usage may be to match the child's progress for each week against his/her medicine plan. This would however, imply that by-need treatments would not be included, which may seem unfair for the child. Additionally, their progress would be deleted every week, which would not have much motivational effect.  
	\\
	\hline
	Experience points & Yes & The stars will work as experience points, which may used to cash in rewards from the parents. 
	\\
	\hline
	Contests & No & Using medical history to participate in contests would be a violation of Norwegian privacy laws. It may also be used to pinpoint ``bad'' parents.      
	\\
	\hline
\end{tabular}
\caption{Assessment of different game mechanisms}
\label{tab:game-mech-in-astmapp}
\end{table}

[This table should probably go somewhere else]


\section{Summary}
\label{sec:gamificationinapp}

Stapleton argues that:

\textit{``A variety of [serious game] applications can be thought of here [in Health Care] such as games as a form of motivation and reward for patients undergoing some form of treatment. Games could also be to distract patients during certain procedures such as dental work, for example."}\cite{stapleton2004serious}


Stapleton's argument is one the reasons for why we believe in the use of tangible user interfaces and applications as a method for treating children with asthma. Children are easily distracted, and we believe that making the treatment into a serious game will distract the children enough to forget what they are doing and instead look on the treatment process as a fun game.


In AsthmAPP we aimed to use gamification as a distraction and rewarding element for the children. Instead of putting a lot of predefined badges, rewards, experience points or other rewarding elements, we let the users choose their own rewards, which implies that users may decide what works best for them. This stands to reason with the arguments of Nicholson regarding how to design gamification\cite{nicholson2012user}.

As McGonigal's studies show, the effect of gamification tends to wither down after a longer period of time\cite{jane2011reality}. We aimed to solve this problem by letting the user, together with his/her parents, choose his/her own rewards. While putting the responsibility for the reward system in the hands of the user will lead to more work for the user, we believe that the positive effect of the reward system will make it worthwhile. 
 

The children are rewarded with stars based on their health state. The rationale behind this is that the children may have to take more medicine when they have a cold or there is a lot of pollen in the air. The parents have access to a administrator menu where they may set new rewards for the children. The children will then be able to order the rewards when they have earned a sufficient number of stars. This way the parents and the children create their own gamification environment. Examples of possible rewards could be to give the child an extra 10 NOK in weekly allowance, taking him/her to soccer matches or even to the local amusement park. It is an option where the only boundary is the imagination and how much cost and effort parents wish to invest in it.    


The rewards will appear on a ``milestone'' basis. We do not want children to feel they lose something if they spend stars on a reward, which some may feel as if stars were taken away from them. We do not want to force parents into giving away rewards they can not afford or do not wish to give. The use of the reward system is optional and decided by the user, making the user in control of how they wish to gamify the experience. 
We do not wish to have the children spending too much time using the application, since using a tablet or phone at such a young age is considered unhealthy. This had some implications on the complexity of our gamification system. 

