\chapter{Tangible Interfaces}
\label{chp:tangibleinterfaces}

This chapter will introduce the reader to Tangible Interfaces, and elaborate on some existing research that has been done on the concept.   

\section{About tangible interfaces}

In 1997, Ishii et. al. presented an article called ``Tangible Bits: Towards Seamless Interfaces between People, Bits and Atoms''. They established the term ``Tangible User Interface'' (TUI) as a way to move beyond the dominant model of Graphical User Interfaces. The objective of TUI was explained to \emph{augment the real physical world by coupling digital information to everyday physical objects and environments} \cite{ishii1997tangible}. Thus, TUIs are all about giving physical objects a digital meaning. 

Additionally, combining augmented reality (AR) with objects have been proven to improve children's cognitive learning \cite{zhou2004magic}.   


In our case, a Karotz is an example of a tangible user interface. It lets the user interact with a rabbit instead of a desktop or tablet, which contains digital information about whether it is time to take medicine, and can send digital messages to notify our database that a child has taken their medicine.      


\section{Usage}
Tangible User Interfaces have been proven to work in several different settings. In this section, we will discuss some of these approaches, giving an overview on which domains the concept has been proven to work. 

\paragraph{Learning}
Terrenghi et al. designed a cube for learning, giving children quizzes where answers had the shape of text or images \cite{terrenghi2006cube}. Children could then rotate the cube in order to get the correct answer pointing upwards, sort of like a dice.

\paragraph{Collaborative learning}
Scarlatos et. al. created a system called TICLE (Tangible Interface in Collaborative Learning), which are used to help children solve a Tangram \cite{scarlatos1999ticle}.  

\paragraph{Interactive storytelling}
Zhou et. al. designed a cube for storytelling, using a head mounted display and a ``Magic story cube'' in order to let children explore the world while being told a story \cite{zhou2004magic}. Stanton et. al. created a ``Magic Carpet'', giving children possiblity to influence a story in the classroom \cite{stanton2001classroom}. 
 
\paragraph{Social Context}
Marble Answering Machine is an invention by Durrell Bishop, dating back to 1992. The interface allows users to drop marbles into a play-back indent on the system, which plays a recorded message. Today, we have Karotz who have the ability to read Twitter or Facebook posts, streaming music, etc.   

\section{Effects of robots}

In 2003, Wada et. al. conducted a study on how robotics affected elderly \cite{wada2004effects}. They conducted a study at a day service center in Japan, where they placed a robotic seal, named Paro, together with the elderly. The argument supporting this study was that it has been found that animals have a positive effects on blood preassure, depression and loneliness. (Omskriving?). The problems is that animals are not allowed in a lot of hosiptals and care centers, because people may have allergic reactions or get scratch marks from it. They placed a robotic seal in the care center, and analyzed the reactions from the elderly. 

The results showed that their mood was better after interacting with Paro over five weeks, and became worse once Paro was no longer there. In addition, nurses burnout rate decreased during the experiment, which implies that they had easier days whenever Paro was there.          

\section{Are Tangible Interfaces more fun?}
In 2008, Xie et. al. performed a study on how children reacted using different interfaces in order to solve a jigsaw puzzle \cite{xie2008tangibles}. The different interfaces were a physical interface (i.e. a standard jigsaw puzzle), a TUI and a GUI. Their findings were mainly that children enjoyed playing with the different interfaces similarly. However, the children were more likely to start a puzzle over again if the interface were physical or tangible, which implies that a repeated task is more likely to be performed if they're playing with a tangible or physical interface, while it becomes boring to do the same task over and over again on a graphical user interface.   


\section{What is gamification?}
\label{sec:gamification}

``Gamification'' as a term was first mentioned by Currier in 2008\cite{gamificationcurrier}, but did not become a wide-spread term before 2010. Today gamification is a much used term both in programming and in the spoken language. Smartphone applications and manufacturers have helped make the term gamification a widespread notion. Examples of this is the application Foursquare, which is built around gamifying ``checking in'' at restaurants, historical sites and similar places \cite{foursquare}. Apple developed a Game Center for iOS in 2010, giving every iPhone/iPod and iPad user a hub for challenges, awards and other gamelike activities\cite{applegamecenter}. Lately there has been many games built singularily around gamification, such as Cookie Clicker \cite{cookieclicker} or Farmville \cite{farmville}. Even console's like PLAYSTATION 3 and XBOX contain gamification support per default, with their achievement/trophy systems \cite{xbox, playstation}. While there are many users of such games, they are often critiqued for using gamification to lure players into playing. 

There are many different ways of describing gamification. Deterding, Dixon, Khaled and Nacke\cite{Deterding:2011:GDE:2181037.2181040} defines Gamification as:

\textit{Gamification is the use of game design elements in non-game
contexts.}

Huotari and Hamari\cite{huotari2012defining} defines gamification as:

\textit{Gamification is a process of enhancing a service with affordances for gameful experiences in order to support user's overall value creation.}

Deterding, Dixon, Khaled and Nacke's definition often commonly referred to, because of it's simplicity and understandability for people who have little or no connection to traditional games or games consoles.
Gamification is a much discussed theme, where there does not seem to be an agreement as to which gamification is a useful or not. 
Antin and Churchill\cite{antin2011badges} argues that gamification may be used for goal setting or instruction. Goal setting challenge the users to meet the mark that is set for them, and is known to be an effective motivator \cite{ling2005using}. 

Bogost goes as far as naming gamification as ``marketing bullshit''\cite{gamificationbullshit}, used as a way of moneytizing bad business.

McGonigal's studies\cite{jane2011reality} on how rewards are perceived over time show that: 

\textit{After three hours of consecutive online play, gamers receive 50 percent fewer rewards (and half the fiero) for accomplishing the same amount of work.}

Steinung\cite{steinung2012interessante} arguments for gamification not being powerful enough to make a task interesting. Simply adding points, badges, a leveling system or similiar, won't make a task interesting on its own. Since gamification is based on behavioural pshychology, poor design may be perceived as interesting, for a shorter period of time \cite{steinung2012interessante}. Zichermann makes a similar statement, saying gamification needs to take ethical precautions \cite{zichermann2011gamification}.

McGonigal's statement is central to our research, since we aim to research how [INSERT APPNAME] is perceived by children over a longer period of time. While McGonigal's research dives into how rewards are percieved when playing over a longer consecutive time, our goal is to make the users use the application for a short period of time, everyday.

In order to achieve a meaningful use of gamification Nicholson\cite{nicholson2012user} suggests using a user-centered design approach\cite{usercentereddesign} when developing system with elements of gamification. 

\section{Use of Gamification in [INSERT APPNAME]}

Our hypothesis is that children need a motivating factor in order to take their medicine, as they do not necessarily understand why they take it, and gamifying their treatment might be a feasible motivation factor.  


In [INSERT APPNAME] we aim to use gamification as a distraction and rewarding element for the children. Instead of putting a lot of predefined badges, rewards, experience points or other rewarding elements, we let the users choose their own rewards, which implies that users can decide what is best for themselves.  
 

The children are rewarded with stars based on their health state. The reason for this is that the children may have to take more medicine when they have a cold or there is much pollen in the air. The guardians have access to a administrator menu where they may set new rewards for the children. The children will then be able to order the rewards when they have earned enough stars. This way the guardians and their children create their own gamification environment, and the application do not force parents to do something that for some ethnographical or sociological reason. Examples on possible rewards could be to give their child an extra 10 kr. in week salary, taking them to football matches or even go to the local amusement park. It is an option where the only boundary is the imagination and how much effort guardians want to put into it.    


The rewards will appear at a ``milestone'' basis. We do not want children to feel they lose something if they buy a reward, which some probably could feel if stars were taken away from them.


An overview of the screen shots, architecture and logic behind the gamification elements is found in Section [INSERT REFERENCE].

%Grunnlag:
% We as researchers do not have to force parents to do something they cannot afford, are not interested in,  etc.
%The problem with our target group, is that we don\'t want children to use too much time on their parent's smartphone. Spending too much time on such devices in our targeted age is broadly considered unhealthy. Thus we have the following proposition to our gamification element.

%TODO: Få med denne setningen, på et vis
%This relates to our research, because we don't necessarily want children to be spending a considerable amount of time playing around with the application. We need to figure out a gamification element that does not give this need to children. 