\section{Co-design}
\label{sec:codesign}
This section will describe co-design and how we plan to use a co-design approach when developing our TUI and the gamification aspect of AsthmAPP.

\subsection{What is co-design?}
Co-design is a product, service, or organization development process where design professionals empower, encourage, and guide users to develop solutions for themselves. Co-design encourages the blurring of the role between user and designer, focusing on the process by which the design objective is created \cite{sanders2008co}. By encouraging the trained designer and the end user to create ideas and solutions together, the final result will be more appropriate and acceptable to the end user \cite{albinsson2007co}. 
Albinsson et. al. \cite{albinsson2007co} discovered that having a co-design approach to development of a e-mail sorting system lead to the ability of easier conflict management, ability to centre innovation on the client/customer and made it easier to manage a project with an unknown outcome. While co-design has been around for over 40 years, it draws it's roots from user-centered design and participatory design. Co-design differs from user-centered design approaches in that it acknowledges that the client or beneficiary of the design may not be using the artifact itself \cite{norman1986user}.



\subsection{Using co-design to build AsthmaBuddy and AsthmAPP}
Based on the research we have read, and recommendations from domain experts, we have decided to try a co-design approach to develop the gamification aspect of AshtmAPP and our TUI; AsthmaBuddy. 
When developing AsthmaBuddy, we will build a low-fidelity prototype which we can show to a group of test persons and domain experts. These sessions will take place at a usability lab in order to make sure we record all answers we get during the sessions. A session will consist of us showing the functionality and design of the current prototype, a test where the test persons may use the prototype in order to get to know it better and an interview/brainstorming where the test persons may give feedback. In order to achieve useful feedback we will answer the questions stated in Table \ref{tab:tuidesign} beforehand, thus limiting the options of the participants to a reasonable level. 

Based on the feedback and ideas from the co-design session we will build a new prototype which we will bring back to the next co-design session. These sessions will be arranged at regular short intervals, so we may get much feedback and ensure that we do not go off track when developing AsthmaBuddy.

Regarding the development of AsthmAPP, we will focus mainly on the gamification aspect during the codesign sessions. The test persons will be asked to test out the application at home, and give feedback on how the test went. The first version of AthmAPP has a very open and user-controlled system for the rewards (see Section \ref{gamificationinapp}). While this is an idea we consider appropriate initially, this may change based on user experience and feedback. The reward system using stars will also need testing, since we have little knowledge of how this is percieved by children over time.
