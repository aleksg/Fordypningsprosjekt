\section{What is gamification?}
\label{sec:gamification}

``Gamification'' as a term was first mentioned by Currier in 2008\cite{gamificationcurrier}, but did not become a wide-spread term before 2010. Today gamification is a much used term both in programming and in the spoken language. Smartphone applications and manufacturers have helped make the term gamification a widespread notion. Examples of this is the application Foursquare, which is built around gamifying ``checking in'' at restaurants, historical sites and similar places \cite{foursquare}. Apple developed a Game Center for iOS in 2010, giving every iPhone/iPod and iPad user a hub for challenges, awards and other gamelike activities\cite{applegamecenter}. Lately there has been many games built singularily around gamification, such as Cookie Clicker \cite{cookieclicker} or Farmville \cite{farmville}. Even console's like Playstation 3 and XBox contain gamification support per default, with their achievement/trophy systems \cite{xbox, playstation}. While there are many users of such games, they are often critiqued for using gamification to lure players into playing. 


There are many different ways of describing gamification. Deterding, Dixon, Khaled and Nacke\cite{Deterding:2011:GDE:2181037.2181040} defines Gamification as:

\textit{Gamification is the use of game design elements in non-game
contexts.}

Huotari and Hamari\cite{huotari2012defining} defines gamification as:

\textit{Gamification is a process of enhancing a service with affordances for gameful experiences in order to support user's overall value creation.}

Deterding, Dixon, Khaled and Nacke's definition often commonly referred to, because of it's simplicity and understandability for people who have little or no connection to traditional games or games consoles.
Gamification is a much discussed theme, where there does not seem to be an agreement as to which gamification is a useful or not. 
Antin and Churchill\cite{antin2011badges} argues that gamification may be used for goal setting or instruction. Goal setting challenge the users to meet the mark that is set for them, and is known to be an effective motivator \cite{ling2005using}. 

Bogost goes as far as naming gamification as ``marketing bullshit''\cite{gamificationbullshit}, used as a way of moneytizing bad business.

McGonigal's studies\cite{jane2011reality} on how rewards are perceived over time show that: 

\textit{After three hours of consecutive online play, gamers receive 50 percent fewer rewards (and half the fiero) for accomplishing the same amount of work.}

Steinung\cite{steinung2012interessante} arguments for gamification not being powerful enough to make a task interesting. Simply adding points, badges, a leveling system or similiar, won't make a task interesting on its own. Since gamification is based on behavioural pshychology, poor design may be perceived as interesting, for a shorter period of time \cite{steinung2012interessante}. Zichermann makes a similar statement, saying gamification needs to take ethical precautions \cite{zichermann2011gamification}.

McGonigal's statement is central to our research, since we aim to research how [INSERT APPNAME] is perceived by children over a longer period of time. While McGonigal's research dives into how rewards are percieved when playing over a longer consecutive time, our goal is to make the users use the application for a short period of time, everyday.

In order to achieve a meaningful use of gamification Nicholson\cite{nicholson2012user} suggests using a user-centered design approach\cite{usercentereddesign} when developing system with elements of gamification. 

\section{Use of Gamification in [INSERT APPNAME]}

Our hypothesis is that children need a motivating factor in order to take their medicine, as they do not necessarily understand why they take it, and gamifying their treatment might be a feasible motivation factor.  


In [INSERT APPNAME] we aim to use gamification as a distraction and rewarding element for the children. Instead of putting a lot of predefined badges, rewards, experience points or other rewarding elements, we let the users choose their own rewards, which implies that users can decide what is best for themselves.  
 

The children are rewarded with stars based on their health state. The reason for this is that the children may have to take more medicine when they have a cold or there is much pollen in the air. The guardians have access to a administrator menu where they may set new rewards for the children. The children will then be able to order the rewards when they have earned enough stars. This way the guardians and their children create their own gamification environment, and the application do not force parents to do something that for some ethnographical or sociological reason. Examples on possible rewards could be to give their child an extra 10 kr. in allowance, taking them to soccer matches or even go to the local amusement park. It is an option where the only boundary is the imagination and how much effort guardians want to put into it.    


The rewards will appear at a ``milestone'' basis. We do not want children to feel they lose something if they buy a reward, which some probably could feel if stars were taken away from them. We do not want to force parents in to giving away rewards they can not afford or do not wish to use, therefore we will be testing out the application without the gamification elements, in order to find how the elements will affect the use. The use of rewards is also optional and decided by the user, making them in control of how they wish to gamify the experience. 
We do not wish to have the children spending too much time using the application, since using a tablet or phone at such a young age is considered unhealthy, which is taken into consideration with making an application which is mostly used ``outside'' the digital application itself. The children may recieve rewards and use the TUI without touching the smart phone.

An overview of the screen shots, architecture and logic behind the gamification elements is found in Section \ref{sec:architecture}.

