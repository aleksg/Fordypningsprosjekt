\section{What is gamification?}
\label{sec:gamification}

``Gamification'' as a term was first mentioned by Currier in 2008\cite{gamificationcurrier}, but did not become a wide-spread term before 2010. 

There are many ways to describe gamification. Deterding, Dixon, Khaled and Nacke\cite{Deterding:2011:GDE:2181037.2181040} defines Gamification as:

\textit{Gamification is the use of game design elements in non-game
contexts.}

Huotari and Hamari\cite{huotari2012defining} defines gamification as:

\textit{Gamification is a process of enhancing a service with affordances for gameful experiences in order to support user's overall value creation.}

Deterding, Dixon, Khaled and Nacke's definition often commonly referred to, because of it's simplicity.

Gamification is a much discussed theme, where there does not seem to be an agreement as to which gamification is a useful or not.

Antin and Churchill\cite{antin2011badges} argues that gamification may be used for goal setting or instruction. Goal setting challenge the users to meet the mark that is set for them, and is known to be an effective motivator \cite{ling2005using}. 

Bogost goes as far as naming gamification as ``marketing bullshit''\cite{gamificationbullshit}, used as a way of moneytizing bad business.

McGonigal's studies\cite{jane2011reality} on how rewards are perceived over time show that: 

\textit{After three hours of consecutive online play, gamers receive 50 percent fewer rewards (and half the fiero) for accomplishing the same amount of work.}

Steinung\cite{steinung2012interessante} arguments for gamification not being powerful enough to make a task interesting. Simply adding points, badges, a leveling system or similiar, won't make a task interesting on its own. Since gamification is based on behavioural pshychology, poor design may be perceived as interesting, for a shorter period of time \cite{steinung2012interessante}. Zichermann makes a similar statement, saying gamification needs to take ethical precautions \cite{zichermann2011gamification}.

McGonigal's statement is central to our research, since we aim to research how CAPP/GAPP/KAPP is perceived  by children over a longer period of time.
%Mulig den siste setningen er litt far-out. Må finne mer referanser.

In order to achieve a meaningful use of gamification Nicholson\cite{nicholson2012user} suggests using a user-centered design approach\cite{usercentereddesign} when developing system with elements of gamification. 

