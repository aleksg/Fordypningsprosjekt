\chapter{Results}
\label{chp:results}


\section{Feedback from parents}

[SKAL SKRIVES OM VED FLERE RESULTATER]

We spoke to a mother of a child who have a ``weak'' form of asthma, i.e. he did not have severe attacks very often. This section will summarize some of the findings: 

Boy, about 5 years old
\begin{itemize}
  \item She did not have to ``fight'' with him very often
  \item It was hard to stay on the medicine plan correctly
  \item She missed an application that could easily structure when and how much medicine had been taken.
  \item She found it difficult to meet up at doctor appointments, and remember when she had switched medicine plans. 
  \item In order to calm her child, she would sing or count while he was breathing
  \item As far as rewards go, she had experimented with using bumperstickers on a sheet of paper while giving him toilet training. In her opinion, this was rewarding enough. She did not that our reward system could be useful for other kids though. 3-7 years old is a very large span in terms of cognitive development, and we should have this in mind. 
  \item Alarms were, in her opinion, the most important feature. 
  \item Demonstration worked pretty good, as it calmed the child down. 
  \item She was positive to having a bear at home that could help children take their medicine. However, she would not let him take medicine by himself. 
  \item Sidenote: She is a kindergarten teacher, which implies that she gives asthma medicine to several children. An application that could help kindergarten teachers remind when a specific child was to take their medicine would be much appreciated.
  \item She did not see that it was need for By Need treatments via an application or a teddybear, as it would slow the process down. Children were often stressed out when they had an asthma attack, so having a teddybear \emph{could} help during the process to calm him down. 
  \item Teaching children about their disease could be given more of an effort. However, this was outside the scope of our project. 
  \item Having a shared mobile application across caregivers could be benefitiary, as she often had to explain to give a medicine to others.  
\end{itemize} 



\section{Effects of Gamification in The Treatment Process}
Our solution to gamification is highly coupled with parents' initiative. In order for our reward system to have any motivational effect, parents has to be highly involved. They need to understand how often their child needs a reward, in addition to understanding what defines a ``good'' reward for their children. 

Webster-Stratton and Herbert claims that


\textit{Preschool children aged between the ages of three and four may be rewarded by the special sticker or token itself without needing a back-up reinforcer. Youngsters aged four to six should be able to trade in stickers for something each day if they like. Children of seven and eight can wait a few days before getting a reward. } \cite{webster1994troubled}


\section{How can tangible user interfaces be used to help children with asthma?}
When we first started this project we aimed to look into how tangible user interfaces may be of use for children with asthma. This section presents our findings.


\paragraph{Information spreading and educating asthmatic children and their parents}
[Reasons we believe in this]

\textit{``Preschool children aged between the ages of three and four may be rewarded by the special sticker or token itself without needing a back-up reinforcer. Youngsters aged four to six should be able to trade in stickers for something each day if they like. Children of seven and eight can wait a few days before getting a reward.''} \cite{webster1994troubled}

In order for our AsthmAPP and AsthmaBuddy to be a motivational success, parents have to be aware over their children's maturity. They need to understand the above statement, and use this for their own children. 
