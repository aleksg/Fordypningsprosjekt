\chapter{Results}
\label{chp:results}

\section{Interviews}
\label{sec:interviewresults}

We performed interviews on a set of domain experts in order to receive input on the work we had accomplished. Table \ref{tab:interviewsperformed} summarizes the interviews we performed, why we interviewed these subjects and the topics we covered during the interviews. Transcripts from the interviews are included in Appendix \ref{app:interview-transcripts}. 

Where we have interviewed parents of children with asthma, we have named those Parent N (N is a number), in order to protect the identity of their children.  

\begin{sidewaystable}
\centering
\begin{tabular}{| p{3.0cm} | p{4.0cm} | p{3.5cm} | p{6.0cm} | p{2.5cm} |}
	\hline
	\textbf{Name} & \textbf{Background} & \textbf{Rationale} & \textbf{Topics covered (keywords)} & \textbf{Reference to interview transcript} \\
	\hline
	Nanna S\o nnichsen Kayed & PhD/Researcher in Psychology. & Input on reward system. & Rewards for children, Gamification, Motivation. & Appendix \ref{sec:psychinterview} \\
	\hline
	Marikken H\o iseth & PhD cand in industrial design. Experience from the BLOPP project. & Collect data to design for children. & Reward systems for children, designing for children, interaction design. & Appendix \ref{sec:marikkeninterview} \\
	\hline
	Rose Lyngra & Senior Advisor at NAAF. & Has expertise on asthma in general. & Asthma among children, smartphone applications, motivation. & Appendix \ref{sec:roseinterview} \\
	\hline
	Two nurses & Nurses with expertise on asthma treatment. & Collecting data from treatment experts. & Information in \ab{} and \app{}, smartphone applications, problems with asthma. & Appendix \ref{sec:nursesinterview} \\
	\hline
	Parent 1 & Parent of a child with asthma. Has expertise in computer science. & Collecting data from parents. & Asthma in the family, smartphone applications, TUI, motivation, rewarding children, asthmatic child at school/kindergarten. & Appendix \ref{sec:parent1interview} \\
	\hline
	Parent 2 & Parent of a child with asthma. Works in a kindergarten. & Collecting data from parents. & Asthma in the family, reward systems, asthma in the kindergarten, teaching inexperienced users. & Appendix \ref{sec:parent2interview} \\
	\hline     
\end{tabular}
\caption{Interviews performed during the project}
\label{tab:interviewsperformed}
\end{sidewaystable}  

\subsection{Discoveries Found During Interviews}

\textbf{Operating time}

The longest duration for taking a planned treatment was about 2 minutes for an experienced user, and about 2.5 minutes for an inexperienced user. This treatment included the part where \buddy{} completes the treatment before the child. A normal treatment taken without \ab{} usually takes 1 - 2 minutes, according to our interview subjects. Since \ab{} only takes 0.5 - 1.5 minutes longer than normal, it should not be considered a time-waster. From another point-of-view, if a child does not want to take his/her medication, it may cause an argument and it may take a much longer time to complete the treatment. If \ab{} can help on shortening the time spent arguing with the child, a time may be saved.      

\textbf{By need treatments}

We got some feedback regarding the \emph{by need} treatments. Some parents stated that they would not use \buddy{} to complete a by need treatment, as their child was suffering from an asthma attack. However, one of the parents noted that if children were used to performing their treatment with \buddy{}, it could create a dependency toward it, i.e. \buddy{} could help the parents to calm the child. Whether or not the by need functionality would actually be used needs more research, as \buddy{} would need to be placed within a home and be nearby when such an attack occurs. 

\textbf{\ab{} as a stand-alone system}

One of the interview subjects commented that \ab{} should be able to operate as a stand-alone system if the parents do not allow the child to borrow a smartphone and the child does not have a smartphone of his/her own. The same interview subject commented that she would not take the time to use \ab{} for two minutes every time the child needed to take a treatment, and that \ab{} might become boring after a period of time.

\section{Testing \ab{} on Inexperienced Users}
\label{chp:interaction-methods}

Before we did user tests on children, we ran a round of tests to verify \ab{}'s ability to explain the treatment process for inexperienced users. We tested on ten students at NTNU, where one of them had asthma during childhood. While doing so, we also tested the different interaction methods \ab{} could be used with. The reason we wanted to test the interaction methods was to get an overview of how adults perceived the interactions. If adult users are incapable of doing some of the interaction methods, we figured that children were probably incapable of doing the same. Additionally, since we had problems to get a significant amount of children to test the system on (which will be elaborated further in Chapter \ref{sec:difficultyfindingtestusers}), we did not want to waste a usability test on a child by initially having a bad interaction design.         

The results of the interaction testing is summarized in table \ref{tab:interactioneval}.  

\begin{table}[H]
\begin{tabular}{|p{4.0cm} | p{7.5cm} | p{2.5cm} |}
\hline 
\textbf{Interaction Method} & \textbf{Comments} & \textbf{Suitable for children?}\\
\hline
	Give AsthmaBuddy a ``High Five'' & Worked out fine. A high five is cool and may make \ab{} seem more friendly to the children. & Yes \\
\hline
	Hold AsthmaBuddy's hand & Easy to understand and use during a treatment. Should give feedback to indicate that the user has interacted correctly. & Yes\\
\hline
	Hold smartphone close to AsthmaBuddy's belly & Smart phones are cool, but may be easily damaged if dropped. To risky to let small children handle a smartphone. The size of the smartphone may require use of two hands, which may cause complications for the child. & No \\
\hline
	Press AsthmaBuddy's nose & Users tended to press the LED light on the nose. Risk of damage to light. & No\\
\hline
	Press \buddy{}'s belly & Easy to understand and use during a treatment. Should give feedback to indicate that the user has interacted correctly. & Yes\\
\hline
	Hold medicine close to AsthmaBuddy's mouth & Created some complications when the user was supposed to hold the mask to his/her mouth and then hold the medicine close to \ab{}'s mouth to proceed. & No \\
\hline
	Hold RFID-chip close to \buddy{}'s nose & The thickness of \buddy{}'s nose made it difficult for the RFID tag to communicate with our reader, this caused problems for the user. & No \\
\hline
	Hold RFID-chip close to \buddy{}'s belly & Works fine. Letting children have their ``magic token'' which interacts with \ab{} may be cool for them. & Yes\\
\hline
	Clap your hands & Works fine. At one point the user has to clap hands when having the mask in his/her hands, which may cause some problems, but should not be a big problem. & Yes\\
\hline
	A variation of the above interactions & Some of the interaction methods made it confusing for the user, e.g. they were asked to hold the medicine towards \ab{} before having fetched the medicine. & Yes, but in a revised form.\\
\hline
\end{tabular}
\caption{Evaluation of interaction methods for AsthmaBuddy}
\label{tab:interactioneval}
\end{table}

\subsection{Observations Made During Tests}

In order for \ab{} to be useful for inexperienced users, it could have even clearer and more informative instructions. Even though it may seem self-explanatory to take the cap off the medicine before mounting it on the mask, it may not be that obvious to new users. Since \ab{}'s purpose is to instruct and inform, it should have a completely ``foolproof'' instructions. For instance some of the test users tried to attach the inhaler to the mask without removing the protective cap from the inhaler. Since the mask's mount is made from rubber, it gave them the idea that one should just push the medicine into the mount by force, which is incorrect. 

When using \ab{}, some users found it difficult to hear all of the instructions. Supporting replay of the last instruction was important. 

\section{Usability Test Results}
\label{sec:usabilityresults}
In order to protect the identity of our test subjects, especially considering the children, we have used identifiers as names. Names starting with the letter ``A'' is an adult user, and names starting with ``C'' denotes a child.

\subsection{Parent partition tests}
\begin{table}[H]
\begin{tabular}{|p{4.0cm} | p{4.0cm} |}
	\hline
	\textbf{Name} & AU1\\
	\hline
	\textbf{Age} & 36 \\
	\hline
	\textbf{Date} & May 2nd, 2014 \\
	\hline
	\textbf{Testleader} & Aleksander\\
	\hline
	\textbf{Observer} & Esben\\
	\hline	
\end{tabular}
\end{table}

\begin{singlespacing}
\begin{table}[H]
\begin{tabular}{| p{1.0cm} | p{4.0cm} | p{4.5cm} | p{4.0cm} |}
\hline
	\textbf{Task} & \textbf{Problem} & \textbf{Cause} & \textbf{Proposal for solution} \\
	\hline
	0 & The user where unable to separate between the image for child and parent partition & The images were not entirely intuitive & The image for adults could have a bearded man, or other recognizable features. \\
	\hline
	1 & It was unclear whether he had to press on the healthy medicine plan, as the child was in the healthy medicine plan by default. & Challenging GUI & Could make it clearer for the user which treatment plan is being followed \\
	\hline
	1 & The spinners for hour/minutes should be able to be written into. & The standard Android slider contains a minor bug when one writes into it.  & N/A \\
	\hline
	1 & The time that shows when the alarm should fire contained seconds, which the test subject found unnecessary. & N/A & Remove seconds from the timestamp \\
	\hline
	2 & The user expected that he could be able to press the medicine, and not just the checkbox which was the case. He found this a little annoying. & Implementation of listener & Make the entire list item touchable \\
	\hline
	2 & He wanted functionality for adding two medicines at once.  & This has not been implemented yet.  & Implement it later \\
	\hline
	3 & The view shows a button with ``Add Activity'', which he felt was wrong. & This was not intended, as it should have said ``Add Reward'' & Change it to ``Add Activity'' \\
	\hline
	3 & The user pressed the back button one too many times. This caused the PIN-challenge to be presented again, which the test user said could be annoying for some users.  & PIN-challenge appears as soon a parent returns from the parent partition. & Could implement a timer who checks when the user last completed the challenge.\\
	\hline
	4 & The test user said it was not logical to see where he could check the air quality cast. & Bad task description & Change task description to make it clear \\
	\hline
	4 & The test user wanted functionality for different views, for instance showing the log for a week or a single day. & Too high expectations & This feature could be implemented with more time and resources.  \\
	\hline
\end{tabular}
\label{tab:test1}
\caption{Usability result}
\end{table}
\end{singlespacing}

\subsection{Children partition}

During the user tests, \ab{} was configured to use the varied interaction scheme. This involved a preset combination of the remaining interactions from Table \ref{tab:interactioneval}. This decision was made in order to test all of the possible interactions, as we did not have enough users to test them one-by-one. 

\subsubsection{Child User 1}
\begin{table}[H]
\begin{tabular}{| p{4.0cm} | p{4.0cm} |}
\hline
 \textbf{Name} & CU1 \\
 \hline
 \textbf{Age} & 6 years old \\
 \hline 
 \textbf{Date} & May 2nd, 2014 \\
 \hline
 \textbf{Testleader} & Aleksander \\
 \hline
 \textbf{Observer} & Esben \\
 \hline
\end{tabular}
\end{table}

\begin{table}[H]
\begin{tabular}{| p{3.0cm} | p{3.0cm} | p{3.0cm} | p{3.0cm} |}
\hline
	\textbf{Task} & \textbf{Problem} & \textbf{Cause} & \textbf{Proposal for solution} \\
	\hline
	2 & It seemed like she had a hard time keeping up with \ab{}'s instructions & The voice of \ab{} was speaking to fast & Record sounds with lower speed \\
	\hline
	4 & It was difficult to drag the medicine above the mask in order to start the treatment & The treatment only starts when the medicine is directly above the mask & Should make this functionality simpler to start.  \\
	\hline
	4 & It was hard for the child to keep ut with the voice of the rabbit. & The voice talked to quickly, and \app{} does not have a repeat functionality when a treatment is running. & Should consider implementing repeat.\\ 
	\hline
	5 & It was hard to get a clean read of the RFID tag. & It was not entirely clear where the user had to put the card in order to get a read. & \ab{} should have an indicator as of where the card should be held in order to be read.  \\
	\hline
\end{tabular}
\label{tab:test2}
\caption{Usability result}
\end{table}

CU1 was very shy when arriving at the test lab. It quickly became clear for us that we had to leave the area and rather observe from the back room, in order for her to speak up. The parent was instructed with the tasks that were to be performed, and he explained the tasks to her. Once we were back stage, she started responding to the instructions given. We made a note that the observer should sit in the back room and observe from there, in order for the children to respond more easily.   

When asked which method CU1 preferred, i.e. \app{} or \ab{}, CU1 replied ``I don't know''. CU1 was also asked if both were equally fun, which to CU1 replied ``Yes''. CU1 was also asked if the usage of \app{} and \ab{} was more fun than a regular treatment, which to CU replied ``Yes''\footnote{There is reason to belive that the answer was biased due to the reward}. We asked if this was because of her reward, which was candy, but we were unable to get a reply. 

\subsubsection{Child User 2}
\begin{table}[H]
\begin{tabular}{| p{3.0cm} | p{3.0cm} | p{3.0cm} | p{3.0cm} |}
\hline
	\textbf{Task} & \textbf{Problem} & \textbf{Cause} & \textbf{Proposal for solution} \\
	\hline
	place & place & place & place \\
	\hline
	place & place & place & place \\
	\hline
	place & place & place & place \\
	\hline
	place & place & place & place \\
	\hline
	
\end{tabular}
\label{tab:test1}
\caption{Usability result}
\end{table}

\begin{table}[H]
\begin{tabular}{| p{3.0cm} | p{3.0cm} | p{3.0cm} | p{3.0cm} |}
\hline
	\textbf{Task} & \textbf{Problem} & \textbf{Cause} & \textbf{Proposal for solution} \\
	\hline
	place & place & place & place \\
	\hline
	place & place & place & place \\
	\hline
	place & place & place & place \\
	\hline
	place & place & place & place \\
	\hline
\end{tabular}
\label{tab:test1}
\caption{Usability result}
\end{table}



\section{Evaluation}
\subsection{\ab{}}

After completing all of the validation tests, it occured to us that not one out of the three children we tested \ab{} with were able to listen to \emph{every} instruction and interact accordingly. I.e. every user had to make use of the repeat functionality. There is reason to believe that this is a problem caused by the speed \ab{} talks in. It seemed like a somewhat hard task for children to keep up with both the instruction, e.g. ``Shake the blue medicine, and fetch it to your mask'', and the interaction that was to performed, e.g. ``Clap your hands to proceed''. 

As mentioned, CU3 discovered that \ab{} was not stable enough to handle a proper high five. This problem would probably have been avoided if the child had gotten used to interacting with \ab{}, giving him more knowledge of which preventive measures that needs to be taken, e.g. supporting his back while giving him the high five. However, a more valid argument is that the choice we made regarding \ab{} as a teddy bear was not ideal in the first place.  

Overall, it seemed like all of the children were able to interact with \ab{} as intended. Additionally, it seemed like all of the children found \ab{} enjoyable. 

\subsection{\app{}}

During the user test on AU2, we discovered a critical error that almost rendered the application completely useless. The error originates from the fact that \app{} needs to communicate with the database, and we had not implemented proper feedback to the user that an error had occured. This error should obviously not have occured in the first place, as these types of errors will result in less incentive for parents to use \app{}. 

As for the children, it seemed like all of them liked to use \app{}. However, with the low sample size and the fact that it was children we tested in mind, it is reason to question the validity of this result. On one hand, none of the children had big problems to do what \app{} told them to do and had very few problems navigating the application. On the other hand, the only child capable of reading was CU2, who is seven years old. It should have been given more of an effort to make the process of purchasing a reward even more clear for the youngest users of our target group.     

As far as the gamification system goes, it seemed like children were happy with the fact that stars appear immediately after the treatment is finished. They also seemed happy with the rewards they were given. However, a question that remains unanswered is whether our approach to a gamification system is sustainable over a longer period of time. 

% Critical errors that should have been avoided. 
% Instructions were Understandable for children
% Easy interactions
% Positivt at stjerner kommer med en gang.
% Repeat burde blitt implementert. 
% Flere konkurrenter paa markedet. 