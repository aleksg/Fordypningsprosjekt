\chapter{Results}
\label{chp:results}

\section{Gamification}

\subsection{Bartle's Four Player Types}
[HOW TO MOTIVATE DIFFERENT PLAYERS eller noe]

\section{Tangible Interfaces}


\subsection{Learning}


\subsection{Motivation}


\subsection{Distraction}


\subsection{Information}

\paragraph{Pollution levels, pollen forecast and indoor air quality}

\paragraph{Reminders}

\section{Other Aspects}

\subsection{Helping Kindergartens, Schools and Caregivers}
During our interviews, we discovered a potential problem when children are in the kindergarten or at pre-school. It may occur that the caregivers don't know how to handle an asthma attack properly. According to one of our interview subjects: 

\textit{``The biggest problem is tat the teachers/kindergarten teacher may not have knowledge of what to do when an asthma attack occurs. An application with instructions may be of help to them.''}

From our viewpoint, having a shared \buddy{} in a kindergarten could lead to complexity and problems. First, \buddy{} would have to learn the names of the children in order to keep track of children whose turn it is. Second, there could be no overlapping of treatments, which might become inefficient (depending on the teachers). Third, having a shared \buddy{} in a kindergarten could easily be destroyed. If placed in a kindergarten, \buddy{} in it's current state would probably cause more problems rather than helping the kindergarten teachers. With changes and modifications, we still see the use for tangible interfaces in kindergartens and preschools, as a useful tool to help teachers and children.    

A solution to this problem was provided by a kindergarten teacher we interviewed, who said that it was hard to keep track of which child was supposed to take his/her medicine at the correct time. They sometimes had this information stored on their own phone, or had a note in their pocket. In some cases, no such tool were used, which relies heavily on the teachers' memory. If the teacher forgets it, there is a possibility that the child don't take their medicine properly at the given day. 
The kindergarten teacher proposed an application that allowed parents to register a medicine that was to be taken, and sent push notifications to the teachers, that could remind them of their child's need for a treatment. We concluded that this functionality is out of the scope in this thesis, but we found the idea interesting.         


\subsection{Tangible Interfaces to Help Parents Help Children}
When children suffers from asthma, they often have to rely on their parents in order to maintain control of their disease. Parents have to maintain a clean house and they have to keep an eye on pollen, as pollen and asthma often are related. One of the features \buddy{} could have in order to help parents is a morning cast, informing parents about the weather, pollen distributions and air quality. 

In the future, \buddy{} could communicate with dust sensors, that could indicate whether or not parents actually needed to clean the house. Additionally, \buddy{} could communicate with a Roomba \fnurl{iRobot Roomba}{http://www.irobot.com/us/}, which in turn could start cleaning. \buddy{} could also indicate the air humidity at the child room, starting up the air condition.   

[Pieter: Visjonshistorie?]

\subsection{Do's and Don'ts when Using a TUI}

\paragraph{Mobility}
When developing a TUI for children it is important that the TUI is mobile. Children become attached to their toys and like to take them with them. To make the most out of a tool such as \buddy{} it is important that the children may take it with them. The problem of power usage may be solved by a battery. The problem of recharging can be solved by charging at nights. 

\paragraph{LED lights}
With \buddy{} we tried the use of LED lights to make \buddy{} more interesting than a normal teddy bear. During a treatment the LED light would indicate which medicine was supposed to be taken, by beaming lights in the same color, blink to count the seconds when breathing and using red light to indicate the seriousness of having to find an adult to overview the process. 

[Paavirket det pustingen til barn?]
[Ble barna forvirret?]

One of our interview subjects, a PhD candidate of product design stated in an interview: 
\textit{``People’s perception of and preference for sensory stimuli differs. The use of lights and sound may affect the children in different ways, but that will have to be explored in user studies.''} 

We believe this is a complete field of research on it's own. There already exists some research, such as M\ae hlum's \iref{}, and we have not done enough research to draw conclusions on the use of LED lights. 


\paragraph{Interaction Methods}
\buddy{} in it's prototype form was not able to sense interaction, and was operated by using a ``Wizard-of-Oz technique''\cite{wilson1988rapid}. \buddy{} was not able to give feedback that the user did interaction correctly. The only form of confirmation was that the next sound clip would start playing. This may lead to confusion and uncertainties among users as to whether they interact correctly. [REFERENSE PÅ DETTE???]

[NOE OM INTERAKSJONSMETODER? HAR DET VERDI?]

\section{Feedback from Parents}

[SKAL SKRIVES OM VED FLERE RESULTATER]

We spoke to a mother of a child who have a ``weak'' form of asthma, i.e. he did not have severe attacks very often. This section will summarize some of the findings: 

Boy, about 5 years old
\begin{itemize}
  \item She did not have to ``fight'' with him very often
  \item It was hard to stay on the medicine plan correctly
  \item She missed an application that could easily structure when and how much medicine had been taken.
  \item She found it difficult to meet up at doctor appointments, and remember when she had switched medicine plans. 
  \item In order to calm her child, she would sing or count while he was breathing
  \item As far as rewards go, she had experimented with using bumperstickers on a sheet of paper while giving him toilet training. In her opinion, this was rewarding enough. She did not that our reward system could be useful for other kids though. 3-7 years old is a very large span in terms of cognitive development, and we should have this in mind. 
  \item Alarms were, in her opinion, the most important feature. 
  \item Demonstration worked pretty good, as it calmed the child down. 
  \item She was positive to having a bear at home that could help children take their medicine. However, she would not let him take medicine by himself. 
  \item Sidenote: She is a kindergarten teacher, which implies that she gives asthma medicine to several children. An application that could help kindergarten teachers remind when a specific child was to take their medicine would be much appreciated.
  \item She did not see that it was need for By Need treatments via an application or a teddybear, as it would slow the process down. Children were often stressed out when they had an asthma attack, so having a teddybear \emph{could} help during the process to calm him down. 
  \item Teaching children about their disease could be given more of an effort. However, this was outside the scope of our project. 
  \item Having a shared mobile application across caregivers could be benefitiary, as she often had to explain to give a medicine to others.  
\end{itemize} 



\section{Effects of Gamification in The Treatment Process}
Our solution to gamification is highly coupled with parents' initiative. In order for our reward system to have any motivational effect, parents has to be highly involved. They need to understand how often their child needs a reward, in addition to understanding what defines a ``good'' reward for their children. 

Webster-Stratton and Herbert claims that


\textit{Preschool children aged between the ages of three and four may be rewarded by the special sticker or token itself without needing a back-up reinforcer. Youngsters aged four to six should be able to trade in stickers for something each day if they like. Children of seven and eight can wait a few days before getting a reward. } \cite{webster1994troubled}


\section{How can tangible user interfaces be used to help children with asthma?}
When we first started this project we aimed to look into how tangible user interfaces may be of use for children with asthma. This section presents our findings.


\paragraph{Information spreading and educating asthmatic children and their parents}
[Reasons we believe in this]

\textit{``Preschool children aged between the ages of three and four may be rewarded by the special sticker or token itself without needing a back-up reinforcer. Youngsters aged four to six should be able to trade in stickers for something each day if they like. Children of seven and eight can wait a few days before getting a reward.''} \cite{webster1994troubled}

In order for our AsthmAPP and AsthmaBuddy to be a motivational success, parents have to be aware over their children's maturity. They need to understand the above statement, and use this for their own children. 

\paragraph{Learning}
\textit{``Children below your target group (i.e. younger than 3 years old) can be even harder, as children in the group 3-5 years old has an understanding as to why they need to take their medicine.''}
Children in the age of 3-5 years old understands that they get better from taking their medicine. However, few parents actually tells them specifically what is wrong with them [Sitat?]. \buddy{} could have been used to inform children about what happens with their lungs before and after they take their medicine. 
[TODO: Write more.] 