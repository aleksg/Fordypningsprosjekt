\section{Usability Test Results}
\label{sec:usabilityresults}
In order to protect the identity of our test subjects, especially considering the children, we have used identifiers as names. Names starting with the letter ``A'' is an adult user, and names starting with ``C'' denotes a child.

\subsection{Parent partition tests}
\begin{table}[H]
\begin{tabular}{|p{4.0cm} | p{4.0cm} |}
	\hline
	\textbf{Name} & AU1\\
	\hline
	\textbf{Age} & 36 \\
	\hline
	\textbf{Date} & May 2nd, 2014 \\
	\hline
	\textbf{Testleader} & Aleksander\\
	\hline
	\textbf{Observer} & Esben\\
	\hline	
\end{tabular}
\end{table}

\begin{singlespacing}
\begin{table}[H]
\begin{tabular}{| p{1.0cm} | p{4.0cm} | p{4.5cm} | p{4.0cm} |}
\hline
	\textbf{Task} & \textbf{Problem} & \textbf{Cause} & \textbf{Proposal for solution} \\
	\hline
	0 & The user where unable to separate between the image for child and parent partition & The images were not entirely intuitive & The image for adults could have a bearded man, or other recognizable features. \\
	\hline
	1 & It was unclear whether he had to press on the healthy medicine plan, as the child was in the healthy medicine plan by default. & Challenging GUI & Could make it clearer for the user which treatment plan is being followed \\
	\hline
	1 & The spinners for hour/minutes should be able to be written into. & The standard Android slider contains a minor bug when one writes into it.  & N/A \\
	\hline
	1 & The time that shows when the alarm should fire contained seconds, which the test subject found unnecessary. & N/A & Remove seconds from the timestamp \\
	\hline
	2 & The user expected that he could be able to press the medicine, and not just the checkbox which was the case. He found this a little annoying. & Implementation of listener & Make the entire list item touchable \\
	\hline
	2 & He wanted functionality for adding two medicines at once.  & This has not been implemented yet.  & Implement it later \\
	\hline
	3 & The view shows a button with ``Add Activity'', which he felt was wrong. & This was not intended, as it should have said ``Add Reward'' & Change it to ``Add Activity'' \\
	\hline
	3 & The user pressed the back button one too many times. This caused the PIN-challenge to be presented again, which the test user said could be annoying for some users.  & PIN-challenge appears as soon a parent returns from the parent partition. & Could implement a timer who checks when the user last completed the challenge.\\
	\hline
	4 & The test user said it was not logical to see where he could check the air quality cast. & Bad task description & Change task description to make it clear \\
	\hline
	4 & The test user wanted functionality for different views, for instance showing the log for a week or a single day. & Too high expectations & This feature could be implemented with more time and resources.  \\
	\hline
\end{tabular}
\label{tab:test1}
\caption{Usability result}
\end{table}
\end{singlespacing}

\subsection{Children partition}

During the user tests, \ab{} was configured to use the varied interaction scheme. This involved a preset combination of the remaining interactions from Table \ref{tab:interactioneval}. This decision was made in order to test all of the possible interactions, as we did not have enough users to test them one-by-one. 

\subsubsection{Child User 1}
\begin{table}[H]
\begin{tabular}{| p{4.0cm} | p{4.0cm} |}
\hline
 \textbf{Name} & CU1 \\
 \hline
 \textbf{Age} & 6 years old \\
 \hline 
 \textbf{Date} & May 2nd, 2014 \\
 \hline
 \textbf{Testleader} & Aleksander \\
 \hline
 \textbf{Observer} & Esben \\
 \hline
\end{tabular}
\end{table}

\begin{table}[H]
\begin{tabular}{| p{3.0cm} | p{3.0cm} | p{3.0cm} | p{3.0cm} |}
\hline
	\textbf{Task} & \textbf{Problem} & \textbf{Cause} & \textbf{Proposal for solution} \\
	\hline
	2 & It seemed like she had a hard time keeping up with \ab{}'s instructions & The voice of \ab{} was speaking to fast & Record sounds with lower speed \\
	\hline
	4 & It was difficult to drag the medicine above the mask in order to start the treatment & The treatment only starts when the medicine is directly above the mask & Should make this functionality simpler to start.  \\
	\hline
	4 & It was hard for the child to keep ut with the voice of the rabbit. & The voice talked to quickly, and \app{} does not have a repeat functionality when a treatment is running. & Should consider implementing repeat.\\ 
	\hline
	5 & It was hard to get a clean read of the RFID tag. & It was not entirely clear where the user had to put the card in order to get a read. & \ab{} should have an indicator as of where the card should be held in order to be read.  \\
	\hline
\end{tabular}
\label{tab:test2}
\caption{Usability result}
\end{table}

CU1 was very shy when arriving at the test lab. It quickly became clear for us that we had to leave the area and rather observe from the back room, in order for her to speak up. The parent was instructed with the tasks that were to be performed, and he explained the tasks to her. Once we were back stage, she started responding to the instructions given. We made a note that the observer should sit in the back room and observe from there, in order for the children to respond more easily.   

When asked which method CU1 preferred, i.e. \app{} or \ab{}, CU1 replied ``I don't know''. CU1 was also asked if both were equally fun, which to CU1 replied ``Yes''. CU1 was also asked if the usage of \app{} and \ab{} was more fun than a regular treatment, which to CU replied ``Yes''\footnote{There is reason to belive that the answer was biased due to the reward}. We asked if this was because of her reward, which was candy, but we were unable to get a reply. 

\subsubsection{Child User 2}
\begin{table}[H]
\begin{tabular}{| p{3.0cm} | p{3.0cm} | p{3.0cm} | p{3.0cm} |}
\hline
	\textbf{Task} & \textbf{Problem} & \textbf{Cause} & \textbf{Proposal for solution} \\
	\hline
	place & place & place & place \\
	\hline
	place & place & place & place \\
	\hline
	place & place & place & place \\
	\hline
	place & place & place & place \\
	\hline
	
\end{tabular}
\label{tab:test1}
\caption{Usability result}
\end{table}

\begin{table}[H]
\begin{tabular}{| p{3.0cm} | p{3.0cm} | p{3.0cm} | p{3.0cm} |}
\hline
	\textbf{Task} & \textbf{Problem} & \textbf{Cause} & \textbf{Proposal for solution} \\
	\hline
	place & place & place & place \\
	\hline
	place & place & place & place \\
	\hline
	place & place & place & place \\
	\hline
	place & place & place & place \\
	\hline
\end{tabular}
\label{tab:test1}
\caption{Usability result}
\end{table}

