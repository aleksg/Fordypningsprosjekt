\chapter{Scenario and tasks}
\label{app:scenarioandtasks}

Slett AsthmAPP's brukerdata p\r{a} forh\r{a}nd.

\emph{Forklar hvorfor vi ikke har kalt medisinene Flutide, Seretide og Ventoline, men at vi har kalt de etter fargekoder}.

\section{Foreldre}

Du er verge til et barn p\r{a}  4 \r{a}r. Dere har nylig v\ae rt hos en lege med spesialkompetanse p\r{a} barnesykdommer.
Barnet har f\r{a}tt diagnostisert astma. For \r{a} enklere holde orden p\r{a} at medisinene blir tatt til riktig tid, p\r{a} riktig m\r{a}te, 
har du lastet ned applikasjonen AsthmAPP. Systemet har ikke behov for at du registrerer navn eller lignende, 
for du \o  nsker ikke at slik informasjon kommer p\r{a} avveie. For at du best mulig skal kunne benytte deg av AsthmAPP, er det
n\o  dvendig at du gjennomf\o  rer noen oppgaver.

\paragraph{Oppgave 0:}
G\r{a} inn p\r{a} voksendelen av applikasjonen. Du vil bli bedt om \r{a} lage en PIN-kode. Sett denne til ``1111''. 

\paragraph{Oppgave 1:}
Du skal sette opp en medisineringsplan i henhold til anbefaling fra legen. I f\o rste omgang skal du kun sette opp
medisineringsplanen som gjelder n\r{a}r barnet er frisk. Legg til en varsel for ``Oransje'' kl 13:37 og en varsel for ``Lilla'' kl 18:30.
Velg deretter \r{a} f\o  lge denne medisineringsplanen.


\paragraph{Oppgave 2:}
Barnet ditt tok allerede en dose med medisin da dere var hos legen. Du \o nsker \r{a} starte loggf\o ringen med en gang. Etterrigstrer en dose
av ``Lilla'' for dagens dato.


\paragraph{Oppgave 3:}
Du \o nsker \r{a} motivere barnet ditt til \r{a} ta medisinene ved hjelp av en bel\o nning.
Lag en premie barnet ditt kan f\r{a} dersom hun/han har fulgt medisineringsplanen sin.
Som premiebilde kan du velge en fotball fra ``Standardbilder''. Premietekst kan du sette til ``Fotballbillett'' og sett antall stjerner til 3. Premien skal ikke gjentas. 


\paragraph{Oppgave 4:}
Se hvordan barnet ditt ligger an i forhold til m\r{a}let du satte i forrige oppgave?


\paragraph{Oppgave 5:}
Du \o  nsker \r{a} sjekke hvordan luftkvaliteten er i dag, f\o r barnet skal p\r{a} fotballtrening. Se i loggen om du b\o r ta hensyn til luftkvaliteten f\o r du sender barnet av g\r{a}rde.  

 
Du har n\r{a} gjennomf\o rt testen. Vi vil gjerne sp\o rre deg noen sp\o rsm\r{a}l om hvordan du opplevde dette. \emph{Aleks stiller sp\o rsm\r{a}l fra Appendix D.}

\section{Barn}

Selv om dette er en oppgaver som g\r{a}r ut p\r{a} \r{a} ta medisinen din, s\r{a} er ikke dette en ekte medisin, s\r{a} den vil ikke ha noen effekt. Hvis du ikke har lyst til \r{a} gjennomf\o re testen kan du bare si i fra. 
 
\paragraph{Oppgave 0:}
Dette er Blipp. Han har f\r{a}tt astma han ogs\r{a} , akkurat som deg. Du kan hilse p\r{a}  han og sp\o rre om han er klar for \r{a} leke litt. 

\paragraph{Oppgave 1:}
Kan du sjekke hva du kan tjene p\r{a}  \r{a}  ta medisinen din i fremtiden. Se i lekebutikken om du finner noen premier. Hvor mange stjerner trenger du for \r{a} kj\o pe deg premien?

\paragraph{Oppgave 2:}
N\r{a}  er det p\r{a}  tide \r{a}  hjelpe Blipp med \r{a}  ta pustemedisinen hans. Dersom du er klar, kan du klappe han p\r{a}  hodet. For \r{a}  hjelpe han, m\r{a}  du gj\o re akkurat som han forteller deg.  

[Kj\o r til Blipp har tatt medisinen sin]

\paragraph{Oppgave 3:}
N\r{a}  er det din tur til \r{a}  ta pustemedisin. F\o lg instruksene til Blipp videre. Det kan hende at du vinner noen stjerner som du kan kj\o pe premie med!

\emph{Demonter medisin fra masken}


\paragraph{Oppgave 4:}
N\r{a}  ringte det en alarm p\r{a}  mobilen som sier at du m\r{a}  ta en medisin til. Kan du gj\o re dette igjen, og gj\o re som kaninen p\r{a}  telefonen sier til deg? Telefonen vil ogs\r{a}  gi deg stjerner n\r{a}r du har gjort det. 

\paragraph{Oppgave 5:} Bj\o rnen passer p\r{a} selv om du ikke tar medisinen sammen med bj\o rnen. Ved \r{a} holde dette kortet foran magen hans, kan han minne deg p\r{a} hvor mange stjerner du har samlet. Kan du holde kortet foran magen hans?

\paragraph{Oppgave 6:}
N\r{a}  har du samlet mange stjerner. Sjekk i butikken om du kan kj\o pe deg en premie, og kj\o p den hvis du har r\r{a}d.  


\begin{itemize}
  \item Synes du dette var mer moro enn n\r{a}r du gj\o r det til vanlig?
  \item Hvilken likte du best? Bj\o rnen eller kaninen?
  \item Kunne du tenkt deg \r{a}  leke med bj\o rnen en annen gang? 
\end{itemize}

\section{N\o dplaner}
\paragraph{Vi tar med oss f\o lgende utstyr som kan roe ned barna}
Tegnesaker, spill p\r{a}  mobilen. 

\paragraph{Hva gj\o r vi dersom barnet ikke har lyst til \r{a}  leke med bj\o rnen?}
Sp\o r foreldre om de kan kj\o re en gjennomgang for \r{a}  demonstrere for barnet at det ikke er noe farlig. 
Dersom foreldre ikke vil, pr\o ver vi selv \r{a}  kj\o re en gjennomgang for \r{a}  demonstrere at det ikke er noe farlig. 
 
\paragraph{Dersom str\o m/internett-ledninger g\r{a} r ut}
``Blipp er dessverre ikke noen superhelt, s\r{a}  han er avhengig av str\o m for \r{a}  fungere, akkurat som en TV eller lampe. Han har det helt fint, vi m\r{a}  bare koble han opp igjen. Det er ikke noe farlig. I mellomtiden kan du tegne litt p\r{a}  papirene dine''

\paragraph{Dersom barna ikke gjennomf\o rer testen, og vi m\r{a}  avbryte:}
Gi de et ``gavekort'' p\r{a}  3 stjerner, som kan cashes inn. 

\paragraph{Hva gj\o r vi dersom barna er helt ville?}
Sp\o rre foreldre om de kan roe de ned? I verste fall, sp\o rre om de vil vente p\r{a}  siderommet og tegne litt.

\section{Bakromsoppgaver}
Ha to terminalvinduer oppe (T1 og T2)

\paragraph{T1}
Kj\o r f\o lgende kommandoer:
\begin{itemize}
  \item \$ ser
  \item \$ node index.js  
\end{itemize}
La serveren kj\o re gjennom hele brukertesten.
\paragraph{T2}
\begin{itemize}
  \item \$ blopp
  \item \$ ./v2.sh CN  
\end{itemize}
C er her fargen p\r{a} medisinen som barnet har, N er sekvensnummer som skal kj\o res. N = {0-9}

Dette vil kj\o re Blipp-scriptet, og vil terminere n\r{a}r et barn er ferdig. ./v2.sh CN m\r{a} kj\o res p\r{a} nytt n\r{a}r et nytt barn ankommer. 

  
