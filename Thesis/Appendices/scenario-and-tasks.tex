\chapter{Scenario and tasks}
\label{app:scenarioandtasks}

[Remember to delete the user data from \app{} before we start]. 

\emph{Remember to explain to the test person why the medicines are named ``Purple'', ``Blue'' and ``Orange''}.


\section{Test for adult users}
\label{sec:parentstest}
You are the guardian of a child who has recently been diagnosed with asthma. In order to keep track of taking the medication on the right time in the right manner you have downloaded the application \app{}. The system does not require you to register your name or any other personal information, since you do not want this information to be obtained by others. In order to make the best possible use of \app{}, you must complete some tasks. 

\paragraph{Task 0:}

Navigate to the adult partition of the application. you will be asked to register a PIN-code. Set this to ``1111''.


\paragraph{Task 1:}

You shall now set up a medication plan according to the recommendation from the doctor. To start you must set up the plan for the ``Healthy'' health state. You can start by adding an Orange medicine to 1:37PM and a Purple medicine to 6:30PM. Thereafter, choose to follow the ``Healthy'' treatment plan.


\paragraph{Task 2:}

Your child already took a dosage of medication when you were at the doctor's office. In order to start the logging, you must add the use of a Purple medicine to the log. 


\paragraph{Task 3:}

In order to motivate your child to take his/her medicine, you wish to add a reward he/she may get when he/she has followed his/her treatment plan for some time. 
Navigate to the menu for adding rewards and add a reward named ``Ticket to a football match''. As a picture you may choose the picture of a football from the menu of standard icons. Set the number of stars to three stars. The reward shall not be repeated. 


\paragraph{Task 4:}

Look for information about your child's progress towards getting the reward you just listed. Tell us what information you find.

\paragraph{Task 5:}

You wish to check the air quality readings for today, before your child leaves for football practice. Take a look in for the air quality readings and check if you should take preventive measures before going to practice.  

You have now completed the test. We would like to ask you some questions regarding the use of the application. 

\emph{Test leader asks questions from Appendix \ref{app:interviewafter}.

\section{Test for child users}
\label{sec:childtest}

Even though this test will require you to take a dose of medicine using the inhaler, this is not real medicine. This is a placebo inhaler which only contains compressed air, and it will not have any effect on your asthma. If you do not wish to continue the test, it is okay, and you do not have to continue if you do not want to. 
 
\paragraph{Task 0:}

This is \ab{}. He has recently been diagnosed with asthma, and since he is a bear, he needs some help taking his medication. Can you say hi to him and ask if he is ready to complete his treatment. 

\paragraph{Task 1:}

If help \ab{} completing his treatment, you may earn a reward. Can you take a look in the toy shop in \app{} to take a look for rewards. How many stars do you need in order to buy the reward?


\paragraph{Task 2:}

Now it's time to help \ab{} taking his medication. If you are ready to help him, you must clap him on the head. In order to help him, he will instruct you on how to do it. 


\paragraph{Task 2.5:}

Now it's your turn to take some medicine. Follow the instructions given by \ab{}. Maybe you will earn some stars for taking your medication.

\emph{[Disassemble the mask and medication before continuing. Then set an alarm using \app{}]}


\paragraph{Task 3:}

Soon an alarm will ring on \app{}. You will need to follow the instructions given on the \app{}. If you follow the instruction, you will be awarded with more stars. 


\paragraph{Task 4:} 
\ab{} keeps track of your stars even if you do not take the medicine together with him. By holding your card in front of his belly, \ab{} will tell you how many stars you have collected. Can you try if it is working?


\paragraph{Task 5:}
Now you have collected many stars. Check in the shop if you have earned enough stars to buy a reward. If you have earned enough, you may choose your reward and buy it. 


\section{Emergency plan}
\label{sec:emergencyplan}
In order to make sure the children are not intimidated by the test lab, we tried to limit the amount of unfamiliar persons in the test room together with the children. We also brought coloring pencils and mobile games the children could play with if they were bored.  

\paragraph{What will we do if the children do not want to play with \ab{}?}
- We ask the parents if they want to finish a treatment in order to demonstrate for the children that it is not dangerous.
If the parent does not want to demonstrate, we will demonstrate by ourselves. 
- We tell the children that they do not need to take the medicine, but they can give it to \ab{} twice instead.
- If the children seem shy, we will leave the room leaving the child and parent alone in the room, and follow the test using cameras. 
 
\paragraph{If power-/internet-cord is disconnected from \ab{}}
We tell the children that \ab{} is no super-hero, and that he needs power to function properly, like a television or a lamp. He is not damaged, and we just need to put the cords back in their place. In the meanwhile the children will have to wait. 


\paragraph{If the children do not finish the test, and we have to abort}
If the children do not wish to finish the test, we give them a gift card for three stars. The gift card can be exchanged for a reward on their way out. 

\paragraph{What will we do if the children start crying/become frustrated?}
We ask the parents if they can talk to the children and try to calm them down. In the meanwhile we leave the room and turn off the cameras recording. 

  
